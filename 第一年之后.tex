\section{第一年以后}
\subsection{ME51015 实习}
%王昊辰
\subsection{TUD4040 Joint Interdisciplinary Project}
JIP项目是一个企业合作的联合项目,项目时间在Q1,持续十二周左右,申请时间一般在前一个学期的5到6月,可以作为代替实习/额外课程的15学分。项目的宣讲会和Q4组织的各个实验室的宣讲一起进行。项目内容覆盖不同的领域,具体可以前往项目官网/bs上的课程页面查看。报名流程在线上进行,一般来说都会尽量分配第一志愿,基本没有报名但没办法被选中的情况。理论上小组组成之后就不能退出了,实际操作上在七月初之前提出要退出都是可以的。项目要求fulltime,实际工作上和队友商量好的话,可以在这期间参加1-2门课程或是开始literature review,但是要注意考试时间和final review的时间可能会撞上。项目在第一周举行problem definition presentation,在第六周举行midterm presentation,这两次都是不计分,面向相同领域的小组和他们的导师,主要提供交流的空间。评分基于final review和报告,主要是基于前者。同时每两周需要提交小组blog和个人的成长报告,这部分准时提交不要太敷衍一般都会给接近满分(占20\%)。项目工作量和工作形式每个小组都不同,一般来说工作量适中。项目周期很短,成员和企业的导师沟通确定项目方向和预期结果是最初的工作重点。有兴趣的话可以在毕设继续和企业导师合作,接着JIP项目或是开启新项目都是可行的。

\begin{flushright}
刘彦菁; 08/06/2023
\end{flushright}
\subsection{IFEEMCS520100 Fundamentals of Artificial Intelligence}

\subsection{毕业项目}
%王昊辰