\chapter[下辑:生活杂谈]{下辑:\\生活杂谈}

\section{公共交通\&交通卡}
荷兰的公共交通系统,有些类似日本;各种轨道交通间并没有泾渭分明的界限。火车的运作方式有些类似大陆地铁,电车又时而跑去地下。各种交通系统也都仅仅需要一张OV Chipkaart(OV卡)就可以互通换乘。

OV卡,分为三种:匿名,实名,以及商务。

匿名卡的运作方式和上海紫卡及大多数大陆普通交通卡类似,不记名;需要7.5欧工本费押金;除开坐火车的Joint Journey Discount 以外没有其它优惠。2023年的某日起。普通芯片银行卡,包括荷兰许多地方不支持的Visa或MasterCard芯片卡;都可以实现匿名OV卡的大部分功能的。使得匿名卡略显鸡肋。

久居荷兰的话,个人觉得其实还是自己办理一张实名OV卡比较划算;作为荷兰痕迹的纪念品回国,笔者觉得也挺有意义的。实名卡也需要工本费,且需要拥有荷兰银行卡及BSN才可以办理。它的优势在于可以在交通公司网站上购买诸多优惠套餐,经常出行的话比较划算。当然,其实估计大家也发现了;许多个人会私下里出租自己购买了优惠套餐的交通卡;这的确可行,但又实在违规。自行判断,不做评价了于此笔者就。

至于商务卡,若是二年级实习的公司较为慷慨的话;可能会提供。但大多数学生其实比较难接触到吧。

刚刚来到荷兰,若是没有芯片银行卡;可以在手机APP上购买火车票(其实APP购票似乎还经常有优惠)。不同于英国或大陆的是,荷兰的火车完全按照距离计费,火车公司并不在意你乘坐坐哪条线路,花多长时间抵达(除了一些额外标注收费的特快线路)。一般情况下,荷兰火车站需要刷公交卡、银行卡、实体票或是APP二维码过闸机。但某些站点,如诸位极有可能抵达荷兰的第一站史基普机场;是没有闸机的。如此情况下,刷卡的话不要忘记在一根黄色柱子上上进站;否则就会在出站时候,默认你从火车的首发站点上车,花费整条行路的大价钱。但若是手机购票或是实体购票,就直接上车好了;无需担心。

以上泛泛关于一般意义的公共交通;但除此,就出租车与OV Fiets笔者还有些牢骚。就以下再单独拎出来两个小标题稍微讲一下罢。
\begin{flushright}
王昊辰; 05/06/2023
\end{flushright}

\subsection{出租车}
之所以拎出租车出来单独讲;主要还是源于Delft独特的出租车生态吧。其他欧洲城市非常通行的Uber或Bolt之类的网约车,在Delft非常不灵,或者更准确的说,是完全不可用。笔者经验,2022年8月28日到达Delft火车站,Uber苦等1小时+不见任何车子踪迹。后来也和司机聊过,似乎是Delft本地的出租车工会组织之类,认为uber这类平台的抽成太高了;于是集体抵制。和很多店面不支持MasterCard和Visa Card一样地缘由。

如此,在这样的情境下;在Delft应该如何打车呢? 真的很古早的方式,其实就是电话或者邮件提前预约。这里我就附上我来到Delft时候,\href{https://taxidelfland.nl/}{\uline{司机所在公司的名片网站吧}}。

当然,其实预约也不是特别必要。火车站附近,和大陆一样的是,在工作日的工作时间,往往是有许多出租车聚集在站点附近等着拉活的。在Delft的话,出租车会聚集在火车站旁边公交车场的一侧。值得注意的是,荷兰的出租车并不像大陆会有明确的样式:它们没有统一的涂装,也没有统一的立牌竖在车顶(事实上好多立牌在车顶的车会是驾校的训练车)。荷兰出租车与私家车的唯一区别,就是他们的车牌照会是深蓝色的底色。于普通私家车,牌照会是深黄与白色作底。

而且,鉴于荷兰的出租车,若是轿车,车型往往较为高档(似乎奔驰E级为普遍)。两个28寸的行李箱,若是后备箱塞不下;司机很有可能不会同意你把行李箱放在后座。但这里也时常有类似Uber XL的MPV作为出租车在运营,所以果然若是有MPV,还是将它们作为第一选择吧。但荷兰的出租车不同于大陆许多城市与香港,并没有所谓的行李搬运费;司机帮忙搬运行李是没有必要付小费或任何额外费用的,放心叫他们帮忙就好。
\begin{flushright}
王昊辰; 07/06/2023
\end{flushright}

\subsection{公共自行车}
游玩别的城市,或是友人来访。可能也需要接触到公共自行车的租借出租吧。荷兰广为使用的公共自行车服务是OV Fiet的,其中Fiet也就是Bike的荷兰语,它是覆盖几乎所有荷兰火车站的,以实名OV卡进行脚踏车日租的服务。收费以目前2023年6月而言,是4.45欧元24小时。

值得注意的是,这项服务是需要和NS的其它套餐一样需要在网上提前订购并机器上加载的。这里的订购并不是说提前购买预付多少
小时的租车额度;而是指对OV Fiet这项服务进行开通;仅仅开通并激活服务的话,就是字面意思的0欧元一年。

租的车子,富的话完全可以跨站点甚至跨城市归还。例如Delft租的车子,在海牙HS归还也是可以的;只不过会收取10欧元的手续费。而租借,归还流程本身,因为车型各有不同,实在多种多样。就Delft站而言,在2023年6月这个节点,所有的车子都是机械锁,租借归还完全人工。
\begin{flushright}
王昊辰; 07/06/2023
\end{flushright}
\vspace{\betsubsec} %section间留白
\section{私家自行车}
于Delft,拥有以自行车为代表的二轮交通工具,可以极大地拓展行动半径。或早或晚地,每个留学生都大概率会长期对一台车辆进行占有。自行车,电助力车还是电瓶车?购买还是长租?新车或是二手?这部分将分为三个小节于此进行阐明。

\subsection{法规}
在荷兰,以人力驱动的车辆,无论三轮二轮,以什么姿势进行操作,用什么部位来驱动车辆;都能够算进普遍的,字面意义的自行车的范围,是完全无需驾照即可上路行驶的。

而给自行车加上汽油机或是电动机以后;继续地划分便会逐渐模糊开来了。泛泛讲,粗糙地划分可以归于两类,最高时速25km/h以下(且于燃油动力而言排量在50cc以下)的,及25km/h以上的。前者,毫无疑问地,无论何种助力方式,都会被划入摩托车的范畴,需要持有驾照按照机动车管理,悬挂黄色牌照,走机动车道。后者最高理论时速25km/h以下的,便是事情真正复杂的地方。

按照官方法律划分,最高时速25km/h以下;若是Twist and Go类型,及在没有人体做功时候,仅仅激发油门就能开始奔驰的车辆,会被化为moped。这类交通工具管理类似国内电瓶车,允许行走非机动车道,但需要上牌(蓝色牌照)。与国内不同的,是荷兰的moped,还需要AM驾照才可以驾驶。于荷兰视角,AM驾照能在取得汽车驾照,或者大于A1的摩托车驾照时候自动获取,因此与荷兰成年人并没有增加额外的驾照负担。但于大概率不会拥有本地汽车驾照的留学生,也就意味着蓝色牌照车辆,至少在刚来的时候,是无法合法上路的。

若是非Twist and Go类型,即需要在人体做功的情况下,助力系统才能够生效的车辆(助力车)。荷兰法律将它们视作寻常自行车,一样不需要任何证件及牌照。如此看来似乎这样的法律非常清晰,可问题是Twist and Go需要如何界定,助力车只是助力,那如果骑车人速度超过了25km/h如何是好?官方条款里,允许非Twist and Go车辆,以方便残疾人为目的,在人体不做功的情况下低速自启动,其实又给助力车留下了非常大的空子。

也就因此,荷兰道路上时常有一些,大多由外卖员驾驶的,轮胎宽大,风驰电掣的助力车。他们可以轻松恣意在非机动车道上超过25km/h的限速(毕竟可以argue助力系统只是助力,是人天生神力才猪突猛进),且完全无需任何证件即可驾驶。这也就是或许在2022-2023学年里,时常有新闻出现的,需要大力整治,可能会专门立法禁止继续上路的电助力车。

但是于留学生视角,其实这完全无需担心。助力车价格本来就不菲(二手价格300-1000欧元比较正常),那类狂飙突进车辆就更是昂贵,正常情况应该也不会纳入留学生的考量范围。笔者个人觉得,若是想购买助力车,则荷兰本地人常见式样,与普通自行车外形几无二致的助力车,是能够非常放心购买的。
\begin{flushright}
王昊辰; 04/07/2023
\end{flushright}

\subsection{购买}
Delft本地商家众多,同样功用的二手车在不同商家处,甚至相差数倍。全新车辆价格昂贵,但比较透明,新生季也会是许多留学生的安全选择。

全新车里面,迪卡侬的价格较为廉价且透明,且有着可以轻易购买的\hyperlink{礼品卡}{\uline{礼品卡}}。笔者身边统计学,买新车的留学生们十之八九是采购自迪卡侬。可是,迪卡侬于Delft并未有门店,虽说坐一号线可以直达海牙迪卡侬门口,但回程,要么网上采购送货上门(需要自己动手安装),要么只有骑车奔袭,长征回家。

个人之间的交易也是常见的选择,各大二手交易群,于大陆留学生而言会是常见的管道 ;而在新生周的时候,有一天的中午是二手自行车交易,可以提前过去看看有没有自己心仪的自行车。荷兰人常用的二手平台,\href{https://www.marktplaats.nl/}{\uline{Marktplaats}}也是甚好的交易途径。

除此以外也可以前往Delft的二手自行车店进行购买,谷歌地图搜索'second hand bike' 或‘refurbished bike’ 即有众多选项。但请千万注意价格,一般行情来看,7速内变速的自行车,在200欧以下,是算比较正常的价格;更高就实在有些不划算了。

在所有二手店里,笔者购买过比较推荐的,是Roland附近的‘Brik-Fit’;这是一家慈善机构,残疾人会在里面,培训如何翻修自行车等,收入也全部归于慈善机构本身。这家价格十分合理,甚至不乏50欧元以下的自行车出售。但值得注意,这家店铺一周只有周一至周三营业;且理论上只收刷卡,但事实上,由于里面有不少志愿者,在表明自己刚来到荷兰,还未有本地银行卡时候,志愿者可能会收下你的现金,而后用Ta的私人银行卡帮你完场支付。车辆购买后会提供六周的免费维修,注意保存购买小票作为售后凭证。
\begin{flushright}
王昊辰; 04/07/2023
\end{flushright}

\subsection{租赁}
用的最多的租自行车的就是Swapfiets(街上前轮是蓝色的自行车都是从这家租的),网址是‘https://swapfiets.nl/’,代尔夫特的门店在代尔夫特的火车站附近。租车的好处是不需要担心维修,如果有需要修的时候直接送到他们门店就能修,如果损坏到无法骑行的程度还可以派车去取,代尔夫特常规修自行车的成本都比较高,建议去学生群里问问有没有有偿修车的。而且租车不需要担心被偷的问题,在不是因为你主观因素造成自行车丢失的情况下(没有锁车或把自行车出借),你不需要支付任何费用,你只需要报警取得回执,就可以再领一辆新的。(为什么会有人去偷一辆正规出租公司出租的有编号的自行车,都没法销赃。)

他们出租的常规自行车分为Original和Deluxe 7两个档次。Original是无变速脚刹(需要反向蹬踏板才能刹车),一个月月租19.9欧元。Deluxe 7是有七个变速档并同时有脚刹和手刹,一个月月租23.9欧元。个人推荐Deluxe 7版,主要是脚刹比较难操作,手刹在紧急情况下更容易使用,而且变速在上桥或大风天还是比较有用的。而且这两个版本就差4欧元/月。Swapfiets还出租电动自行车,如果能解决充电问题,也是可以考虑的。

如果想租车,可以去他们的官网,输入自己的信息并选择车的版本并预定取车的时间。你可以选择按月付钱,可以随时取消,也可以一次性支付6个月的钱,在6个月内不能取消合同,6个月之后按月付钱,这种方法能稍微省点钱。去找正在用swapfiets的同学要个邀请码,双方都能得到优惠。之后去可以他们的门店取车试车并签订租赁合同。
\begin{flushright}
张先治;07/07/2023
\end{flushright}

\vspace{\betsubsec} %section间留白
\section{荷兰交规}
来到荷兰,自行车大概率会成为通勤的主要手段。然而,完全不同于大陆行车地无比混乱;荷兰的交通,是约束着非机动车,且规则十分明确的。搞清楚路权地优先级,十分重要;相对应路口没有做出理应的让行,很容易发生危险,或是被荷兰人破口大骂。 以下,笔者非常主观地列举了印象中的优先级,其中排序越是靠前便越是优先;并不保证绝对正确,但至少在实践中运用十分顺利。

\begin{enumerate}
\item 紧急服务车辆:执行任务时的警车、消防车、救护车。
\item 交通信号和道路标志
\begin{enumerate}
    \item 紧急警告标志:警告驾驶员紧急情况,如事故或道路封闭。
    \item 禁止和限制标志:如“禁止通行”、“禁止左转”或“速度限制”。
    \item 优先权标志
    \begin{enumerate}
        \item "让行"标志(鲨鱼鳍):当鲨鱼鳍的尖头对准你所在车道时候,必须让别的车辆先行 (或许是最需要记住的条款了)。
        \item "停"标志:必须在此停车,确保没有即将进入交叉口的车辆。
    \end{enumerate}
    \item 告示标志:如“单行道”或“无自行车道”等。
    \item 信息标志:如路名、目的地指示等。
\end{enumerate}
\item 行人:在人行横道上。
\item 自行车和摩托车:在自行车道和自行车交叉路口。
\item 连续道路:在交叉口,如果某条道路的路面颜色或图案是连续的,这通常意味着该道路的车辆有优先权。
\item 右侧来车:在没有交通信号或特殊路标的情况下,车辆应让右侧来的车辆先行。
\item 环形交叉路口:在无交通信号灯控制的环形交叉路口,已在环内的车辆具有优先权。
\item 小转弯(右转)和大转弯(左转):如果两个车辆同时到达交叉口,并且一个车辆打算右转,另一个车辆打算左转,右转(小转弯)的车辆优先。
\item 原有车道:在合并车道、超车或进出停车位时,原有车道的车辆有优先权。
\end{enumerate}

另外也需注意,对于成队的车辆,如军车车队,或是统一涂装的丧葬队伍等。它们在荷兰的交规里往往被视为一辆特别长的车。如此,比方十字路口,对方车队还在行驶,红绿灯却已经变化了,此刻车队并不会中途停下来让行,必须要让车队全部通过。

而对于骑自行车,除开路权的优先度;非常值得注意的是,在荷兰骑自行车看手机,哪怕只是掏出来看一下导航;及天黑后没有不开尾灯,都是违法的。这两项似乎常有国际生被处罚,罚金听说高达150欧。
\begin{flushright}
王昊辰; 04/06/2023
\end{flushright}

\vspace{\betsubsec} %section间留白
\section{必要杂货购买}
这一部分,首先将简要介绍荷兰一些可以集中进行此类物品购买的门店。这并不会覆盖所有Delft售卖此类商品的店家,只会抛砖引玉些知名连锁或是网站。除开这些连锁店,其实也有些很不错的私人店铺,例如Roland附近De Hoven内,就有名叫Lion的一家较大的床上用品专营店。但这类店铺,鉴于实在仅便利于特定住宅区附近,未见多少代表性,笔者这里就只能遗憾遗漏了。

而除开门店的相关讯息,也会提及某些,例如中餐调料等于华人而言或许重要的商品,于代尔夫特而言的购买管道。

\subsection{宜家家居}
荷兰,作为宜家商业意义的母国(被荷兰人控股了),宜家门店密度还是比较高的;代尔夫特本地,离学校3公里左右就有一家比较大的门店。荷兰的宜家,和大陆的布置装潢,购物体验,商品等等完全一样,甚至会员卡都是通用的。价格据笔者观察,相交国内略有上调,但差价也不会特别大,总体在荷兰物价下,算是极为物美价廉的。

但是大件物品前往门店购买,果然回程的运输会比较麻烦。和大陆一样,荷兰的宜家也可以通过APP或网页进行订货,送货上门;不过送货价格相对较高。笔者印象里,送到楼下运费需要20几欧元,运送进家门价格会更贵;不过如此价格总体还是基本与打车回家持平的。

值得注意,要是希望送货上门,笔者经验,在8月返校的高峰期,宜家的某些商品及送货时间都会比较紧俏。如果希望于宜家购买且送货上门,或许可以考虑出发前还在家里时候,就提前网上下单进行预定吧。

宜家的礼品卡,整个欧洲地区都是通用的,甚至英国的英镑礼品卡都可以按实时汇率在荷兰使用。要想节省,其实可以去大陆一些二手平台购买欧洲地区的礼品卡,价格会是实际汇率的八折左右。关于礼品卡的使用等,会在\hyperlink{礼品卡}{\uline{后续小节单独叙述}}。
\begin{flushright}
王昊辰; 01/07/2023
\end{flushright}

\subsection{Action杂货}
Action是一家专门售卖各种廉价杂货的门店,环境较为脏乱差;颇有种21世纪初笔者小时候,大陆地区私人超市的感觉。它的价格极为低廉,同样品牌同型号商品会相较其它超市低出一截。但这类商品在Action内占据少数,其门店多数位置还是为其自营商品或是贴牌产品占据。

不同于AH等超市以食品为主,Action相当大的空间留给了五金,小家电等小商品。质量往往相较别处略差,但价格确实为其它门店同类商品的甚至五折或更低。笔者个人认为,适合于此购买一些本就不会十分常用的家具电器,例如电动打蛋器,打气筒之类。
\begin{flushright}
王昊辰; 01/07/2023
\end{flushright}

\subsection{HEMA杂货}
HEMA,是一家完全指售卖自己贴牌商品的公司。笔者个人观感,有些类似于线下版本的荷兰网易严选或是京东京造。价格不高不低,质量不高不低,产品涉猎不可不叫广博,从冷鲜的蛋糕到自行车的轮胎都可以找到。它在市中心及Roland附近都有门店,离Delft主流的国际学生聚居区都不是很远,算是闭眼去买普普通通能用商品的不错地方。
\begin{flushright}
王昊辰; 01/07/2023
\end{flushright}

\subsection{TK maxx折扣店}
TK maxx是一家专门售卖各种打折(大牌)尾货的折扣商场,在Delft并没有门店,但其海牙店就在1号电车站台旁边;坐电车能够非常轻松地抵达。

由于是折扣店,其内部除开各类衣服鞋子,也会有专门售卖厨具,电器,及床上用品的区域。个人感觉它相对于宜家,同样价格质量会更好许多,甚至时常能用寻常货色价格买到高端产品。只是又是因为本质折扣店,并不很摸得透它具体当天会有什么商品陈列。若闲情逸趣,的确可以碰碰运气。
\begin{flushright}
王昊辰; 01/07/2023
\end{flushright}

\subsection{亚马逊(荷亚,德亚,英亚)}
亚马逊,应该是在荷兰仅次于Bol的最大电商了;充值Prime后多数商品白天购买即可第二天抵达。它的价格相对于实体门店往往相对较高,但如此差价实际上,如果不介意\hyperlink{礼品卡}{\uline{礼品卡}},可以被礼品卡的差价抹平。个人觉得亚马逊比较适合一些品牌家电或是数码产品。

荷兰亚马逊,由于近些年刚刚成立的原因,商品比较有限;但英国亚马逊,德国亚马逊都可送货至荷兰;只是多数情况下速度略慢,且需要支付4欧元左右的运费。各个亚马逊对于同样商品价格往往并不相同,有些时候就算将运费纳入考量,德亚或英亚仍会具有价格优势(当然也有时候是荷亚价格更低)。

快递配送方面,荷兰亚马逊多数时候会自己承运或是由DHL或PostNL代运。荷兰快递,时有寄丢,或是送至楼下被别人偷走的情况;这些时候不用担心,联系亚马逊客服他们会立刻进行退款或是补发,无需任何证明。亚马逊可以说在欧洲的售后是极好的,笔者购买两年的压蒜器损坏,或是购买一年的iPhone数据线断裂,依然售后全额退款;甚至有次135欧元的冰箱亚马逊发错,直接让我自行保留,又给我补发了一个正确的型号。

当然实在担心快递失窃的话,亚马逊可以选择寄送到代收点或是快递柜;代收点类似大陆的菜鸟驿站,只是多数由便利店或是书店等兼职运营,需要注意它们的营业时间,总得来说还算比较方便。
\begin{flushright}
王昊辰; 01/07/2023
\end{flushright}

\vspace{\betsubsec} %section间留白


\section{礼品卡}\hypertarget{礼品卡}{} 
礼品卡,其实多少涉及些灰色交易。具体从何而来的细节,笔者也只能不负责任地捕风捉影自己猜测。不过无论如何,其的确是能够以相对低廉汇率购买的。但也要注意,使用礼品卡存在风险;市场规律也说明了,折扣越高,往往风险越大。笔者体验也的确如此,亚马逊的折扣,无论荷亚,英亚,还是德亚都可以低至原汇率的甚至六折;相对应的也是相对较高的封号概率,这点与Steam比较类似。而迪卡侬,宜家,Zalando等原汇率八折附近的礼品卡,一般来说风险就甚至可以忽略不计了。

这些卡要从何购买呢?截至写下词条的今日,由于亚马逊,Zalando礼品卡市场比较广大,大陆橙色软件会有大量店铺经营此类生意。某个店内消费后,大概率可以进去专门群聊或是添加店家私人联系方式。于是,就有可能可以在橙色软件之外,在绿色聊天软件内,订购一些相对小众的礼品卡了。

值得注意,宜家,迪卡侬,Zalando,Uber Eats等的礼品卡;全欧洲都是通用的。而亚马逊或是TK maxx等,则仅限于发行国家。许多礼品卡全欧洲的通用性,也带来了一些更为稳健的曲线礼品卡购买方式,笔者以下就以Wunschgutschein为例说明。

Wunschgutschein是一种德国的福利卡,有些类似大陆地区常见的华泽微福等。相较于其它礼品卡,这类礼品卡来源比较透明;据笔者购买处的店家讲述,它和国内一样,大多源自企业福利,或是银行、保险公司等对客户逢年过节或是生日的贺礼。又和国内一样的,是有许多这类卡片会流入二级市场。这种卡片的折扣往往在原价八五折左右,但可以兑换于包括迪卡侬,宜家,Zalando等在内的许多平台,非常通用,且风险较低,是笔者个人经常使用的一类礼品卡。

亚马逊的礼品卡,鉴于其高风险高折扣,笔者这里单独拎出来讲述一下。其实坦白讲,高风险这个词语并不十分合适;笔者身边人为样本,个人估算其实亚马逊因礼品卡导致封号的概率甚至不到十分之一。且由于亚马逊会提供次日达服务,不在账户内囤积余额,随用随冲,大概率就算封号,也已经是拿到商品以后了。

话虽如此,大多数如上所述礼品卡,作为末端用户,并不知情其来源,且不负责任的猜想和一些购买过程里的奇怪之处多少说明它们或许与非法行为有所关联;使用礼品卡,从这点上讲,道德性是可能存疑的。具体的使用与否,也请权衡风险道德后自行的判定罢。

除开上文所述需要从大陆跨国购买的礼品卡们。也有一类礼品卡,打折程度极低(指并不较原价便宜多少),大多原价九折以内;但这类卡却有着非常正规的来源,没有道德风险,极为稳健;笔者个人认为属实不薅白不薅的程度。

这类礼品卡源于返利网站;欧洲许多店家的商业模式,会有着广告返利佣金,于是便诞生了如英国的Top Cash Back之类的返利平台。其大致商业逻辑,简单讲大概就是将商家返还给其的推广佣金与用户共享。它平台下有着一个名为\href{https://www.topcashback.co.uk/top-gift-cards/}{\uline{Top Gift Cards}}的礼品卡平台,专门出售扣除返利后价格的礼品卡。由于这是一家英国公司,购买前还请详细查看相关店家的礼品卡政策,确保礼品卡在荷兰的可用性(荷兰的类似公司笔者还没有找到)。于2023年这个节点,笔者亲身实践可以使用的是迪卡侬的礼品卡。还请注意,由于\href{https://www.topcashback.co.uk/top-gift-cards/}{\uline{Top Gift Cards}}理论上仅仅服务英国,购买时候结算货币为英镑,推荐使用N26等英镑汇率较为划算的银行卡进行结算;且结账时候需要填写英国地址(可以问英国朋友借用,或者直接填写谷歌地图上随机的学生公寓)。

此外,荷兰本国也会有一些礼品卡交换网站,但由于其本质是从用户手中收购礼品卡再转售,个人感觉安全性并不见得会比橙色软件高多少,加之折扣较低,大多仅在原价九五折附近(2023年而言),笔者个人并未有多少使用。不过对于世界范围内比较小众的荷兰本土购物网站,例如Bol等,这类网站会是唯一的去处。其中比较有名气的是\href{https://en.wissel.nl/}{\uline{wissel.nl}}。

\begin{flushright}
王昊辰; 01/07/2023
\end{flushright}

\vspace{\betsubsec} %section间留白

\section{超市优惠——以AH为例}
荷兰的超市,感觉相比大陆;优惠花头经多得多,且往往优惠也会更在实在一些;不盯着优惠或者打折,属实纯纯富姐(哥)了。打折标签,种种优惠活动之类到底什么意思;虽然漫漫生活里自然就会熟悉;然而果然还是先讲述一下比较好吧。如此刚来到荷兰的日子,也可以少花点冤枉钱,逛超市时候也不至于一直手机搜索,移动路障。

由于笔者光顾Roland 楼下的Albert Heijn (AH)比较多些;这里就主要以AH举例了。Delft主流超市,也就是Dirk, AH, Jumbo了;这三家打折标签大同小异;不过AH的APP优惠词条确实仅仅适用于AH。此外也有Lidl,Aldi,麦德龙之类的;笔者并未曾前往多少次所以并不了解。

需要注意的是,AH和Jumbo的优惠,都必须在办理免费的会员卡的前提下才可以被探用,Dirk是不需要的。AH的会员卡称之为‘Bonuskaart’ (Bonus Card); 而Jumbo的一般称为Jumbo Extra's。跟服务台营业员说一下就可以当场免费领取;绑定手机APP以后可以查看积分,以及探索更多APP上面的优惠。另外AH的APP也可以提前在周三时候看到下周的优惠讯息,提前规划购买。

各超市的会员卡,也能方便使用Hand Scanner 进行结账。Hand Scanner于大陆笔者想不出来对应的产物;大概就是一个扫码枪,一路挑选一路直接扫码:总价,优惠,和你买的东西都会由此显示到其屏幕上,一目了然,不过也不要忘记不想要的东西放回去了在扫码枪上删除。使用扫码枪的话,偶尔会在结账时候被收银员随机抽查;大概就是他们再扫一遍看看你有没有蒙混;身正不怕影子歪。

各个超市Hand Scanner的操作略有不同; 租借都是在机器上扫描会员卡;但结账归还,AH是先归还扫码枪,再在自助收银及扫描会员卡。而Jumbo,则是先使用扫描仪扫描自助收银机的二维码,再归还扫码枪。 另外AH的扫码枪更加智能一些,饮料瓶退换的押金条可以直接用扫码枪扣除,结账总经额也会因此甚至自动变为负数;但Jumbo的押金条,却只能在收银机上扫描。

AH的Hand Scanner \href{https://www.youtube.com/watch?v=7MSqE_vt5Po}{\uline{教程可点此处查看(荷兰语,但youtube有自动翻译)}}。
\begin{flushright}
王昊辰; 04/06/2023
\end{flushright}



\subsection{常见优惠名词}
AH的优惠,一般都会十分醒目的以橙色标牌标出。除了一般优惠以外,还会经常有临期食品-35\%的贴纸,这种额外优惠非常简单易懂;这里也就主要讲讲橙色优惠里面,较为常见的荷兰语名词了。

\begin{flushleft}
\begin{tabular}{p{0.185\textwidth}|p{0.75\textwidth}}
\textbf{荷兰语} & \textbf{中文解释} \\ \hline
Korting & 就是cutting的意思。例如 "€1 Korting" 意味着该商品在原价基础上减1欧元。 \\ \hline
Gratis & 直译为免费,一般会跟其它东西搭配描述;比如表示购买一打啤酒可以免费获得一个杯子什么的。偶尔也会有新产品的小样什么的,放在那里标个Gratis;免费让你拿了试吃试用。 \\ \hline
2e Halve Prijs & e是序数词标记;可以直译为2nd half price。顾名思义,第二件半价。 \\ \hline
n+1 Gratis & 如前所述Gratis意为免费,此处意思就是买n件送1件;值得注意,比如买1送1的话,不是说只要扫描1件就好了,还是需要扫描2件商品的;只不过第二件会自动价格扣掉免费。 \\ \hline
Op=Op & 无关原友。OP这里可以理解为最后机会;这里意思就是,最后机会=最后机会;卖完再也没有了。约等于大陆清仓优惠。 \\ \hline
x voor €y & voor, 读音就可看出,意思是x for €y;买x件总价只要y欧元。 \\ \hline
x voor y & 偶尔也可以看见这种;这里的x与y一般都是整数,不带小数点也不带欧元标志。这里的意思就是x件商品只算y件的价格。例如3 voor 2, 其实和2+1 Gratis一个意思。 \\ \hline
Prijsverlaging & 价格降低,表示商品已经被永久或长期减价。 
\end{tabular}
\end{flushleft}
\begin{flushright}
王昊辰; 04/06/2023
\end{flushright}

\subsection{APP优惠:Bonus Box}
Bonus Box; 是AH的APP上最醒目的。简单来说,就是AI会猜你喜欢;在一般优惠以外,可以在这里勾选额外的5样你想要的专属优惠;如果氪了会员,则有10件。
\begin{flushright}
王昊辰; 04/06/2023
\end{flushright}

\subsection{APP优惠:Koopzegels}
Koopzegels这个东西蛮复杂的;很多人甚至会没发现自动开通了这项优惠,导致每次给超市白白多送10\%的钱。

Koopzegels,翻译过来,意思大概就是“集邮”或者印章。有点像小时候的奶茶店,买一杯贴一个贴纸,贴满多少个贴纸送你一杯奶茶的工作原理,但又有所不同。简单来讲,开通了这个优惠,在AH每消费1欧元,就会自动多花0.1欧元购买一个Koopzegel(也就是多花购物金额去除小数部分的10\%)。

当你集满490个Koopzegel时;AH就会溢价8\%(2023年6月)回购你持有的Koopzegels。换言之,即在AH每消费满大约490欧元,就可以一次性体现53欧元,AH会多送你4欧元。在手机APP上可以实时追踪你的集邮进度。提现操作一般就去服务台就可以了,可以选择现金或者打到你的银行卡上。
\begin{flushright}
王昊辰; 04/06/2023
\end{flushright}
\vspace{\betsubsec} %section间留白

\section{银行卡}
在Delft,一般比较常见外国人办理的线下银行就是三家:ING,ABN以及Rabobank。Rabobank和ING都在Delft有线下支行,但是ABN没有。三家各有优劣吧。Rabobank的支行比较大,但是它的APP没有英文版,官网也是。ING有支行,而且个人觉得学生套餐的政策比ABN好一些,卡面也是纯橙色(笔者感觉)比较好看;但ING办理需要在市政厅注册时花费10.99欧元开具Uittreksel BRP (也有传闻说2023年新学期可能不需要了)。这也是额外成本,总的来说各有优劣吧。

荷兰因为支付系统独树一帜,虽然在逐渐改进,但在许多地方仍然不接受Visa或MasterCard;因此,拥有一张当地的Vpay或Maestro卡还是很有必要的。正因如此,如果还没有Visa或MasterCard芯片卡,去欧洲其他国家时可能会陷入无卡可用的境地。为了解决这个问题,可以另外申请一张网上银行的Visa或MasterCard卡。在荷兰,选择并不是很多,截至目前(2023年6月),常见的选择有BUNQ、N26和Revolut。虽然Revolut似乎使用的人比较多,但笔者主观上并不太推荐大额使用:与N26、BUNQ或Curve相比,它的欧元兑英镑、瑞士法郎、兹罗提汇率都不太理想。如果要用瑞郎购买奢侈品,3000法郎就能差十几二十欧元了。然而,Revolut的好处在于可以同时开设多种货币的钱包,这在接收潜在的英国同学转账给你的英镑等方面具有一些独特的作用。

鉴于网上银行的卡开设几乎零成本,甚至还有时会有返现活动,笔者个人觉得开设多几个也无妨。然而,线下银行的开设确实需要一些手续,这里简要介绍一下线下银行的银行卡开设流程。
\begin{flushright}
王昊辰; 04/06/2023
\end{flushright}

\subsection{ING银行}
在Delft,ING一共有2个网点。其中一个位于市中心火车站对面,非常显眼;另一个位于Roland附近的一家书店里面。柜台甚至不如超市试吃摊位,非常隐蔽,而且这个网点的ING员工似乎并不全职,对于开卡的很多问题她自己也不清楚。建议直接去火车站那家网点。

ING开卡的话,很可能需要使用到前文提到的Uittreksel BRP。这个文件在你注册时和市政厅的工作人员交流一下,他们就会了解。这是一张克数比较大的A4纸,上面是你的身份信息,触感有点类似奖状。带上这张BRP,以及学校的注册证明(如果打算在抵达时就去办卡,建议提前在大陆打印好;当然也可以在宿舍找好心人帮忙打印),再加上护照,就足够去开卡了。

在荷兰,ING的营业厅和大陆的风格完全不同,更像是一个咖啡厅,没有窗口也没有长长的队伍。进入营业厅,直接找个空位子坐下来就好。8、9月高峰期,或许可能会有指示牌写明学生需要坐在哪里等待开卡;如果有指示,按照指示行事,如果没有指示,随便找个座位坐着就行。大厅里还有咖啡机,可以给自己泡杯茶。就像在餐厅一样,等待营业员们忙完手头的顾客,他们自然会前来接待你。

然后就是与营业员交流,并按照流程开卡。但是有一点需要特别注意。在开卡过程中或开卡结束后(具体我不记得了),可能营业员会要求你使用手机的NFC扫描护照以进行激活;然而,如果你使用的是国行版本的iPhone,其NFC是无法读取护照信息的。这个问题在欧版手机上并不存在,所以大多数营业员并不会考虑到这一点。如果遇到问题,还请一定记得寻求帮助。

开卡后,几个工作日之内,你的卡片应该会寄到你的邮箱。收到卡片后,不要忘记将套餐改成学生套餐,这样可以免除每月几欧的手续费,并享有12次免费提款。

此外,血泪教训;还有一点额外特别注意:ING的密码是4位的,这是不可更改的!荷兰的ATM机通常不会提供更改密码的选项。然而,在海外,以笔者的经历为例,英国的ATM机可以更改ING的密码。可是一旦你手贱操作,更改密码,整个卡的信息都会乱掉。结果像笔者一样,旧密码和新密码都无法使用,直接废卡一张了;最好只好申请新卡补办,非常麻烦。请务必记住不要随意更改密码!
\begin{flushright}
王昊辰; 04/06/2023
\end{flushright}
\subsection{ABN AMRO银行}
ABN AMRO的储蓄卡是全程线上办理的不需要去线下网点(虽然它也没有线下网点),全程在它的app上操作。完成注册之后,它会分批次的把电子密码器,银行卡以及PIN码邮寄到你的邮箱。之后在电子密码器上激活使用。
\begin{flushright}
张先治; 07/06/2023

\end{flushright}


\vspace{\betsubsec} %section间留白
\section{医疗与医保——以Insure2Study体验为例}
荷兰的学生保险有很多选择。大多政策都会非常清爽地写明;这里笔者也就强调一点很多官网模棱两可,关于旧病的政策方面吧。如果说,在来到荷兰之前,就有着一些固有疾病(除了近视)的复查或后续治疗,除了学校推荐的insure 2 study外,其它国际生里比较流行的OOM与AON都是不会提供报销的。

此部分的GP,以SGZ为例;而医院,笔者仅仅光顾过代尔夫特的Reinier de Graaf Gasthuis。 这里的就诊体验,仅能代表笔者于2023年上半年,携insure 2 study保险的经历;并不代表其它。

关于急诊,笔者并未经历;但较为严重如骨折等,大概可以直接前往Reinier de Graaf Gasthuis(\href{https://reinierdegraaf.nl/specialisme/spoedeisende-hulp}{\uline{见此链接}})。不严重,但GP不上班有点急,可以求助于\href{https://hapschievliet.nl/language/english/}{\uline{Huisartsenpost Delft}}。


\subsection{全科医生}
在Delft,若是注册在SGZ,预约还是比较方便的:拨打015-7999050按提示操作。SGZ采取回拨制度,首次拨打以后,会在有空时给你回电。笔者的经验,回电往往半小时到一小时内就能接到。在电话里,告诉对方自己的基本情况,对方就会安排预约时间了。个人感觉SGZ的看病流程很快,大概率能预约到两三天以内,甚至第二天或当天的时间段。

到达预约时间时,需要在前台确认身份并付款。用于报销的材料,工作人员会用订书机为你装订好。回家后,扫描这些材料并附上填好的Insure2Study的报销单,通过邮件发送给保险公司即可完成报销操作;保险公司一段时间后会直接将钱打到银行卡上。

在SGZ,付好钱。当预约的时间到来时,医生会把你请进办公室。这方面和大陆没有太大分别。如果情况复杂,可能会进行转诊;转诊所去的大医院大概率就是代尔夫特的Reinier de Graaf Gasthuis。医院会给你发送邮件,进行一些基本的注册和身份的确认;而后会有信函寄送到你的信箱,按照信件所示时间前往就行了。
\begin{flushright}
王昊辰; 04/06/2023
\end{flushright}

\subsection{大医院}
携带医院所寄信函到达Reinier de Graaf Gasthuis,首先要进行就诊卡的注册。医院很大,堪比3me教学楼的迷宫,建议首次前往最好提前至少15-30分钟。 注册就诊卡的地方在大门进门的左手边,比较隐蔽,可以向前台询问,让工作人员进行指引。就诊卡的注册需要先取号,取号机器完全荷兰语,同样建议咨询工作人员。

拿到就诊卡以后,前往此前信函上所指示的科室。要注意,到达科室以后也需要在机器上进行签到;这里的机器又是完全荷兰语,且需要进行一些较为复杂的身份信息确认,每个界面甚至还有倒计时,手机翻译其实不太来得及。笔者强烈建议这步拉下脸皮寻求附近工作人员的帮助。注册完成后,就是根据所拿到的号码等待进入科室了。由此,就和大陆看病没有什么不同了,只不过等待时间完全不会像大陆那么长罢了。

从诊室出来,有可能要向前台或其它地方进行一些其它方面的操作。这里因人而异,一定记得和医生问问清楚哇。
\begin{flushright}
王昊辰; 04/06/2023
\end{flushright}

\subsection{配眼鏡——以Specsavers为例}
其实就算没有保险,笔者个人感觉在荷兰配眼镜还是相当划算的:验光相比国内专业很多(至少比JINS验光项目多很多);虽然相比国内小眼镜店价格略贵,但其实也与JINS或ZOFF等连锁店价格在同一水平线。尤其对于高度数人群,1.74的镜片价格比较实惠(于2023年6月,260欧2副pentax的1.74)。

若是有了保险,价格优势会更加明显些。以Insure2Study为例,在满足保险要求的情况下;可以单次报销150欧元,价格足以购买2副Specsavers的普通镜框和普通1.61镜片了。但满足保险,其实也有一定要求,主要可以概括为如下三项:

\begin{enumerate}
\item 需要有相关证明,证明你在本次配镜以前就近视了。
\item 距离上一次配镜,时间已经过去超过一年。
\item 相较上一次眼镜的度数,这次屈光度上调超过0.5(50度)。
\end{enumerate}

这三份材料可以是同一文件,也可以单独文件;只要足够证明就好。Insure2Study并不要求这些文件需在荷兰开具;笔者自己所使用的就是2021年在英国读书时候的验光配镜单。但笔者也并不清楚大陆或别的地区以华文写就的材料是否能被Insure2Study顺利认识;可能可以自己咨询一下客服吧。

配镜,至少Specsavers是需要预约的;但slot非常疏松。基本工作日一小时后的slot都是空的,属于可以一拍脑袋就去配镜的程度。Specsavers的镜框与镜架需要单独购买,镜架价格在29欧元到239欧元不等;镜片也是丰俭由人。但这些价格,无论镜片还是镜架,至少在2023年6月这个时间为止;其实都是双份价格,买一送一;也就是最终你都会收到两幅眼镜(镜架在相同价格和系列可以不一样)。
\begin{flushright}
王昊辰; 08/06/2023
\end{flushright}


\vspace{\betsubsec} %section间留白
