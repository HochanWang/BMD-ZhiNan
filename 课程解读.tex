\section{课程解读}
BMD第一年,理想状态下需要依照Form2的要求,修满60学分的课程;但54学分修满,则已经可以开始第二年的graduation project。而若仅为保留在tud的学生身份,则仅仅30学分即可。此部分下,将依照Form2对课程的划分,极其主观第一方地对各课程进行评价与评分,当然评价部分,有话则长,无话则短。相关编者也会附上一些2022-2023学年(若有其它学年历史讯息则也一并奉上)里,每门课首次考试里学校提供的量化数据,及在某些课程下尽可能地附上些但愿有用的帮助材料。

关于评分,相关编者将会采用蜘蛛图的形式,就知识是否有用或丰富, 授课质量, 作业难易程度, 课程的易通过性, 主观感受到的工作量, 及相关编者自己所感受到的有趣程度进行打分;分数越高,总是说明于学生更加友好:即作业更少,考试越简单,工作量很低,知识非常有用等。

也有一些课程;或是以为时间距离对应部分作者落笔时间太过遥远,印象实在模糊;于是也就只会提供量化数据,而没有主观评分及评论了。

以及,对于BMD而言;Q4的必修课有3门,且其中ME41055与ME41006两门,至少在2022-2023学年工作量都极大;因此不是很建议在Q4选择太多选修课了,否则很难兼顾(不过2022-2023学年已经通过AvL向学校反馈,就是不知道日后的学年里会不会改进)。笔者个人感觉,至多6学分的额外选修已经人类极限了。

\begin{flushright}
王昊辰; 05/06/2023
\end{flushright}

\subsection{示例,若是只有数据}
\begin{minipage}{\textwidth}
\centering
\begin{tabularx}{\textwidth}{l|X|X|X|X}
\textbf{ } &\textbf{通过率} & \textbf{平均分} & \textbf{参与学生数} & \textbf{学分} \\ \hline
数据21-22 & 100\% & 7.08 & 91 & 3 EC \\
\end{tabularx}
\end{minipage}

\vspace{\betsubsec} %section间留白
\section{ME Core Courses}
\subsection{ME41055 Multibody Dynamics B}\hypertarget{ME41055}{} 
\begin{minipage}{0.45\textwidth}
\centering
\begin{tikzpicture}
\begin{polaraxis}[
    width=0.65\textwidth,
    height=0.65\textwidth,
    yticklabels={},
    xtick={0,60,...,300},
    xticklabels={知识性, 授课质量, 作业友好,工作量适宜,易通过, 有趣},
    ymin=0,
    ymax=5,
    ytick={1,2,...,5},
    y grid style={gray},
    grid=both,
    minor grid style=gray,
    major grid style={gray},
]
\addplot[mark=*, ultra thick, black, data cs=polar] coordinates {
    (0,5)  % 知识性得分,这里修改为你的分数
    (60,5)  % 授课质量得分,这里修改为你的分数
    (120,1)  % 作业友好得分,这里修改为你的分数
    (180,1)  % 工作量得分,这里修改为你的分数
    (240,2)  % 易通过得分,这里修改为你的分数
    (300,4)  % 有趣得分,这里修改为你的分数
    (360,5)  % 为了闭合图形,最后的一个点应该和第一个点一样,这里修改为你的分数
};
\end{polaraxis}
\end{tikzpicture}
\end{minipage}%
\begin{minipage}{0.45\textwidth}
\raggedleft
\begin{tabular}{r|c|c}
\textbf{ } & \textbf{22-23} & \textbf{21-22}\\ \hline
通过率 & 58.17\% & 81.1\%\\ 
平均分 &   4.54& 6.75\\ 
参与学生数 &  153& 111\\
学分 & 4 EC & 4 EC\\
\end{tabular}
\end{minipage}\\

这门课的老师是美国人,2021-2022学年刚刚从UC Davis跳槽来TUD;感觉算是比较好说话的,教学质量也很不错。但是,这门课几乎可以说是整个第一年里,workload最大的一门课了;它每周都会有\href{https://drive.google.com/file/d/1bz5jCnF8Walknjwm1NXwLhsIZyFFHFMX/view?usp=sharing}{作业},Q3作业量是正常的,但到了Q4,很轻易地就可以\href{https://drive.google.com/file/d/1bz5jCnF8Walknjwm1NXwLhsIZyFFHFMX/view?usp=sharing}{每周在作业}上花费十余小时。不过,今年已经通过AvL及课程反馈会议向学校进行了反馈,明年或许会有所改进吧。

其实公平讲,Workload负担大也不全是这个老师的过错。这门课的打分机制,在2022-2023学年是这样的:考试占了最后总评分的60\%,\href{https://drive.google.com/file/d/1bz5jCnF8Walknjwm1NXwLhsIZyFFHFMX/view?usp=sharing}{每周在作业}40\%。但是,当你最终考试的成绩高于平时成绩的话,平时成绩就不会被考虑。这个老师辩解,如此设置是为了确保每个学生,就算你考试发挥不佳,最终成绩也不至于太难看。他扬言既然最终成绩既然总归需要及格,由此就算是不做作业也是没关系的。

的确承认,他讲的其实是有道理的,他说每周的作业并没有被设计地需要全部被完成,也无需全部完成。但作为学生,至少在Delft,我遇到的无论哪国人,每人似乎每周都会打磨作业到全部正确为止。或许是希望能由此尽可能地,在就算最终考试发会不佳的情况下分数也不至于太过于难看?

但这门课的内容,确实受益匪浅。它的内核理论并不深奥,其中大多数在本科阶段的Dynamics课其实多少会被提及;从牛顿欧拉,到凯恩方程,到拉格朗日和TMT法。就是一系列的解决运动问题的方法吧。但至少不同于我本科阶段的,完全用笔和纸解决比较异常简单的考试问题;这节课是完全基于编程计算,而一半以上的教学其实也是在教你如何操作。主要是用Python,探用Simpy,NumPy,SunDial等等包来计算复杂的动态;后期还会要求你把这些模拟以影像形式呈现。

这个老师似乎不是很喜欢用BrightSpace,教学内容的录像会被发布在Youtube,而所采用的教学材料,是他自己做的一个\href{https://moorepants.github.io/learn-multibody-dynamics/}{\uline{网站}};以章节地形式叙述了我上述的全部内容。而这些内容,除开编程,基于一本1985年的已经绝版的经典教材,但相比那本经典教材,个人感觉他网站上的文字会更加通俗易懂,非常说人话。甚至笔者个人认为,在有一定基础的情况下,他的网站甚至算是运动学最好懂的教材之一了。就算不选这门课(虽然BMD而言是必修),也十分值得一读。

考试的话,2022-2023学年看,是电脑考试:10道选择及两道大题;其中选择和平时作业区别不大,大题一道考验Constrain,一道Kane方法,难度不小。
\begin{flushright}
王昊辰; 06/07/2023
\end{flushright}

\subsection{ME41106 Intelligent Vehicles 3ME}\hypertarget{ME41106}{} 
\begin{minipage}{0.45\textwidth}
\centering
\begin{tikzpicture}
\begin{polaraxis}[
    width=0.65\textwidth,
    height=0.65\textwidth,
    yticklabels={},
    xtick={0,60,...,300},
    xticklabels={知识性, 授课质量, 作业友好,工作量适宜,易通过, 有趣},
    ymin=0,
    ymax=5,
    ytick={1,2,...,5},
    y grid style={gray},
    grid=both,
    minor grid style=gray,
    major grid style={gray},
]
\addplot[mark=*, ultra thick, black, data cs=polar] coordinates {
    (0,4)  % 知识性得分,这里修改为你的分数
    (60,5)  % 授课质量得分,这里修改为你的分数
    (120,3)  % 作业友好得分,这里修改为你的分数
    (180,1)  % 工作量得分,这里修改为你的分数
    (240,5)  % 易通过得分,这里修改为你的分数
    (300,4)  % 有趣得分,这里修改为你的分数
    (360,4)  % 为了闭合图形,最后的一个点应该和第一个点一样,这里修改为你的分数
};
\end{polaraxis}
\end{tikzpicture}
\end{minipage}%
\begin{minipage}{0.45\textwidth}
\raggedleft
\begin{tabular}{r|c}
\textbf{ } & \textbf{22-23} \\ \hline
通过率 & 90.7\% \\ 
平均分 & 7.57 \\ 
参与学生数 & 107 \\ 
学分 & 5 EC\\
\end{tabular}
\end{minipage}\\

该课主要讲述了自动驾驶的相关技术技术,包含了Detection,Localization 和 Trajectory Planning 3个方面,是一门对自动驾驶的很不错的入门课(同时也意味着课程的深度并不高)。该课的主讲人Dr.Gavrila 之前一直再奔驰做自动驾驶, 所以该课不管从有趣性和授课质量上都属于上乘。同时该课还是机器人学院的Machine Perception 的姊妹课,两门课的内容有2/3的重叠,所以不能同时选择。听一听 Machine perception 的Guest Lecture也是蛮有意思的。

该课的作业也属于比较有意思的,可以再作业中实现一些Detection以及Localization的算法,并配有很不错的可视化,对于理解很有帮助。但是由于作业需要用线上的Jupyter Hub做,所以有的时候服务器,网络的问题也是真的恶心(这点在做大作业的时候体现的极其明显)。

由于这门课是机器人学院开的,所以通过率实属不错(机器人学院的课基本付出了时间认真做很难挂人),中国人普遍能得到8分左右(图示数据为Machine Perception的课程通过率和平均分, 但两者基本相近)。但是机器人学院的课普遍工作量很大很占时间(平时作业+大作业+考试),所以会导致考期前两周比较忙。

\begin{flushright}
马润禹; 03/06/2023
\end{flushright}

\subsection{ME44210 Drive \& Energy Systems}\hypertarget{ME44210}{} 
\begin{minipage}{0.45\textwidth}
\centering
\begin{tikzpicture}
\begin{polaraxis}[
    width=0.65\textwidth,
    height=0.65\textwidth,
    yticklabels={},
    xtick={0,60,...,300},
    xticklabels={知识性, 授课质量, 作业友好,工作量适宜,易通过, 有趣},
    ymin=0,
    ymax=5,
    ytick={1,2,...,5},
    y grid style={gray},
    grid=both,
    minor grid style=gray,
    major grid style={gray},
]
\addplot[mark=*, ultra thick, black, data cs=polar] coordinates {
    (0,2)  % 知识性得分,这里修改为你的分数
    (60,4)  % 授课质量得分,这里修改为你的分数
    (120,5)  % 作业友好得分,这里修改为你的分数
    (180,5)  % 工作量得分,这里修改为你的分数
    (240,5)  % 易通过得分,这里修改为你的分数
    (300,3)  % 有趣得分,这里修改为你的分数
    (360,2)  % 为了闭合图形,最后的一个点应该和第一个点一样,这里修改为你的分数
};
\end{polaraxis}
\end{tikzpicture}
\end{minipage}%
\begin{minipage}{0.45\textwidth}
\raggedleft
\begin{tabular}{r|c}
\textbf{ } & \textbf{22-23} \\ \hline
通过率 & 80.0\% \\ 
平均分 & 7.07 \\ 
参与学生数 & 155 \\ 
学分 & 3 EC\\
\end{tabular}
\end{minipage}\\

这门课,是关于电机,变压器等的基础知识,会大概介绍各种常见电机,AC-DC转换,变压器等的原理和计算。若是在高中物理阶段学过选修3-1的话,会觉得学起来非常简单,但同时也就意味着所得到的知识不是很多吧。

它的考试非常容易通过,老师会在很早的时候就放出非常多得往年试卷,其超高的通过率和较高的平均分足以说明。

总结讲,一门非常好的用来水学分的课。

\begin{flushright}
王昊辰; 28/05/2023
\end{flushright}




\subsection{ME45042 Advanced Fluid Dynamics}\hypertarget{AFD}{} 
\begin{minipage}{0.45\textwidth}
\centering
\begin{tikzpicture}
\begin{polaraxis}[
    width=0.65\textwidth,
    height=0.65\textwidth,
    yticklabels={},
    xtick={0,60,...,300},
    xticklabels={知识性, 授课质量, 作业友好,工作量适宜,易通过, 有趣},
    ymin=0,
    ymax=5,
    ytick={1,2,...,5},
    y grid style={gray},
    grid=both,
    minor grid style=gray,
    major grid style={gray},
]
\addplot[mark=*, ultra thick, black, data cs=polar] coordinates {
    (0,4)  % 知识性得分,这里修改为你的分数
    (60,5)  % 授课质量得分,这里修改为你的分数
    (120,5)  % 作业友好得分,这里修改为你的分数
    (180,5)  % 工作量得分,这里修改为你的分数
    (240,2)  % 易通过得分,这里修改为你的分数
    (300,4)  % 有趣得分,这里修改为你的分数
    (360,4)  % 为了闭合图形,最后的一个点应该和第一个点一样,这里修改为你的分数
};
\end{polaraxis}
\end{tikzpicture}
\end{minipage}%
\begin{minipage}{0.45\textwidth}
\raggedleft
\begin{tabular}{r|c}
\textbf{ } & \textbf{22-23} \\ \hline
通过率 & 53.4\% \\ 
平均分 & 6.00 \\ 
参与学生数 & 88 \\ 
学分 & 5 EC\\

\end{tabular}
\end{minipage}\\

流体力学这门课,至少在2022年由Dr. Tam执教的时候,是很有趣也很有知识性的。但有传闻说2023年教纲会改掉,而执教的教师也会换掉。

就2022-2023学年来说的话;这门课的Q1部分主要就是在重新讲述一些基础知识,老生常谈的量纲分析,N-S方程推导之类;而Q2部分,就是关于Invisid Flow了,会学习像古早的工程师一样使用Joukowsky Transform,Complex Field之类来计算机翼升力等等; 课程最后也会学习一下Boundary Layer, 提一嘴各种Wave与稳定性。

其授课质量,在我主观看来数一数二。这个老师来自MIT,用的也是MIT的教纲和资料;上课完全没有PPT,纯纯板书推导。美中不足,Dr. Tam的字迹比较难以辨认,可能需要一两周的适应。

关于作业,并不强制,无需上交也不参与评分。作业比较有难度,而且参考答案也不是很清晰。但我的体验是,就算如我一样完全不做作业对最终考试也不会有多少影响。

这门课可能是因为难度的原因,通过率和均分并不是很好看,由此虽然是提供给整个ME学院可选的核心课程,在荷兰人中却并不火爆,许多人会知难而退,这点从相对较少的参与学生数中也可以看出。但我个人觉得,要是对流体力学稍微有点兴趣,都还是很推荐的。

\begin{flushright}
王昊辰; 28/05/2023
\end{flushright}



\subsection{ME46006 Physics for Mechanical Engineers}\hypertarget{ME46006}{} 
\begin{minipage}{0.45\textwidth}
\centering
\begin{tikzpicture}
\begin{polaraxis}[
    width=0.65\textwidth,
    height=0.65\textwidth,
    yticklabels={},
    xtick={0,60,...,300},
    xticklabels={知识性, 授课质量, 作业友好,工作量适宜,易通过, 有趣},
    ymin=0,
    ymax=5,
    ytick={1,2,...,5},
    y grid style={gray},
    grid=both,
    minor grid style=gray,
    major grid style={gray},
]
\addplot[mark=*, ultra thick, black, data cs=polar] coordinates {
    (0,2)  % 知识性得分,这里修改为你的分数
    (60,4)  % 授课质量得分,这里修改为你的分数
    (120,5)  % 作业友好得分,这里修改为你的分数
    (180,5)  % 工作量得分,这里修改为你的分数
    (240,4)  % 易通过得分,这里修改为你的分数
    (300,3)  % 有趣得分,这里修改为你的分数
    (360,2)  % 为了闭合图形,最后的一个点应该和第一个点一样,这里修改为你的分数
};
\end{polaraxis}
\end{tikzpicture}
\end{minipage}%
\begin{minipage}{0.45\textwidth}
\raggedleft
\begin{tabular}{r|c|c}
\textbf{ } & \textbf{22-23} & \textbf{21-22}\\ \hline
通过率 & 64.0\% & 69.7\%\\ 
平均分 & 6.54 & 6.59\\ 
参与学生数 & 197 & 198\\ 
学分 & 4 EC & 4 EC\\
\end{tabular}
\end{minipage}\\

物理,基本就是高中物理选修部分的延申。因为我自己是没有上过国内本科,不太清楚国内的大学物理深度几何。但我幻想,由于TUD的荷兰人们本科阶段,是没有继续在高中物理上精进的,所以这门课的内容应与国内大学物理多有重合。若是国内本科,学完大学物理以后,这门课想必会觉得非常轻松吧。

其教学内容,主要就是一些电磁学,非常基础的量子力学,以及不超过高中物理3-4难度的光学。总的来说并不是一门很难的课。通过率数据看起来不怎么漂亮的原因,大概可能还是教学速度吧。两周四节课,自欧姆定律,猪突讲完麦克斯韦的程度。

这门课没有要交的作业,但每周会有推荐完成的练习。而这些练习内,基于2022-2023学年的试卷来看,会有1,2道原题出现在考试。不过考试难度并不是很高,那些原题,个人实践就算没有做过,考试也是可以完成的。主观总结讲,这是一门放眼整个第一年,都可以认定为非常容易水学分的课程。
\begin{flushright}
王昊辰; 28/05/2023
\end{flushright}


\subsection{ME46007 Measurement Technology}\hypertarget{ME46007}{} 
\begin{minipage}{0.45\textwidth}
\centering
\begin{tikzpicture}
\begin{polaraxis}[
    width=0.65\textwidth,
    height=0.65\textwidth,
    yticklabels={},
    xtick={0,60,...,300},
    xticklabels={知识性, 授课质量, 作业友好,工作量适宜,易通过, 有趣},
    ymin=0,
    ymax=5,
    ytick={1,2,...,5},
    y grid style={gray},
    grid=both,
    minor grid style=gray,
    major grid style={gray},
]
\addplot[mark=*, ultra thick, black, data cs=polar] coordinates {
    (0,4)  % 知识性得分,这里修改为你的分数
    (60,3)  % 授课质量得分,这里修改为你的分数
    (120,5)  % 作业友好得分,这里修改为你的分数
    (180,4)  % 工作量得分,这里修改为你的分数
    (240,2)  % 易通过得分,这里修改为你的分数
    (300,3)  % 有趣得分,这里修改为你的分数
    (360,4)  % 为了闭合图形,最后的一个点应该和第一个点一样,这里修改为你的分数
};
\end{polaraxis}
\end{tikzpicture}
\end{minipage}%
\begin{minipage}{0.45\textwidth}
\raggedleft
\begin{tabular}{r|c|c}
\textbf{ } & \textbf{22-23} & \textbf{21-22} \\ \hline
通过率 & 59.3\% & 66\%\\ 
平均分 & 6.19 & 6.33\\ 
参与学生数 & 241 &186\\ 
学分 & 3 EC & 3 EC\\
\end{tabular}
\end{minipage}\\

这门课,主要就是介绍有关测量的一些技巧吧。老师讲得还算不错的,上课有种拳拳到肉的感觉,并没有几句废话;这样的授课风格,的确可以迅速的获取知识,但某种程度上,却又因为过于集中于所狙击领域,听不见一些课程以外的八卦或有趣的知识。

所教授内容,笔者学完其实都不知如何概括;大概就是测量误差的种种计算,一些信号处理设备,使用光学传感器的实验观测等等。其中混合着的,既十分关乎统计学,又会有快速傅里叶这类十分数学的内容;总之涉猎非常广泛,上课也Delft式地猪突猛进。

考试,比较因人而异;题目并不是那么的死板,和上课例题几无关联。但笔者个人极度主观地觉得,只要做过前几年的题目,搞懂了上课内容,考试成绩都会不错。这门课笔者线下就去过2次,但考试成了第一年里唯一9分以上的科目。但似乎统计学数据,却说明笔者的经验并不是那么地具备代表性。

\begin{flushright}
王昊辰; 17/06/2023
\end{flushright}

根据身边统计学,认识的很多人都对这门课差评如潮,课程内容难度还是可以的,但是考试难度很大,尤其是23年的补考,难度比正考高很多。题目变化极大,把往年考试题做明白不能保证你过,每年都有新的题目类型。
\begin{flushright}
张先治;04/07/2023
\end{flushright}

\subsection{ME46055 Engineering Dynamics}\hypertarget{ME46055}{} 
\begin{minipage}{0.45\textwidth}
\centering
\begin{tikzpicture}
\begin{polaraxis}[
    width=0.65\textwidth,
    height=0.65\textwidth,
    yticklabels={},
    xtick={0,60,...,300},
    xticklabels={知识性, 授课质量, 作业友好,工作量适宜,易通过, 有趣},
    ymin=0,
    ymax=5,
    ytick={1,2,...,5},
    y grid style={gray},
    grid=both,
    minor grid style=gray,
    major grid style={gray},
]
\addplot[mark=*, ultra thick, black, data cs=polar] coordinates {
    (0,4)  % 知识性得分,这里修改为你的分数
    (60,4)  % 授课质量得分,这里修改为你的分数
    (120,4)  % 作业友好得分,这里修改为你的分数
    (180,4)  % 工作量得分,这里修改为你的分数
    (240,4)  % 易通过得分,这里修改为你的分数
    (300,3)  % 有趣得分,这里修改为你的分数
    (360,4)  % 为了闭合图形,最后的一个点应该和第一个点一样,这里修改为你的分数
};
\end{polaraxis}
\end{tikzpicture}
\end{minipage}%
\begin{minipage}{0.45\textwidth}
\raggedleft
\begin{tabular}{r|c}
\textbf{ } & \textbf{22-23} \\ \hline
通过率 & 78\% \\ 
平均分 & 7.04\\ 
参与学生数 & 200 \\ 
学分 & 4 EC\\
\end{tabular}
\end{minipage}\\

这门课的主要内容为力学中的动力学,主要内容有虚位移原理与虚功,完整约束与不完整约束,拉格朗日方程,线性化,稳定性,模态叠加法(Mode Superposition)以及振动。如果有力学基础的话,那么这门课还是比较简单的。

成绩的70\%是期末考试,有选择题和几道大题,这几道大题都有固定的解题思路,把往年的考试题和课上的例题做明白就能做对。

成绩的30\%是报告分,需要组队完成对指定系统的分析,好像已经连续好几年都是手持电钻的模型了,难度不大,报告分基本都挺高的。

\subsection{ME46000 Nonlinear Mechanics}
\begin{minipage}{0.9\textwidth}
\centering
\begin{tabularx}{\textwidth}{l|X|X|X|X}
\textbf{ } &\textbf{通过率} & \textbf{平均分} & \textbf{学生数} & \textbf{学分} \\ \hline
数据21-22 & 52.2\% & 5.95 & 136 & 4 EC \\
\end{tabularx}
\end{minipage}


\subsection{ME46085 Mechatronic System Design}\hypertarget{ME46085}{} 
\begin{minipage}{0.45\textwidth}
\centering
\begin{tikzpicture}
\begin{polaraxis}[
    width=0.65\textwidth,
    height=0.65\textwidth,
    yticklabels={},
    xtick={0,60,...,300},
    xticklabels={知识性, 授课质量, 作业友好,工作量适宜,易通过, 有趣},
    ymin=0,
    ymax=5,
    ytick={1,2,...,5},
    y grid style={gray},
    grid=both,
    minor grid style=gray,
    major grid style={gray},
]
\addplot[mark=*, ultra thick, black, data cs=polar] coordinates {
    (0,2)  % 知识性得分,这里修改为你的分数
    (60,4)  % 授课质量得分,这里修改为你的分数
    (120,3)  % 作业友好得分,这里修改为你的分数
    (180,3)  % 工作量得分,这里修改为你的分数
    (240,4)  % 易通过得分,这里修改为你的分数
    (300,3)  % 有趣得分,这里修改为你的分数
    (360,2)  % 为了闭合图形,最后的一个点应该和第一个点一样,这里修改为你的分数
};
\end{polaraxis}
\end{tikzpicture}
\end{minipage}%
\begin{minipage}{0.45\textwidth}
\raggedleft
\begin{tabular}{r|c}
\textbf{ } & \textbf{22-23} \\ \hline
通过率 & 85.4\% \\ 
平均分 & 7.18\\ 
参与学生数 & 164 \\ 
学分 & 4 EC\\
\end{tabular}
\end{minipage}\\

比较推荐的一节课,主要内容为运动学方程,共振频率,PID控制,电磁驱动器,洛伦兹驱动器以及压电驱动器。

成绩的40\%由是三次作业,三次作业会比较花时间。

成绩的60\%是期末考试,试卷上至少得到35/60并且再与作业加权平均之后大于6才能通过课程。开卷考试,只允许携带指定参考书,参考书在leegwater有卖,58欧元。需要注意的是,允许在书上做笔记并带入考场。

\begin{flushright}
张先治; 07/06/2023
\end{flushright}

\subsection{SC42001 Control System Design}\hypertarget{control}{} 
\begin{minipage}{0.45\textwidth}
\centering
\begin{tikzpicture}
\begin{polaraxis}[
    width=0.65\textwidth,
    height=0.65\textwidth,
    yticklabels={},
    xtick={0,60,...,300},
    xticklabels={知识性, 授课质量, 作业友好,工作量适宜,易通过, 有趣},
    ymin=0,
    ymax=5,
    ytick={1,2,...,5},
    y grid style={gray},
    grid=both,
    minor grid style=gray,
    major grid style={gray},
]
\addplot[mark=*, ultra thick, black, data cs=polar] coordinates {
    (0,4)  % 知识性得分,这里修改为你的分数
    (60,3)  % 授课质量得分,这里修改为你的分数
    (120,5)  % 作业友好得分,这里修改为你的分数
    (180,3)  % 工作量得分,这里修改为你的分数
    (240,3)  % 易通过得分,这里修改为你的分数
    (300,2)  % 有趣得分,这里修改为你的分数
    (360,4)  % 为了闭合图形,最后的一个点应该和第一个点一样,这里修改为你的分数
};
\end{polaraxis}
\end{tikzpicture}
\end{minipage}%
\begin{minipage}{0.45\textwidth}
\raggedleft
\begin{tabular}{r|c}
\textbf{ } & \textbf{22-23} \\ \hline
通过率 & 66.8\% \\ 
平均分 & 6.67 \\ 
参与学生数 & 151 \\ 
学分 & 5 EC\\
\end{tabular}
\end{minipage}\\

几乎是所有 ME 和 EE 的中国学生都会选择的课。这门课基本对于控制系统进行了概述 从最开始的 State Space 稳定性,可控性, 可观测性分析到卡尔曼滤波再到模型预测控制都有涉猎,是一门很不错的控制学科的入门课。就工作量而言,这门课程没有必须交的作业,每两周会有习题并会设有专门的习题课。从个人角度而言,由于是一门纯理论课,所以很难说这门课很有趣。

就考试而言,最优策略是刷近几年的考题。考试难度并不高,但是由于全是选择题且需要答对20题中的14题才能通过,所以极易翻车,需要对计算过程谨慎一些。
\begin{flushright}
马润禹; 03/06/2023
\end{flushright}


\vspace{\betsubsec} %section间留白
\section{ME Social Courses}
\subsection[WM0349WB Philosophy of Engineering Science and Design]{WM0349WB Philosophy of Engineering Science and Design}\hypertarget{WM0349WB}{} 
\begin{minipage}{0.45\textwidth}
\centering
\begin{tikzpicture}
\begin{polaraxis}[
    width=0.65\textwidth,
    height=0.65\textwidth,
    yticklabels={},
    xtick={0,60,...,300},
    xticklabels={知识性, 授课质量, 作业友好,工作量适宜,易通过, 有趣},
    ymin=0,
    ymax=5,
    ytick={1,2,...,5},
    y grid style={gray},
    grid=both,
    minor grid style=gray,
    major grid style={gray},
]
\addplot[mark=*, ultra thick, black, data cs=polar] coordinates {
    (0,3)  % 知识性得分,这里修改为你的分数
    (60,5)  % 授课质量得分,这里修改为你的分数
    (120,5)  % 作业友好得分,这里修改为你的分数
    (180,5)  % 工作量得分,这里修改为你的分数
    (240,3)  % 易通过得分,这里修改为你的分数
    (300,5)  % 有趣得分,这里修改为你的分数
    (360,3)  % 为了闭合图形,最后的一个点应该和第一个点一样,这里修改为你的分数
};
\end{polaraxis}
\end{tikzpicture}
\end{minipage}%
\begin{minipage}{0.45\textwidth}
\raggedleft
\begin{tabular}{r|c}
\textbf{ } & \textbf{22-23} \\ \hline
通过率 & 88.5\% \\ 
平均分 &  7.10\\ 
参与学生数 &  61\\
学分 & 3 EC\\
\end{tabular}
\end{minipage}\\

这门课,强烈推荐;笔者个人角度,觉得堪称整个第一年我所上过课里最有趣的。它分为两部分,Lecture和Seminar。其中Lecture并非强制,但个人觉得因授课质量极高,且非常有趣而十分值得。Seminar强制出勤,席间表现占到总分的40\%,每节都会由一组人进行主持(主持表现会占据一定分数),针对特定主题开展讨论,氛围很好。

如前所言,Lecture授课质量是极高的;老师是一位阿根廷人,英语完全没有什么口音,上课激情澎湃;并不是代尔夫特大多数老师那样,站在讲台上面,泛泛地念着PPT。他会经常走下讲台,在一排排学生走廊里走来又走去。上课的互动性很强,甚至主观毛估估一半内容都由学生提问临时引申涉及。

主要教授内容就是哲学,什么是科学?什么是工程?休谟疑问,Kuhn的科学观点等等等,会讲授百家之言,马克思哲学也占据相当大的内容。但是,这位讲师笔者政治光谱比较正统红色,对CCP时有锐评(其实他对各国都锐评),与大陆角度看他的看法略显政治不正确;如果这让你感觉不舒服,劝还是不要选了吧。

Seminar的主持,会在Seminar进行的前两周提供大概的文档和阅读材料。基于阅读材料如何展开,完全因组而异,自由度很高。依照22-23学年来看,大多数组都会通过摇色子或者抽卡的形式,在Seminar上针对某一问题对在场的人进行分组;而后开展辩论。Seminar完成后需要上交报告,内容涵盖如何主持的计划提纲,以及Seminar进行中的记录;推荐在主持Seminar的同时进行记录或者录音,防止遗忘。

这门课,除了在所主持Seminar的那周需要小组完成会议报告外,没有什么额外笔头的课后作业;但阅读量较高,平均每周需要阅读30页左右的哲学书籍;由于笔者个人觉得书籍内容十分有趣且引人入胜,完全课外书感觉,所以主观也并不会觉得压力很大。

考试的话,难度略高,完全简答题。老师会在考前几周上传模拟试卷等以供复习。虽说考试略难,但笔者主观怀疑评卷其实并不完全基于你考场所写。笔者在29/06/2023考试,考试所写至少也有数千词,30/06/2023下午成绩却已经公布;实在难以想象老师是怎样完成的评卷。结合他上课时常提及他根本不想组织考试,完全学院要求才迫不得已,实在略显可疑。
\begin{flushright}
王昊辰; 30/06/2023
\end{flushright}

\subsection{WM1401TU/WM1402TU Ethics of Healthcare Technologies}\hypertarget{WM1401TU}{} 
\begin{minipage}{0.45\textwidth}
\centering
\begin{tikzpicture}
\begin{polaraxis}[
    width=0.65\textwidth,
    height=0.65\textwidth,
    yticklabels={},
    xtick={0,60,...,300},
    xticklabels={知识性, 授课质量, 作业友好,工作量适宜,易通过, 有趣},
    ymin=0,
    ymax=5,
    ytick={1,2,...,5},
    y grid style={gray},
    grid=both,
    minor grid style=gray,
    major grid style={gray},
]
\addplot[mark=*, ultra thick, black, data cs=polar] coordinates {
    (0,2)  % 知识性得分,这里修改为你的分数
    (60,3)  % 授课质量得分,这里修改为你的分数
    (120,5)  % 作业友好得分,这里修改为你的分数
    (180,5)  % 工作量得分,这里修改为你的分数
    (240,5)  % 易通过得分,这里修改为你的分数
    (300,3)  % 有趣得分,这里修改为你的分数
    (360,2)  % 为了闭合图形,最后的一个点应该和第一个点一样,这里修改为你的分数
};
\end{polaraxis}
\end{tikzpicture}
\end{minipage}%
\begin{minipage}{0.45\textwidth}
\raggedleft
\begin{tabular}{r|c}
\textbf{ } & \textbf{22-23} \\ \hline
通过率 &100/100 \% \\ 
平均分 & 8.19/8.11 \\ 
参与学生数 &21/117 \\
学分 & 3/5 EC\\
\end{tabular}
\end{minipage}\\

医疗技术伦理有两个课程代码,WM1401TU为3学分,WM1420TU为5学分,WM1420TU需要在WM1401TU的基础上额外完成一个小论文,小论文可以以两到三人的小组完成。每周需要完成思考题的作业。必须参加两次Case Study的Tutorial。

2023年的考试为线上考试,考试内容出自课后思考题,没有监考,可以在考试期间浏览任意材料。
总的来说就是一门送分课,并没有太大的学习意义,作业的负担不大,考试难度不大,基本都出在思考题,而且可以用ChatGPT来帮助你回答。建议在相对轻松的Q1上完伦理课从而减轻后续学期的学习压力。
\begin{flushright}
张先治; 07/06/2023
\end{flushright}


\vspace{\betsubsec} %section间留白
\section{ME-BMD Obligatory Courses}
\subsection{BM41040 Neuromechanics \& Motor Control}
\begin{minipage}{0.45\textwidth}
\centering
\begin{tikzpicture}
\begin{polaraxis}[
    width=0.65\textwidth,
    height=0.65\textwidth,
    yticklabels={},
    xtick={0,60,...,300},
    xticklabels={知识性, 授课质量, 作业友好,工作量适宜,易通过, 有趣},
    ymin=0,
    ymax=5,
    ytick={1,2,...,5},
    y grid style={gray},
    grid=both,
    minor grid style=gray,
    major grid style={gray},
]
\addplot[mark=*, ultra thick, black, data cs=polar] coordinates {
    (0,4)  % 知识性得分,这里修改为你的分数
    (60,3)  % 授课质量得分,这里修改为你的分数
    (120,3)  % 作业友好得分,这里修改为你的分数
    (180,3)  % 工作量得分,这里修改为你的分数
    (240,1)  % 易通过得分,这里修改为你的分数
    (300,3)  % 有趣得分,这里修改为你的分数
    (360,4)  % 为了闭合图形,最后的一个点应该和第一个点一样,这里修改为你的分数
};
\end{polaraxis}
\end{tikzpicture}
\end{minipage}%
\begin{minipage}{0.45\textwidth}
\raggedleft
\begin{tabular}{r|c}
\textbf{ } & \textbf{21-22} \\ \hline
通过率 & 68.7\% \\ 
平均分 &  6.3\\ 
参与学生数 &  57\\
学分 & 5 EC\\
\end{tabular}
\end{minipage}\\

这门课,持续两个Q,Q3与Q4;总共的测验包含两次电脑选择题考试(共占40\%)与一次最终笔试(占60\%)。这两次考试都没有卡死的分数,但加权平均分必须达到57.5\%的及格分数。两次Computer难度很大,主要涉及概念知识;如果没有通过,在Q4尾声会给予二次补考的机会,此次补考可以一次修补全部两次考试的分数。但是,在修补考试中所取得的最高分数便是\%60;超过60分要考的部分是不作数的。

电脑考试外的期末笔试,就更多是对知识理解的考验了。完全简答题,大多会以实验的形式提问,难度同样很高。并且期末考试前,并不会提供多少往年试卷等方便做题训练,2022-2023学年的情况是,仅提供了一份2012年和一份2015年的远古试卷来参考。

除了考试外,要想通过课程,还需要\href{https://drive.google.com/file/d/1KxbD13PnAkygeto3pSj9S3xFacHufyzy/view?usp=sharing}{\uline{完成几个作业}}。 作业并不参与最终分数的计算,通过即可,但作业里的内容会在考试中反复提及。作业明面上会有一个死线,但死线前完成并不是必须,仅仅意味着你可以在考试前获得反馈(虽然反馈也往往就一两句话,并不会十分有用);理论上只要期末考试前把作业完成交掉就是可以的。

课程的阅读量会很大,大量的paper与书籍需要阅读。按照之前其它课程的惯常,这门课像笔者一样不去阅读或者挑着阅读也不是完全不行,但这课的上课速度无比风驰电掣且跳跃,在没有阅读材料辅助的情况下很容易迷失跟不上。且基于2022-2023学年考试内容来看,阅读中提及到的实验也会于考试出现。

牢骚以外,讲讲课程的教学内容。课程可以划分两个方面;Q3有关人体运动,会涉及肌肉结构及模拟的相关知识,也会讲授人体动态的模拟;这一部分的内容其实和ME41006多有重复,但ME41006主要关乎实操,这门课大多理论。Q4开始,内容就转向了人体控制;会讲授人体作为控制系统的整套逻辑,以Block Diagram来模拟人体的控制,用Kalman Filter来解释人类的信息获取及决策过程等等。总之内容涵盖非常广泛,是一门个人觉得非常难学的课程。

教学质量,更是给这课的体验学上加霜。虽说大半课程会是Guest Lecturer讲授,但他们的水平实在良莠不齐,大多纯念PPT。对时间把握也有待优化,往往半节课悠哉,半节课猪突。几个固定Lecturer里面,主力一员与ME41065的老师是同一人,纯念PPT。

总的来说,个人感觉这门课堪称ME第一年授课质量地板;可惜BMD必修没办法。不过确实承认,虽然整个课程体验很不好,但讲授的内容确实丰富,考完试还真会有种受益匪浅的感觉。
\begin{flushright}
王昊辰; 30/06/2023
\end{flushright}

\subsection{ME41006 Musculoskeletal Modeling and Simulation}
\begin{minipage}{0.45\textwidth}
\centering
\begin{tikzpicture}
\begin{polaraxis}[
    width=0.65\textwidth,
    height=0.65\textwidth,
    yticklabels={},
    xtick={0,60,...,300},
    xticklabels={知识性, 授课质量, 作业友好,工作量适宜,易通过, 有趣},
    ymin=0,
    ymax=5,
    ytick={1,2,...,5},
    y grid style={gray},
    grid=both,
    minor grid style=gray,
    major grid style={gray},
]
\addplot[mark=*, ultra thick, black, data cs=polar] coordinates {
    (0,3)  % 知识性得分,这里修改为你的分数
    (60,2)  % 授课质量得分,这里修改为你的分数
    (120,1)  % 作业友好得分,这里修改为你的分数
    (180,1)  % 工作量得分,这里修改为你的分数
    (240,3)  % 易通过得分,这里修改为你的分数
    (300,4)  % 有趣得分,这里修改为你的分数
    (360,3)  % 为了闭合图形,最后的一个点应该和第一个点一样,这里修改为你的分数
};
\end{polaraxis}
\end{tikzpicture}
\end{minipage}%
\begin{minipage}{0.45\textwidth}
\raggedleft
\begin{tabular}{r|c|c}
\textbf{ } & \textbf{22-23} & \textbf{21-22}\\ \hline
通过率 & \% & 100\%\\ 
平均分 &   & 7.08\\ 
参与学生数 &  & 91\\
学分 & 4 EC & 3 EC\\
\end{tabular}
\end{minipage}\\

这门课,主要内容顾名思义,肌肉骨骼的模拟;鉴于理论内容其实在BM41040已经大多涵盖,课程教授的可以说就是Opensim这款软件的操作和理解(其实很难)。

授课老师和ME41055的老师一样,也是最近几年(2023年角度看)才从美国来到TUD(他来自斯坦福);同样类似的,是这两门课的Workload 也都格外的高。其长度仅仅一个Q,但于此期间,需在前3周,以每周一个的速度完成掉三支作业;前三周的负荷,基本刚刚就是赶上一个Deadline,就又被堆下来了另一个,非常的夸张。

2022-2023学年而言,前两支作业都是非常开放的研究型问题;第一个为Passive Walker, 需要你在1周内,优化使其能在崎岖路面行走;走得越远,分数越高;除此还要回答理论性的题目和录制一个2分钟的presentation,非常耗费时间,加上其他课,每天甚至能睡4小时;一周下来,绝对感慨自己炎黄超人。第二个作业稍好些,是优化一个肌肉使其在拔河比赛中尽可能战胜对手。而第三个作业,比较轻松,循规蹈矩按照说明操作就好。

在完成三个作业以后,还会有一个最终的研究任务。很多研究课题可以选择,需要在三周时间内做完,时间也比较紧张:第一周上交proposal,第二周上传一个5分钟的presentation,第三周写完论文;猪突猛进直呼逆天。而且论文和presentation上交的两周,是与考试周与复习周重合的。笔者私下感觉,最好在最终作业刚颁布,甚至还没正式确定时候,就抓紧研究,速战速决。

但是,比较宽慰的是,这门课是没有考试的。也就是说,平时若是愿意花时间死磕作业,成绩应该不会难看。笔者的三次作业,都已经上传,\href{https://drive.google.com/drive/folders/1mJun-EmYGX1DovFISP9lXZNkdnutQpuh?usp=sharing}{\uline{于此链接可以查看}},希望有点用处吧。
\begin{flushright}
王昊辰; 16/06/2023
\end{flushright}


\subsection{RO47006 Human Robot Interaction}
\begin{minipage}{0.45\textwidth}
\centering
\begin{tikzpicture}
\begin{polaraxis}[
    width=0.65\textwidth,
    height=0.65\textwidth,
    yticklabels={},
    xtick={0,60,...,300},
    xticklabels={知识性, 授课质量, 作业友好,工作量适宜,易通过, 有趣},
    ymin=0,
    ymax=5,
    ytick={1,2,...,5},
    y grid style={gray},
    grid=both,
    minor grid style=gray,
    major grid style={gray},
]
\addplot[mark=*, ultra thick, black, data cs=polar] coordinates {
    (0,4)  % 知识性得分,这里修改为你的分数
    (60,4)  % 授课质量得分,这里修改为你的分数
    (120,5)  % 作业友好得分,这里修改为你的分数
    (180,5)  % 工作量得分,这里修改为你的分数
    (240,5)  % 易通过得分,这里修改为你的分数
    (300,4)  % 有趣得分,这里修改为你的分数
    (360,4)  % 为了闭合图形,最后的一个点应该和第一个点一样,这里修改为你的分数
};
\end{polaraxis}
\end{tikzpicture}
\end{minipage}%
\begin{minipage}{0.45\textwidth}
\raggedleft
\begin{tabular}{r|c}
\textbf{ } & \textbf{22-23} \\ \hline
通过率 & 100\% \\ 
平均分 & 7.36 \\ 
参与学生数 & 193 \\ 
学分 & 5 EC\\
\end{tabular}
\end{minipage}\\

这门课,其实相当于一大堆知识的大杂烩;完全是把诸多杂七杂八的研究领域糅杂一起的产物。

它所教授的内容,总得概括,大概三部分吧:其一是把人体看作一个控制机构,由此也会衍生2个作业。第二部分,就是人机交互了,会讲述这方面的一些历史,从人去适应机器到机器适应人。最后一部分,是一系列前沿的研究,会请到TUD各个研究室的人来客座。

笔者个人来讲,也可能一直以来都蛮文科的原因,是蛮享受这门课的;但似乎大多荷兰人风评,又觉得挺扯淡的。无论如何,通过率都摆在了这里,实在良心必修课。

这课的评分主要是2方面;作业及考试。作业的P1和P1B部分为小组完成,而P2是个人。P2的期限与寒假重合,作业上讲,工作量十分怡人,是一门并不会感到压力的课程。

考试的话,是开卷的。允许你携带任意数量的资料。似乎几乎每个人都是3,400页资料起步;基本人均缩放打印了全部PPT。考题分为2部分,选择与后面的开放问题,两者都会就课上讲到的某些犄角旮旯知识大做文章。个人建议在上课时候就在PPT上做好标注,提前构建一个类似于知识点字典一样的目录,方便考试时候快速翻阅查找。

P1的A与B两个作业因为是小组完成,我这里就不提供啦。 但P2,虽然似乎每年都略有不同,但我还是提供一下通往我2022年所做\href{https://drive.google.com/file/d/1GwwX7ZxEE8R7JeDm6NcHQu_IA34ZqyRe/view?usp=sharing}{\uline{的链接吧}}。这里我的第一题计算并不准确,除此以外分数都还是在8分以上的。

而且需要注意的是,P2作业有些非常犄角旮旯的要求,比如引用就会有些细则;还请详细阅读要求啊。这些规定不符合的话,听说会有直接0分的情况。
\begin{flushright}
王昊辰; 02/06/2023
\end{flushright}



\vspace{\betsubsec} %section间留白
\section{ME-BMD Design Project Courses}
\subsection{ME41096 Bio Inspired Design}
\begin{minipage}{0.45\textwidth}
\centering
\begin{tikzpicture}
\begin{polaraxis}[
    width=0.65\textwidth,
    height=0.65\textwidth,
    yticklabels={},
    xtick={0,60,...,300},
    xticklabels={知识性, 授课质量, 作业友好,工作量适宜,易通过, 有趣},
    ymin=0,
    ymax=5,
    ytick={1,2,...,5},
    y grid style={gray},
    grid=both,
    minor grid style=gray,
    major grid style={gray},
]
\addplot[mark=*, ultra thick, black, data cs=polar] coordinates {
    (0,1)  % 知识性得分,这里修改为你的分数
    (60,3)  % 授课质量得分,这里修改为你的分数
    (120,5)  % 作业友好得分,这里修改为你的分数
    (180,5)  % 工作量得分,这里修改为你的分数
    (240,5)  % 易通过得分,这里修改为你的分数
    (300,3)  % 有趣得分,这里修改为你的分数
    (360,1)  % 为了闭合图形,最后的一个点应该和第一个点一样,这里修改为你的分数
};
\end{polaraxis}
\end{tikzpicture}
\end{minipage}%
\begin{minipage}{0.45\textwidth}
\raggedleft
\begin{tabular}{r|c}
\textbf{ } & \textbf{22-23} \\ \hline
通过率 & 100.0\% \\ 
平均分 & 7.65 \\ 
参与学生数 & 198 \\ 
学分 & 5 EC\\
\end{tabular}
\end{minipage}\\

先定个结论,属实良心课程。

这门课,顾名思义,就是仿生设计。话虽如此,实际操作中却更多是已经想到了某种解决方案,而后套到一个生物的机构上面。与其说是工程与设计,最后的产出,大多数组来看其实都是蛮科幻甚至抽象的;所以这方面完全没有太大必要担心,通过率100\%,已然说明问题;属实良心至极。

而设计的项目,会在课程开始的时候,从一系列项目里,让你填表选出自己想去的项目,想要一起合作的人。当然,若是像笔者一样的自闭儿,其实sign Up as an Individual也是完全没有问题的(甚至后来发现大多荷兰人都没有提前自己设组,全是个人身份填表)。而后就是持续2个Q的设计,一周隔一周的,会进行Presentation,汇报一下进度什么的;但Presentation,其实是完全不会纳入分数考量的;字面意思的,仅仅是给你一个集思广益获取同学反馈的渠道,所以也完全不用担心吧。

课程期间也会有Lecture,除了刚开始的几节的确有关设计作业有关课程。此后大多完全关于一些仿生设计知识,启迪式地,往往是Guest Lecture; 荷兰人极少出席,甚至有整个lecture hall个位数学生的尴尬场面。但笔者个人看看录像,其实觉得课程大多还蛮有价值的。

总之,这课就是一个设计作业,最后上交一篇小组报告。工作量很合理;Lecturer,也就是Paul为人也十分不错。我是十分推荐的。而且恶臭功利角度,顺带一提,Paul的研究组有很多有趣的毕设项目可选,且与一些公司和医院都有着很不错的关系。选这门课,和Paul混个脸熟,其实对于毕设与实习有微微帮助的。
\begin{flushright}
王昊辰; 02/06/2023
\end{flushright}

\subsection{ME46015 Precision Mechanism Design}
%待完成;张先治



\subsection{ME46115 Compliant Mechanisms}
\begin{minipage}{0.45\textwidth}
\centering
\begin{tikzpicture}
\begin{polaraxis}[
    width=0.65\textwidth,
    height=0.65\textwidth,
    yticklabels={},
    xtick={0,60,...,300},
    xticklabels={知识性, 授课质量, 作业友好,工作量适宜,易通过, 有趣},
    ymin=0,
    ymax=5,
    ytick={1,2,...,5},
    y grid style={gray},
    grid=both,
    minor grid style=gray,
    major grid style={gray},
]
\addplot[mark=*, ultra thick, black, data cs=polar] coordinates {
    (0,4)  % 知识性得分,这里修改为你的分数
    (60,5)  % 授课质量得分,这里修改为你的分数
    (120,3)  % 作业友好得分,这里修改为你的分数
    (180,2)  % 工作量得分,这里修改为你的分数
    (240,3)  % 易通过得分,这里修改为你的分数
    (300,4)  % 有趣得分,这里修改为你的分数
    (360,4)  % 为了闭合图形,最后的一个点应该和第一个点一样,这里修改为你的分数
};
\end{polaraxis}
\end{tikzpicture}
\end{minipage}%
\begin{minipage}{0.45\textwidth}
\raggedleft
\begin{tabular}{r|c}
\textbf{ } & \textbf{21-22} \\ \hline
通过率 & 99.1\%\\ 
平均分 & 7.74 \\ 
参与学生数 & 113 \\ 
学分 & 4 EC\\
\end{tabular}
\end{minipage}\\

与BID (Bio Inspired Design)一样的是,这门课最终的评分大致也取决于你的最终报告。所不同的是,这门课的期中,也就是Q1结束时(这课占2个Q),会有额外的一个考试;这个考试,就我所在的的小组而言,通过率并不十分理想。我们小组4人,其中也就是包括我在内2人通过,于此推演,甚至只有50\%.

关于授课内容,主要也就是顾名思义,柔性机构。会阐述给你一些Pseudo-Rigid-Body Model(PRBM)分析的理论。但是笔者经验算下来,要想应付评分大头的设计项目,其实这些教授的知识并不是十分足够,自己也是需要进行一些课程以外的拓展与阅读的。

这课,同BID一样的是,每两周也要进行presentation。但又不同于BID的科幻,这门课最终产出的结果是的确需要你进行打样与实验的。所以你真的就得把你设计的作品做出来,由此带来了更多的工作量。大概就是在圣诞前你们的设计和打样就得基本完成掉。

理论上,在这门课的进行中,是需要使用Ansys APDL进行FEM的。然而其实际操作,这些软件更多的是取决于小组个人的选择,例如我们小组,就事实上使用了Nastran。之前很多人问起我TUD究竟用什么CAD软件,什么FEM软件。其实这真的没有在实际操作中有太严厉的标准,更多的是只要小组一致同意,就是可以合理操作的。

结论,抛开性价比而言。这课我是觉得还蛮好的。但仅仅4学分,总让我觉得与工作量不甚相配。总的来说,选还是不选,还是希望按照个人兴趣决定罢。
\begin{flushright}
王昊辰; 02/06/2023
\end{flushright}

\subsection{ME41085 Biomechatromics}
\begin{minipage}{0.45\textwidth}
\centering
\begin{tikzpicture}
\begin{polaraxis}[
    width=0.65\textwidth,
    height=0.65\textwidth,
    yticklabels={},
    xtick={0,60,...,300},
    xticklabels={知识性, 授课质量, 作业友好,工作量适宜,易通过, 有趣},
    ymin=0,
    ymax=5,
    ytick={1,2,...,5},
    y grid style={gray},
    grid=both,
    minor grid style=gray,
    major grid style={gray},
]
\addplot[mark=*, ultra thick, black, data cs=polar] coordinates {
    (0,4)  % 知识性得分,这里修改为你的分数
    (60,5)  % 授课质量得分,这里修改为你的分数
    (120,3)  % 作业友好得分,这里修改为你的分数
    (180,2)  % 工作量得分,这里修改为你的分数
    (240,3)  % 易通过得分,这里修改为你的分数
    (300,4)  % 有趣得分,这里修改为你的分数
    (360,4)  % 为了闭合图形,最后的一个点应该和第一个点一样,这里修改为你的分数
};
\end{polaraxis}
\end{tikzpicture}
\end{minipage}%
\begin{minipage}{0.45\textwidth}
\raggedleft
\begin{tabular}{r|c}
\textbf{ } & \textbf{21-22} \\ \hline
通过率 & 100\% \\ 
平均分 & 7.59 \\ 
参与学生数 & 81 \\
学分 & 4 EC\\
\end{tabular}
\end{minipage}\\

\vspace{\betsubsec} %section间留白
\subsection{ME-BMD Electives}
\subsubsection{BM41155 3D Printing}
\begin{minipage}{0.45\textwidth}
\centering
\begin{tikzpicture}
\begin{polaraxis}[
    width=0.65\textwidth,
    height=0.65\textwidth,
    yticklabels={},
    xtick={0,60,...,300},
    xticklabels={知识性, 授课质量, 作业友好,工作量适宜,易通过, 有趣},
    ymin=0,
    ymax=5,
    ytick={1,2,...,5},
    y grid style={gray},
    grid=both,
    minor grid style=gray,
    major grid style={gray},
]
\addplot[mark=*, ultra thick, black, data cs=polar] coordinates {
    (0,3)  % 知识性得分,这里修改为你的分数
    (60,3)  % 授课质量得分,这里修改为你的分数
    (120,5)  % 作业友好得分,这里修改为你的分数
    (180,5)  % 工作量得分,这里修改为你的分数
    (240,5)  % 易通过得分,这里修改为你的分数
    (300,4)  % 有趣得分,这里修改为你的分数
    (360,3)  % 为了闭合图形,最后的一个点应该和第一个点一样,这里修改为你的分数
};
\end{polaraxis}
\end{tikzpicture}
\end{minipage}%
\begin{minipage}{0.45\textwidth}
\raggedleft
\begin{tabular}{r|c}
\textbf{指标} & \textbf{数据} \\ \hline
通过率 & 88.6\% \\ 
平均分 & 6.97 \\ 
参与学生数 & 166 \\
学分 & 4 EC\\
\end{tabular}
\end{minipage}\\

强烈推荐,白给的学分
\subsubsection{ME41035 Special Topics in Sports Engineering}
% 待完成;张先治 q5的课7.17前写完


\subsubsection{ME41065 System Identification and Parameter Estimation}
\begin{minipage}{0.45\textwidth}
\centering
\begin{tikzpicture}
\begin{polaraxis}[
    width=0.65\textwidth,
    height=0.65\textwidth,
    yticklabels={},
    xtick={0,60,...,300},
    xticklabels={知识性, 授课质量, 作业友好,工作量适宜,易通过, 有趣},
    ymin=0,
    ymax=5,
    ytick={1,2,...,5},
    y grid style={gray},
    grid=both,
    minor grid style=gray,
    major grid style={gray},
]
\addplot[mark=*, ultra thick, black, data cs=polar] coordinates {
    (0,4)  % 知识性得分,这里修改为你的分数
    (60,2)  % 授课质量得分,这里修改为你的分数
    (120,3)  % 作业友好得分,这里修改为你的分数
    (180,3)  % 工作量得分,这里修改为你的分数
    (240,5)  % 易通过得分,这里修改为你的分数
    (300,4)  % 有趣得分,这里修改为你的分数
    (360,4)  % 为了闭合图形,最后的一个点应该和第一个点一样,这里修改为你的分数
};
\end{polaraxis}
\end{tikzpicture}
\end{minipage}%
\begin{minipage}{0.45\textwidth}
\raggedleft
\begin{tabular}{r|c|c}
\textbf{指标} & \textbf{数据} & \textbf{21-22}\\ \hline
通过率 & 87.5\% & 91.7\% \\ 
平均分 & 6.94 & 6.65\\ 
参与学生数 & 48 &44\\
学分 & 7 EC&\\
\end{tabular}
\end{minipage}\\

每上三节课需要完成一次作业,使用matlab完成,作业不算分,但是必须要全部通过才能录入成绩,没有严格的deadline,在deadline之前提交可以得到反馈。
成绩的40\%由4次digitial test 决定,每上三节课都进行,形式是闭卷选择题,内容为概念以及作业每考完两次digitial test可以报名查看正确答案,但是不能拍照或做笔记。在期末考试之前可以进行digitial test的补考,补考的最高分数为6
成绩的60\%是期末考试,笔答闭卷考试,考试内容更偏向于概念理解,可以参考的往年考试题只有14,16的两张卷,studocu.com上有一些比较好的总结可以用来复习。
总的来说是一门不错的课程比较推荐,能对相关领域的知识有大概的理解,相应的知识在Mechatronic system design, Neuromehchanics and motor control等课程中都有体现。作业难度适中,考试难度不大,上课有点无聊,老师基本是在念PPT,最重要的是它有7学分还是比较值的。
\begin{flushright}
张先治; 2023/6/7
\end{flushright}

\subsubsection{ME41120 Freehand Sketching of Products and Mechanisms}
\begin{minipage}{0.45\textwidth}
\centering
\begin{tikzpicture}
\begin{polaraxis}[
    width=0.65\textwidth,
    height=0.65\textwidth,
    yticklabels={},
    xtick={0,60,...,300},
    xticklabels={知识性, 授课质量, 作业友好,工作量适宜,易通过, 有趣},
    ymin=0,
    ymax=5,
    ytick={1,2,...,5},
    y grid style={gray},
    grid=both,
    minor grid style=gray,
    major grid style={gray},
]
\addplot[mark=*, ultra thick, black, data cs=polar] coordinates {
    (0,5)  % 知识性得分,这里修改为你的分数
    (60,5)  % 授课质量得分,这里修改为你的分数
    (120,5)  % 作业友好得分,这里修改为你的分数
    (180,4)  % 工作量得分,这里修改为你的分数
    (240,5)  % 易通过得分,这里修改为你的分数
    (300,5)  % 有趣得分,这里修改为你的分数
    (360,5)  % 为了闭合图形,最后的一个点应该和第一个点一样,这里修改为你的分数
};
\end{polaraxis}
\end{tikzpicture}
\end{minipage}%
\begin{minipage}{0.45\textwidth}
\raggedleft
\begin{tabular}{r|c}
\textbf{指标} & \textbf{数据} \\ \hline
通过率 & 100\% \\ 
平均分 & 7.71 \\ 
参与学生数 & 40 \\
学分 & 3 EC\\
\end{tabular}
\end{minipage}\\

真的是非常非常好的一门课,说是第一年最有价值的一门课也不为过!

小班化教学,采用邮件预约制;因为十分火爆,想要上这门课需要在Q1刚开始时候就发邮件,诉说自己的动机进行预约。Q3真正上的时候,40人会分为2个班级;每个班级20人,配备1位lecturer和一位助教,属实小班化教学了。

这门课没有考试,评分就是每周作业和你上课时候的笔记。每周作业一般就是画一个物体,笔者经验,不完美主义精雕细琢,一周至多2,3小时。这些作业文档,都会在课程结束时候要求你扫描上传;笔者就是因为平时上课没有做笔记,最后成绩并不好看。但就算如此,也是所有参与的人都通过了! 通过率100\%毕竟。

至于教学内容;主要就是教授你画画,一些透视技巧,怎样打上阴影。这课要求你必须是此前毫无画画基础的。笔者上这门课之前,画图纯纯火柴人水平;这课以后,能够画出三维的,甚至有点黑白照片感觉的物体了(我的最终作业画了一块手表)。课程结束时候还会像模地像样的举办画展,给参加课程并通过的学生全部发上颇有仪式氛围的结课证书;感觉十分不错的。

如果有意向想要选这门课,高兴的话,其实可以在国内就买好一些必要的器材:Copic 的N3,N5,N8都是必需品(N3和N5可以买2支或者买补充墨水,一支可能不够用);以及A3的马克笔绘画纸和针管笔(推荐买支0.1的买支0.4的)。在荷兰买这些非常之贵,比如马克笔纸,平均一张就要0.35欧。
\begin{flushright}
王昊辰; 02/06/2023
\end{flushright}

\subsubsection{RO47005 Planing and Decision Making}
\begin{minipage}{0.45\textwidth}
\centering
\begin{tikzpicture}
\begin{polaraxis}[
    width=0.65\textwidth,
    height=0.65\textwidth,
    yticklabels={},
    xtick={0,60,...,300},
    xticklabels={知识性, 授课质量, 作业友好,工作量适宜,易通过, 有趣},
    ymin=0,
    ymax=5,
    ytick={1,2,...,5},
    y grid style={gray},
    grid=both,
    minor grid style=gray,
    major grid style={gray},
]
\addplot[mark=*, ultra thick, black, data cs=polar] coordinates {
    (0,4)  % 知识性得分,这里修改为你的分数
    (60,4)  % 授课质量得分,这里修改为你的分数
    (120,4)  % 作业友好得分,这里修改为你的分数
    (180,1)  % 工作量得分,这里修改为你的分数
    (240,5)  % 易通过得分,这里修改为你的分数
    (300,4)  % 有趣得分,这里修改为你的分数
    (360,4)  % 为了闭合图形,最后的一个点应该和第一个点一样,这里修改为你的分数
};
\end{polaraxis}
\end{tikzpicture}
\end{minipage}%
\begin{minipage}{0.45\textwidth}
\raggedleft
\begin{tabular}{r|c}
\textbf{指标} & \textbf{数据} \\ \hline
通过率 & 89\% \\ 
平均分 & 7.66 \\ 
参与学生数 & 89 \\ 
学分 & 5 EC\\
\end{tabular}
\end{minipage}\\

该课主要概述了机器人的运动规划,就本人而言感觉这是一门对该领域很好的入门课,对于想研究(BIO)robotics的同学是一门不错的课。这门课的授课质量,知识型,有趣性,作业布置都属上乘,美中不足的是课程再前期讲的很快,对机器人领域不了解的小伙伴听起来会很吃力,而且老师的西班牙口音稍微有点重,所以我在最开始几周基本没听懂。

和机器人学院的其他五分必修课一样,这门课也是作业+大作业+考试的配置,所以课业压力会很大。大作业需要自己和组员完成一款规划算法的设计与分析,认真完成的话还是很有意义。

\begin{flushright}
马润禹; 03/06/2023
\end{flushright}

\subsubsection{RO47003 Robot Software Practicals}
\begin{minipage}{0.45\textwidth}
\centering
\begin{tikzpicture}
\begin{polaraxis}[
    width=0.65\textwidth,
    height=0.65\textwidth,
    yticklabels={},
    xtick={0,60,...,300},
    xticklabels={知识性, 授课质量, 作业友好,工作量适宜,易通过, 有趣},
    ymin=0,
    ymax=5,
    ytick={1,2,...,5},
    y grid style={gray},
    grid=both,
    minor grid style=gray,
    major grid style={gray},
]
\addplot[mark=*, ultra thick, black, data cs=polar] coordinates {
    (0,5)  % 知识性得分,这里修改为你的分数
    (60,5)  % 授课质量得分,这里修改为你的分数
    (120,5)  % 作业友好得分,这里修改为你的分数
    (180,4)  % 工作量得分,这里修改为你的分数
    (240,5)  % 易通过得分,这里修改为你的分数
    (300,5)  % 有趣得分,这里修改为你的分数
    (360,5)  % 为了闭合图形,最后的一个点应该和第一个点一样,这里修改为你的分数
};
\end{polaraxis}
\end{tikzpicture}
\end{minipage}%
\begin{minipage}{0.45\textwidth}
\raggedleft
\begin{tabular}{r|c}
\textbf{指标} & \textbf{数据} \\ \hline
通过率 & 无数据 \\ 
平均分 & 无数据 \\ 
参与学生数 & 无数据 \\
学分 & 5 EC\\
\end{tabular}
\end{minipage}\\

这门课个人感觉很有用,内容覆盖了linux,git,c++,和ROS的基础知识,是本科没有相关基础的同学的不二选择!

linux主题会涉及一些基本的操作系统概念与文件管理的方法,git主题会对版本管理,分支合并等等概念进行介绍。c++会涉及一些基础的语法以及最重要的编译概念。随后的ROS会涉及基本概念并在后期大致介绍常用的ROS c++库如pcl。

这门课的作业给我的体验很好。需要完成三次不计分的平时小组作业,和一次独立且计分的小项目。小组作业会在上完linux和git后开始,形式为两人利用git进行合作,往gitlab上提交代码。每次作业都会有一个很长的详细的文档介绍,基本上认真读完作业就没有压力。平时作业虽不评分但需要在gitlab上对其他组的代码进行点评。可以说这样的作业形式让我充分实践了课堂所学。最后的独立小项目我当时是要用pcl库编写一个简单的ROS程序,老师会提供很多的模块帮助你看到很有趣的最终效果。

最后这门课会有一次考试,内容为简单的概念问答。
\begin{flushright}
孙天辰; 04/06/2023
\end{flushright}


\subsubsection{WI4771TU Object Oriented Scientific Programming C++}
\begin{minipage}{0.45\textwidth}
\centering
\begin{tikzpicture}
\begin{polaraxis}[
    width=0.65\textwidth,
    height=0.65\textwidth,
    yticklabels={},
    xtick={0,60,...,300},
    xticklabels={知识性, 授课质量, 作业友好,工作量适宜,易通过, 有趣},
    ymin=0,
    ymax=5,
    ytick={1,2,...,5},
    y grid style={gray},
    grid=both,
    minor grid style=gray,
    major grid style={gray},
]
\addplot[mark=*, ultra thick, black, data cs=polar] coordinates {
    (0,4)  % 知识性得分,这里修改为你的分数
    (60,4)  % 授课质量得分,这里修改为你的分数
    (120,3)  % 作业友好得分,这里修改为你的分数
    (180,4)  % 工作量得分,这里修改为你的分数
    (240,5)  % 易通过得分,这里修改为你的分数
    (300,3)  % 有趣得分,这里修改为你的分数
    (360,4)  % 为了闭合图形,最后的一个点应该和第一个点一样,这里修改为你的分数
};
\end{polaraxis}
\end{tikzpicture}
\end{minipage}%
\begin{minipage}{0.45\textwidth}
\raggedleft
\begin{tabular}{r|c}
\textbf{指标} & \textbf{数据} \\ \hline
通过率 & 100\% \\ 
平均分 & 8.15 \\ 
参与学生数 & 198 \\
学分 & 3 EC\\
\end{tabular}
\end{minipage}\\

主要就是教会你C++这门语言吧。其实这课没什么好说的;教学质量一般般,老师一般般,每周作业难度也是一般般,完全一股一般般的腔调。当然这里的一般般都是tud标准下的一般般了,每周作业,笔者经验,大概3小时左右可以完成吧。

这课程没有考试,最终评分主要依照一个大作业。在我们这届,也就是2022-2023学年,是一个passive walker的计算。需要你自己写出2种积分求解器。这是很难的。但由于不是考试,所以可以,至少私下里,到处点头哈腰交流答案;或者之前国内的同学询问之类。总之通过不是什么问题,这也在其100\%的通过率和较高的平均分上可以体现。

总结讲,我觉得这课还是蛮值得的。无论学分的性价比还是知识性而言。
\begin{flushright}
王昊辰; 02/06/2023
\end{flushright}


\subsubsection{ME46060 Engineering Optimisation: Concepts and Application}
\begin{minipage}{0.45\textwidth}
\raggedleft
\begin{tabular}{r|c|c}
\textbf{指标} & \textbf{数据22-23}& \textbf{数据21-22} \\ \hline
通过率 &\% & 96.2\% \\ 
平均分 & & 7.82 \\ 
参与学生数&  & 131 \\
学分 & & 3 EC\\
\end{tabular}
\end{minipage}\\
%张先治,待补充

\subsubsection{BM41050 Applied Experimental Methods: Medical Instruments}
%祁晨晨
\begin{minipage}{0.45\textwidth}
\centering
\begin{tikzpicture}
\begin{polaraxis}[
    width=0.65\textwidth,
    height=0.65\textwidth,
    yticklabels={},
    xtick={0,60,...,300},
    xticklabels={知识性, 授课质量, 作业友好,工作量适宜,易通过, 有趣},
    ymin=0,
    ymax=5,
    ytick={1,2,...,5},
    y grid style={gray},
    grid=both,
    minor grid style=gray,
    major grid style={gray},
]
\addplot[mark=*, ultra thick, black, data cs=polar] coordinates {
    (0,4)  % 知识性得分,这里修改为你的分数
    (60,4)  % 授课质量得分,这里修改为你的分数
    (120,5)  % 作业友好得分,这里修改为你的分数
    (180,5)  % 工作量得分,这里修改为你的分数
    (240,5)  % 易通过得分,这里修改为你的分数
    (300,4)  % 有趣得分,这里修改为你的分数
    (360,4)  % 为了闭合图形,最后的一个点应该和第一个点一样,这里修改为你的分数
};
\end{polaraxis}
\end{tikzpicture}
\end{minipage}%
\begin{minipage}{0.45\textwidth}
\raggedleft
\begin{tabular}{r|c}
\textbf{指标} & \textbf{数据} \\ \hline
通过率 & 无数据 \\ 
平均分 & 无数据 \\ 
参与学生数 & 无数据 \\
学分 & 4 EC\\
\end{tabular}
\end{minipage}\\

这门课是个大水课,正课只有三次,之后便是自行做实验(在家实验,约耗时3h)与四页的实验报告。这就是这门课的全部内容。任课教师原话:“之前好多同学都反映这门课4个EC太多了,它只值2个EC。但是我觉得既然我们的课程已经如此紧张了,为何不用这门课来平衡一下呢?”

通过这门课,可以了解整个科学实验从实验设计到论文写作的全过程,对今后的学习研究有很大帮助。
\begin{flushright}
祁晨晨; 07/06/2023
\end{flushright}

\subsubsection{BM41130 Tissue Biomechanics}
%祁晨晨
\begin{minipage}{0.45\textwidth}
\centering
\begin{tikzpicture}
\begin{polaraxis}[
    width=0.65\textwidth,
    height=0.65\textwidth,
    yticklabels={},
    xtick={0,60,...,300},
    xticklabels={知识性, 授课质量, 作业友好,工作量适宜,易通过, 有趣},
    ymin=0,
    ymax=5,
    ytick={1,2,...,5},
    y grid style={gray},
    grid=both,
    minor grid style=gray,
    major grid style={gray},
]
\addplot[mark=*, ultra thick, black, data cs=polar] coordinates {
    (0,5)  % 知识性得分,这里修改为你的分数
    (60,4)  % 授课质量得分,这里修改为你的分数
    (120,3)  % 作业友好得分,这里修改为你的分数
    (180,2)  % 工作量得分,这里修改为你的分数
    (240,1)  % 易通过得分,这里修改为你的分数
    (300,4)  % 有趣得分,这里修改为你的分数
    (360,5)  % 为了闭合图形,最后的一个点应该和第一个点一样,这里修改为你的分数
};
\end{polaraxis}
\end{tikzpicture}
\end{minipage}%
\begin{minipage}{0.45\textwidth}
\raggedleft
\begin{tabular}{r|c}
\textbf{指标} & \textbf{数据} \\ \hline
通过率 & 70.1\% \\ 
平均分 & 6.45 \\ 
参与学生数 & 97 \\
学分 & 3 EC\\
\end{tabular}
\end{minipage}\\

十分硬核的一门课,其既设计生理知识,又涉及计算。简而言之,不论你本科背景偏生物还是偏工程,这门课总有一部分会让你很难上手,甚至含泪挂科,在第一个学期给你来自TUD的第一记暴击。第一次考试之后仅有60\%的通过率,经过不懈的补考,又有10\%的朋友挣扎上岸,即使上岸,成绩也惨不忍睹,堪称TUD高压适应性课程。

抛开成绩不谈,这门课的知识性毋庸置疑,为BME/BMD的同学提供了有关组织与组织模型的详尽的生理与工程知识,还是很值得一学的(反正大家都过不了,相当于大家都过了)。

课程内容包括信息量极大的讲座若干,两次不用提交但不写期末傻眼的练习,以及两次讲解练习的习题课。由于本课开课仅两年且因为成绩爆炸导致评教哀鸿遍野,任课教授正不断调整课程设置,具体课程内容可能有较大变化。
\begin{flushright}
祁晨晨; 07/06/2023
\end{flushright}
