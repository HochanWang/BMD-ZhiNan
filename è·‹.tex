\chapter{跋}
将近三个月,一共三万多字;二零二三年七月的中旬;这份ME的指南终于写到了结尾。

开头时候,呓语一样的序言,并不一定全与本作相关,却是徘徊已久的想法,借了序言的幌子:那里我说,我们不可避免总是坠落又坠落;被浪潮被洪流,被谁都听得见的问题拖曳着。我说我不希望这样,自以为是的殉道者,自大而天真,貌似十分牛逼的腔调。

写到最后,自己也不知道了。越是写下去,越是衰老越是困顿与无聊。相较于自己的惰性;理想主义的呼喊,到底麻雀一样被吹散,盘旋几下掉在地上。感觉自己像是秋天挂在晒架上破了洞的臭袜子;斜放的太阳,日光温暖;秋风一吹,充盈隆起,慵懒又萧瑟地摆动。

但最终还是写完了,除开因为曾经和太多人吹下的七月写完的牛逼,那些自我幻想里来自他者的期待;还有的便是太多共同使此作出笼的朋友们、同学们的支持,他们贡献了许多词条,于课程又于生活;我十分感动且感谢!

总的来说,这本指南其实还带着些遗憾:它结构零零碎碎;没有什么大纲的指导,写着写着,藤蔓一样,旁逸斜出。本身笔者理想中的指南,或许应像旧时代说书先生话本一样,醒木一拍,不紧不慢地娓娓讲掉关于第一年应当知晓的一切;蹲在极低的位置上,全心全意的为读者服务。结果却因为懒惰多有省略,甚至将全册子编排得,更貌似了字典。

不过作为作者,有些狂妄地讲,我还是对这些粗劣有着相当的满意。我也是幻想着,幻想着或许它也能被作为读者的你欣赏。甚至我还胆大包天地期望,这样的指南可以在一年一年里被不断地更新,修改,又传承。有关它所有的一切内容,都于序言里阐明了开源路径。各种可能下,不知道在多年以后;凋落枯萎,或是化作春泥,曾经这一版里,来自过去的文字还会剩下多少?

再次感谢所有为本作做出过贡献过的朋友们、同学们。非常感谢你们!

\begin{flushright}
王昊辰;12/07/2023 于无锡
\end{flushright}