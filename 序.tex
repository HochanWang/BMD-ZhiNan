\subsection{我的呓语}
已经一年了,很多时候,我都还在疑惑自己当时何以来到TUD,我那时候以为自己想要解开谜团,探索未知;幻想自己以汲取知识为裨益.我以为当时所在的英国本科满是混沌,我把希望与憧憬寄托在了这里,幻想里的乌有乡;和多少年前踏进小学,走入初中,到达高中,又闯入大学的一轮轮没有多少分别。和那时候又相似的,是总算抵达,却发觉究竟围城。

回头看看,是真的不知道自己决定背后的所以然。我似乎总是在自动驾驶,顺水而行的决策里面,度过了至今天大部分的岁月。感觉自己多少一直带着些悲观主意:那拉普拉斯的妖怪,一切的运动规律,过去将来,我虚无缥缈的主观能动,真的在宇宙大爆炸的一瞬间已经注定?《繁花》里面,小毛临死前讲,“上帝不响,像一切全由我定。”上帝不响,命运喧嚣;方舟里面,多少人竖起来石像;一轮又一轮的暮霭,岁月风化,层层叠叠。最后总归全部黑夜漫游,各奔归途。

但作为个人,我还是出发,还是启航。我继续满怀希望,依旧葬身大海。别人跟我讲,南岛民族其实是百越后裔;多少个千年过掉了,轮回又轮回,我的血脉,似乎和他们到底没多少不同。

总是觉得应该写个序,键盘敲敲,这些句子,垂头丧气。留学出发的行前,记忆里激情澎湃,几多焦虑;“This time tomorrow, where will we be ?” 那时候的我这样想着。如今的我回头望望,一天一天,过去的未来,浑芒光晕拨开,其实还是周而复始。地势坤,前路多少漫漫:这此刻行前的兴奋,清单里的未完成,辗转反侧的忧虑恐惧或其它;终归会在autopilot的飞驰里面;和自己无数已然经历的过去一样,巨大动量下,顺其自然间,不知不觉一溜烟全部轰隆完成离去。

有点感冒,又恰了点小酒,边写边默念着,嗓子都又有点哑掉了。星斗吱呀,杯酒涣散。有的时候自己真的都泄气了: 这世道普遍失落,其间眉头我终归解不开的,也懒得假装关心了。但又总自以为是,觉得力所能及的很多方面,自己多少能给点指引,帮忙散开些噪声氤氲的未知。

于是就有了这指南。尽管,我清晰地明白,无论如何,终归生来的宿命就是一而再的跌落又吹散;像陨石或者麻雀。西欧平原,风车旋转的确依旧;秋风日暮,时而为飞蛾,时而幽暗烛火。蜉蝣大树,我所想的,只是让这沧海一粟的我们,坠落得慢点再慢点。
\begin{flushright}
王昊辰;27/05/2023
\end{flushright}

\vspace{\betsubsec} %section间留白
\subsection{一些说明}
这支指南;虽说以BMD命名;但或许于其它许多专业也应该是比较适用的。如封皮所言,所有内容,若非特地说明,则全部在\ccbyncsa\ \href{http://creativecommons.org/licenses/by-nc-sa/4.0/}{\uline{CC BY-NC-SA 4.0}}的License下。于遵守license要求的情境里,还请随意按照自己意愿需求更改使用。笔者乐于看到此作的衍生;希望它能够帮助更多的人;幻想它在长久未来里,一年又一年地不断丰富更新,于是流芳;帮笔者在赛博世界里面留下点今生今世的证据。

指南里的内容,上下辑两部分:其中上辑全部学校以内,有关选课,也有关于第一年以外的实习与毕设;下辑则更多是生活方面,办银行卡,去医院,超市优惠之类。

\vspace*{0.7cm}
所有关于这本小册子的Latex代码,文字内容;全部都可点击如下链接查看:
\begin{enumerate}
\item \href{https://github.com/HochanWang/BMD-ZhiNan}{\uline{Github}}
\item \href{https://www.overleaf.com/9181971742xyhrjrdxdzkt}{\uline{Overleaf(随着了笔者的毕业,可能失效)}}
\item Google Drive
\end{enumerate}

当然,纸质版也可通过亚马逊进行购买(链接之后放出)。


\vspace*{0.7cm}
而若是有私人问题需要联系的话,还请发邮件或添加笔者的社交媒体账号。笔者是2022-2023学年就读Delft的;如果有幸这份文档传承许久,多年以后或许我关于Delft的印象会模糊过时,不确定是否还会受用。但笔者如果还在人世,不出意外总归还是尽量回答的,联系方式如下:
\begin{enumerate}
\item 邮箱:\href{mailto:whch.o@outlook.com}{whch.o@outlook.com}
\item 微信: b190552589
\item QQ: 506007201
\end{enumerate}

\begin{flushright}
王昊辰;17/06/2023
\end{flushright}
