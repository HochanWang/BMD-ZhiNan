\section{下辑:生活杂谈}
\subsection{公共交通\&交通卡}
荷兰的公共交通系统,有些类似日本;各种轨道交通间并没有泾渭分明的界限。火车的运作方式有些类似大陆地铁,电车又时而跑去地下。各种交通系统也都仅仅需要一张OV Chipkaart(OV卡)就可以互通换乘。

OV卡,分为三种:匿名,实名,以及商务。

匿名卡的运作方式和上海紫卡及大多数大陆普通交通卡类似,不记名;需要7.5欧工本费押金;除开坐火车的Joint Journey Discount 以外没有其它优惠。2023年的某日起。普通芯片银行卡,包括荷兰许多地方不支持的Visa或MasterCard芯片卡;都可以实现匿名OV卡的大部分功能的。使得匿名卡略显鸡肋。

久居荷兰的话,个人觉得其实还是自己办理一张实名OV卡比较划算;作为荷兰痕迹的纪念品回国,笔者觉得也挺有意义的。实名卡也需要工本费,且需要拥有荷兰银行卡及BSN才可以办理。它的优势在于可以在交通公司网站上购买诸多优惠套餐,经常出行的话比较划算。当然,其实估计大家也发现了;许多个人会私下里出租自己购买了优惠套餐的交通卡;这的确可行,但又实在违规。自行判断,不做评价了于此笔者就。

至于商务卡,若是二年级实习的公司较为慷慨的话;可能会提供。但大多数学生其实比较难接触到吧。

刚刚来到荷兰,若是没有芯片银行卡;可以在手机APP上购买火车票(其实APP购票似乎还经常有优惠)。不同于英国或大陆的是,荷兰的火车完全按照距离计费,火车公司并不在意你乘坐坐哪条线路,花多长时间抵达(除了一些额外标注收费的特快线路)。一般情况下,荷兰火车站需要刷公交卡、银行卡、实体票或是APP二维码过闸机。但某些站点,如诸位极有可能抵达荷兰的第一站史基普机场;是没有闸机的。如此情况下,刷卡的话不要忘记在一根黄色柱子上上进站;否则就会在出站时候,默认你从火车的首发站点上车,花费整条行路的大价钱。但若是手机购票或是实体购票,就直接上车好了;无需担心。

以上泛泛关于一般意义的公共交通;但除此,就出租车与OV Fiets笔者还有些牢骚。就以下再单独拎出来两个小标题稍微讲一下罢。
\begin{flushright}
王昊辰; 05/06/2023
\end{flushright}

\subsubsection{出租车}
之所以拎出租车出来单独讲;主要还是源于Delft独特的出租车生态吧。其他欧洲城市非常通行的Uber或Bolt之类的网约车,在Delft非常不灵,或者更准确的说,是完全不可用。笔者经验,2022年8月28日到达Delft火车站,Uber苦等1小时+不见任何车子踪迹。后来也和司机聊过,似乎是Delft本地的出租车工会组织之类,认为uber这类平台的抽成太高了;于是集体抵制。和很多店面不支持MasterCard和Visa Card一样地缘由。

如此,在这样的情境下;在Delft应该如何打车呢? 真的很古早的方式,其实就是电话或者邮件提前预约。这里我就附上我来到Delft时候,\href{https://taxidelfland.nl/}{\uline{司机所在公司的名片网站吧}}。

当然,其实预约也不是特别必要。火车站附近,和大陆一样的是,在工作日的工作时间,往往是有许多出租车聚集在站点附近等着拉活的。在Delft的话,出租车会聚集在火车站旁边公交车场的一侧。值得注意的是,荷兰的出租车并不像大陆会有明确的样式:它们没有统一的涂装,也没有统一的立牌竖在车顶(事实上好多立牌在车顶的车会是驾校的训练车)。荷兰出租车与私家车的唯一区别,就是他们的车牌照会是深蓝色的底色。于普通私家车,牌照会是深黄与白色作底。

而且,鉴于荷兰的出租车,若是轿车,车型往往较为高档(似乎奔驰E级为普遍)。两个28寸的行李箱,若是后备箱塞不下;司机很有可能不会同意你把行李箱放在后座。但这里也时常有类似Uber XL的MPV作为出租车在运营,所以果然若是有MPV,还是将它们作为第一选择吧。但荷兰的出租车不同于大陆许多城市与香港,并没有所谓的行李搬运费;司机帮忙搬运行李是没有必要付小费或任何额外费用的,放心叫他们帮忙就好。
\begin{flushright}
王昊辰; 07/06/2023
\end{flushright}

\subsubsection{OV Fiets}
游玩别的城市,或是友人来访。可能也需要接触到OV Fiet的出租吧。Fiet也就是Bike的荷兰语;其实就是火车站都有的,以实名OV卡进行脚踏车日租的服务。收费以目前2023年6月而言,是4.45欧元24小时。

值得注意的是,这项服务是需要和NS的其它套餐一样需要在网上提前订购并机器上加载的。这里的订购并不是说提前购买预付多少
小时的租车额度;而是指对OV Fiet这项服务进行开通;仅仅开通并激活服务的话,就是字面意思的0欧元一年。

租的车子,富的话完全可以跨站点甚至跨城市归还。例如Delft租的车子,在海牙HS归还也是可以的;只不过会收取10欧元的手续费。而租借,归还流程本身,因为车型各有不同,实在多种多样。就Delft站而言,在2023年6月这个节点,所有的车子都是机械锁,租借归还完全人工。
\begin{flushright}
王昊辰; 07/06/2023
\end{flushright}

\subsection{银行卡}
在Delft,一般比较常见外国人办理的线下银行就是三家:ING,ABN以及Rabobank。Rabobank和ING都在Delft有线下支行,但是ABN没有。三家各有优劣吧。Rabobank的支行比较大,但是它的APP没有英文版,官网也是。ING有支行,而且个人觉得学生套餐的政策比ABN好一些,卡面也是纯橙色(笔者感觉)比较好看;但ING办理需要在市政厅注册时花费10.99欧元开具Uittreksel BRP (也有传闻说2023年新学期可能不需要了)。这也是额外成本,总的来说各有优劣吧。

荷兰因为支付系统独树一帜,虽然在逐渐改进,但在许多地方仍然不接受Visa或MasterCard;因此,拥有一张当地的Vpay或Maestro卡还是很有必要的。正因如此,如果还没有Visa或MasterCard芯片卡,去欧洲其他国家时可能会陷入无卡可用的境地。为了解决这个问题,可以另外申请一张网上银行的Visa或MasterCard卡。在荷兰,选择并不是很多,截至目前(2023年6月),常见的选择有BUNQ、N26和Revolut。虽然Revolut似乎使用的人比较多,但笔者主观上并不太推荐大额使用:与N26、BUNQ或Curve相比,它的欧元兑英镑、瑞士法郎、兹罗提汇率都不太理想。如果要用瑞郎购买奢侈品,3000法郎就能差十几二十欧元了。然而,Revolut的好处在于可以同时开设多种货币的钱包,这在接收潜在的英国同学转账给你的英镑等方面具有一些独特的作用。

鉴于网上银行的卡开设几乎零成本,甚至还有时会有返现活动,笔者个人觉得开设多几个也无妨。然而,线下银行的开设确实需要一些手续,这里简要介绍一下线下银行的银行卡开设流程。
\begin{flushright}
王昊辰; 04/06/2023
\end{flushright}

\subsubsection{ING}
在Delft,ING一共有2个网点。其中一个位于市中心火车站对面,非常显眼;另一个位于Roland附近的一家书店里面。柜台甚至不如超市试吃摊位,非常隐蔽,而且这个网点的ING员工似乎并不全职,对于开卡的很多问题她自己也不清楚。建议直接去火车站那家网点。

ING开卡的话,很可能需要使用到前文提到的Uittreksel BRP。这个文件在你注册时和市政厅的工作人员交流一下,他们就会了解。这是一张克数比较大的A4纸,上面是你的身份信息,触感有点类似奖状。带上这张BRP,以及学校的注册证明(如果打算在抵达时就去办卡,建议提前在大陆打印好;当然也可以在宿舍找好心人帮忙打印),再加上护照,就足够去开卡了。

在荷兰,ING的营业厅和大陆的风格完全不同,更像是一个咖啡厅,没有窗口也没有长长的队伍。进入营业厅,直接找个空位子坐下来就好。8、9月高峰期,或许可能会有指示牌写明学生需要坐在哪里等待开卡;如果有指示,按照指示行事,如果没有指示,随便找个座位坐着就行。大厅里还有咖啡机,可以给自己泡杯茶。就像在餐厅一样,等待营业员们忙完手头的顾客,他们自然会前来接待你。

然后就是与营业员交流,并按照流程开卡。但是有一点需要特别注意。在开卡过程中或开卡结束后(具体我不记得了),可能营业员会要求你使用手机的NFC扫描护照以进行激活;然而,如果你使用的是国行版本的iPhone,其NFC是无法读取护照信息的。这个问题在欧版手机上并不存在,所以大多数营业员并不会考虑到这一点。如果遇到问题,还请一定记得寻求帮助。

开卡后,几个工作日之内,你的卡片应该会寄到你的邮箱。收到卡片后,不要忘记将套餐改成学生套餐,这样可以免除每月几欧的手续费,并享有12次免费提款。

此外,血泪教训;还有一点额外特别注意:ING的密码是4位的,这是不可更改的!荷兰的ATM机通常不会提供更改密码的选项。然而,在海外,以笔者的经历为例,英国的ATM机可以更改ING的密码。可是一旦你手贱操作,更改密码,整个卡的信息都会乱掉。结果像笔者一样,旧密码和新密码都无法使用,直接废卡一张了;最好只好申请新卡补办,非常麻烦。请务必记住不要随意更改密码!
\begin{flushright}
王昊辰; 04/06/2023
\end{flushright}
\subsubsection{ABN AMRO}
ABN AMRO的储蓄卡是全程线上办理的不需要去线下网点(虽然它也没有线下网点),全程在它的app上操作。完成注册之后,它会分批次的把电子密码器,银行卡以及PIN码邮寄到你的邮箱。之后在电子密码器上激活使用。
\begin{flushright}
张先治; 07/06/2023

\end{flushright}
\subsection{就医与医保——以Insure2Study体验为例}
荷兰的学生保险有很多选择。大多政策都会非常清爽地写明;这里笔者也就强调一点很多官网模棱两可,关于旧病的政策方面吧。如果说,在来到荷兰之前,就有着一些固有疾病(除了近视)的复查或后续治疗,除了学校推荐的insure 2 study外,其它国际生里比较流行的OOM与AON都是不会提供报销的。

此部分的GP,以SGZ为例;而医院,笔者仅仅光顾过代尔夫特的Reinier de Graaf Gasthuis。 这里的就诊体验,仅能代表笔者于2023年上半年,携insure 2 study保险的经历;并不代表其它。

关于急诊,笔者并未经历;但较为严重如骨折等,大概可以直接前往Reinier de Graaf Gasthuis(\href{https://reinierdegraaf.nl/specialisme/spoedeisende-hulp}{\uline{见此链接}})。不严重,但GP不上班有点急,可以求助于\href{https://hapschievliet.nl/language/english/}{\uline{Huisartsenpost Delft}}。
\begin{flushright}
王昊辰; 04/06/2023
\end{flushright}

\subsubsection{GP}
在Delft,若是注册在SGZ,预约还是比较方便的:拨打015-7999050按提示操作。SGZ采取回拨制度,首次拨打以后,会在有空时给你回电。笔者的经验,回电往往半小时到一小时内就能接到。在电话里,告诉对方自己的基本情况,对方就会安排预约时间了。个人感觉SGZ的看病流程很快,大概率能预约到两三天以内,甚至第二天或当天的时间段。

到达预约时间时,需要在前台确认身份并付款。用于报销的材料,工作人员会用订书机为你装订好。回家后,扫描这些材料并附上填好的Insure2Study的报销单,通过邮件发送给保险公司即可完成报销操作;保险公司一段时间后会直接将钱打到银行卡上。

在SGZ,付好钱。当预约的时间到来时,医生会把你请进办公室。这方面和大陆没有太大分别。如果情况复杂,可能会进行转诊;转诊所去的大医院大概率就是代尔夫特的Reinier de Graaf Gasthuis。医院会给你发送邮件,进行一些基本的注册和身份的确认;而后会有信函寄送到你的信箱,按照信件所示时间前往就行了。
\begin{flushright}
王昊辰; 04/06/2023
\end{flushright}

\subsubsection{大医院}
携带医院所寄信函到达Reinier de Graaf Gasthuis,首先要进行就诊卡的注册。医院很大,堪比3me教学楼的迷宫,建议首次前往最好提前至少15-30分钟。 注册就诊卡的地方在大门进门的左手边,比较隐蔽,可以向前台询问,让工作人员进行指引。就诊卡的注册需要先取号,取号机器完全荷兰语,同样建议咨询工作人员。

拿到就诊卡以后,前往此前信函上所指示的科室。要注意,到达科室以后也需要在机器上进行签到;这里的机器又是完全荷兰语,且需要进行一些较为复杂的身份信息确认,每个界面甚至还有倒计时,手机翻译其实不太来得及。笔者强烈建议这步拉下脸皮寻求附近工作人员的帮助。注册完成后,就是根据所拿到的号码等待进入科室了。由此,就和大陆看病没有什么不同了,只不过等待时间完全不会像大陆那么长罢了。

从诊室出来,有可能要向前台或其它地方进行一些其它方面的操作。这里因人而异,一定记得和医生问问清楚哇。
\begin{flushright}
王昊辰; 04/06/2023
\end{flushright}

\subsubsection{配眼鏡——以Specsavers为例}
其实就算没有保险,笔者个人感觉在荷兰配眼镜还是相当划算的:验光相比国内专业很多(至少比JINS验光项目多很多);虽然相比国内小眼镜店价格略贵,但其实也与JINS或ZOFF等连锁店价格在同一水平线。尤其对于高度数人群,1.74的镜片价格比较实惠(于2023年6月,260欧2副pentax的1.74)。

若是有了保险,价格优势会更加明显些。以Insure2Study为例,在满足保险要求的情况下;可以单次报销150欧元,价格足以购买2副Specsavers的普通镜框和普通1.61镜片了。但满足保险,其实也有一定要求,主要可以概括为如下三项:

\begin{enumerate}
\item 需要有相关证明,证明你在本次配镜以前就近视了。
\item 距离上一次配镜,时间已经过去超过一年。
\item 相较上一次眼镜的度数,这次屈光度上调超过0.5(50度)。
\end{enumerate}

这三份材料可以是同一文件,也可以单独文件;只要足够证明就好。Insure2Study并不要求这些文件需在荷兰开具;笔者自己所使用的就是2021年在英国读书时候的验光配镜单。但笔者也并不清楚大陆或别的地区以华文写就的材料是否能被Insure2Study顺利认识;可能可以自己咨询一下客服吧。

配镜,至少Specsavers是需要预约的;但slot非常疏松。基本工作日一小时后的slot都是空的,属于可以一拍脑袋就去配镜的程度。Specsavers的镜框与镜架需要单独购买,镜架价格在29欧元到239欧元不等;镜片也是丰俭由人。但这些价格,无论镜片还是镜架,至少在2023年6月这个时间为止;其实都是双份价格,买一送一;也就是最终你都会收到两幅眼镜(镜架在相同价格和系列可以不一样)。
\begin{flushright}
王昊辰; 08/06/2023
\end{flushright}

\subsection{超市优惠——以AH为例}
荷兰的超市,感觉相比大陆;优惠花头经多得多,且往往优惠也会更在实在一些;不盯着优惠或者打折,属实纯纯富姐(哥)了。打折标签,种种优惠活动之类到底什么意思;虽然漫漫生活里自然就会熟悉;然而果然还是先讲述一下比较好吧。如此刚来到荷兰的日子,也可以少花点冤枉钱,逛超市时候也不至于一直手机搜索,移动路障。

由于笔者光顾Roland 楼下的Albert Heijn (AH)比较多些;这里就主要以AH举例了。Delft主流超市,也就是Dirk, AH, Jumbo了;这三家打折标签大同小异;不过AH的APP优惠词条确实仅仅适用于AH。此外也有Lidl,Aldi,麦德龙之类的;笔者并未曾前往多少次所以并不了解。

需要注意的是,AH和Jumbo的优惠,都必须在办理免费的会员卡的前提下才可以被探用,Dirk是不需要的。AH的会员卡称之为‘Bonuskaart’ (Bonus Card); 而Jumbo的一般称为Jumbo Extra's。跟服务台营业员说一下就可以当场免费领取;绑定手机APP以后可以查看积分,以及探索更多APP上面的优惠。另外AH的APP也可以提前在周三时候看到下周的优惠讯息,提前规划购买。

使用各超市的会员卡,也可以方便的使用Hand Scanner 进行结账。Hand Scanner于大陆笔者想不出来对应的产物;大概就是一个扫码枪,一路挑选一路直接扫码:总价,优惠,和你买的东西都会由此显示到其屏幕上,一目了然,不过也不要忘记不想要的东西放回去了在扫码枪上删除。使用扫码枪的话,偶尔会在结账时候被收银员随机抽查;大概就是他们再扫一遍看看你有没有蒙混;身正不怕影子歪。

各个超市Hand Scanner的操作略有不同; 租借都是在机器上扫描会员卡;但结账归还,AH是先归还扫码枪,再在自助收银及扫描会员卡。而Jumbo,则是先使用扫描仪扫描自助收银机的二维码,再归还扫码枪。 另外AH的扫码枪更加智能一些,饮料瓶退换的押金条可以直接用扫码枪扣除,结账总经额也会因此甚至自动变为负数;但Jumbo的押金条,却只能在收银机上扫描。

AH的Hand Scanner \href{https://www.youtube.com/watch?v=7MSqE_vt5Po}{\uline{教程可点此处查看(荷兰语,但youtube有自动翻译)}}。
\begin{flushright}
王昊辰; 04/06/2023
\end{flushright}



\subsubsection{常见优惠名词}
AH的优惠,一般都会十分醒目的以橙色标牌标出。除了一般优惠以外,还会经常有临期食品-35\%的贴纸,这种额外优惠非常简单易懂;这里也就主要讲讲橙色优惠里面,较为常见的荷兰语名词了。

\begin{flushleft}
\begin{tabular}{p{0.15\textwidth}|p{0.8\textwidth}}
\textbf{荷兰语} & \textbf{中文解释} \\ \hline
Korting & 就是cutting的意思。例如 "€1 Korting" 意味着该商品在原价基础上减1欧元。 \\ \hline
Gratis & 直译为免费,一般会跟其它东西搭配描述;比如表示购买一打啤酒可以免费获得一个杯子什么的。偶尔也会有新产品的小样什么的,放在那里标个Gratis;免费让你拿了试吃试用。 \\ \hline
2e Halve Prijs & e是序数词标记;可以直译为2nd half price。顾名思义,第二件半价。 \\ \hline
n+1 Gratis & 如前所述Gratis意为免费,此处意思就是买n件送1件;值得注意,比如买1送1的话,不是说只要扫描1件就好了,还是需要扫描2件商品的;只不过第二件会自动价格扣掉免费。 \\ \hline
Op=Op & 无关原友。OP这里可以理解为最后机会;这里意思就是,最后机会=最后机会;卖完再也没有了。约等于大陆清仓优惠。 \\ \hline
x voor €y & voor, 读音就可看出,意思是x for €y;买x件总价只要y欧元。 \\ \hline
x voor y & 偶尔也可以看见这种;这里的x与y一般都是整数,不带小数点也不带欧元标志。这里的意思就是x件商品只算y件的价格。例如3 voor 2, 其实和2+1 Gratis一个意思。 \\ \hline
Prijsverlaging & 价格降低,表示商品已经被永久或长期减价。 
\end{tabular}
\end{flushleft}
\begin{flushright}
王昊辰; 04/06/2023
\end{flushright}

\subsubsection{APP优惠:Bonus Box}
Bonus Box; 是AH的APP上最醒目的。简单来说,就是AI会猜你喜欢;在一般优惠以外,可以在这里勾选额外的5样你想要的专属优惠;如果氪了会员,则有10件。
\begin{flushright}
王昊辰; 04/06/2023
\end{flushright}

\subsubsection{APP优惠:Koopzegels}
Koopzegels这个东西蛮复杂的;很多人甚至会没发现自动开通了这项优惠,导致每次给超市白白多送10\%的钱。

Koopzegels,翻译过来,意思大概就是“集邮”或者印章。有点像小时候的奶茶店,买一杯贴一个贴纸,贴满多少个贴纸送你一杯奶茶的工作原理,但又有所不同。简单来讲,开通了这个优惠,在AH每消费1欧元,就会自动多花0.1欧元购买一个Koopzegel(也就是多花购物金额去除小数部分的10\%)。

当你集满490个Koopzegel时;AH就会溢价8\%(2023年6月)回购你持有的Koopzegels。换言之,即在AH每消费满大约490欧元,就可以一次性体现53欧元,AH会多送你4欧元。在手机APP上可以实时追踪你的集邮进度。提现操作一般就去服务台就可以了,可以选择现金或者打到你的银行卡上。
\begin{flushright}
王昊辰; 04/06/2023
\end{flushright}
