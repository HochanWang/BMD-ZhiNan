\section{ Illustrative Course Selection Examples}

第一年的选课总额,应是60分或更多;如此带来了许多组合方案,不同的就读体验。这里就放上些已经真实经历的选课方案,并附以就读体验及点评。
\begin{flushright}
王昊辰; 08/07/2023
\end{flushright}

\subsection{Example A}
\begin{minipage}{\linewidth}
\begin{ganttchart}[
    hgrid,
    vgrid,
    x unit=0.3cm, % adjust this value to fit the chart within the page
    y unit chart=0.7cm % adjust this value to fit the chart within the page
]{1}{25}
\gantttitle{Q1}{5} \gantttitle{Q2}{5} \gantttitle{Q3}{5} \gantttitle{Q4}{5} \gantttitle{暑假}{5} \\
\ganttbar{\parbox{3cm}{\hyperlink{control}{\uline{SC42001 5EC}}}}{1}{5} \\
\ganttbar{\parbox{3cm}{\hyperlink{ME41096}{\uline{ME41096 5EC}}}}{1}{10} \\
\ganttbar{\parbox{3cm}{\hyperlink{WM1401TU}{\uline{WM1401TU 3EC}}}}{1}{5} \\
\ganttbar{\parbox{3cm}{\hyperlink{ME41065}{\uline{ME41065 7EC}}}}{1}{10} \\
\ganttbar{\parbox{3cm}{\hyperlink{ME46055}{\uline{ME46055 4EC}}}}{1}{5} \\
\ganttbar{\parbox{3cm}{\hyperlink{RO47006}{\uline{RO47006 5EC}}}}{6}{10} \\
\ganttbar{\parbox{3cm}{\hyperlink{WI4771TU}{\uline{WI4771TU 3EC}}}}{6}{10} \\
\ganttbar{\parbox{3cm}{\hyperlink{ME46085}{\uline{ME46085 4EC}}}}{6}{10} \\
\ganttbar{\parbox{3cm}{\hyperlink{ME46007}{\uline{ME46007 3EC}}}}{11}{15} \\
\ganttbar{\parbox{3cm}{\hyperlink{ME41085}{\uline{ME41085 4EC}}}}{11}{20} \\
\ganttbar{\parbox{3cm}{\hyperlink{BM41040}{\uline{BM41040 5EC}}}}{11}{20} \\
\ganttbar{\parbox{3cm}{\hyperlink{ME41055}{\uline{ME41055 4EC}}}}{11}{20} \\
\ganttbar{\parbox{3cm}{\hyperlink{BM41155}{\uline{BM41155 4EC}}}}{11}{15} \\
\ganttbar{\parbox{3cm}{\hyperlink{ME46014}{\uline{ME46014 4EC}}}}{11}{20} \\
\ganttbar{\parbox{3cm}{\hyperlink{ME41006}{\uline{ME41006 4EC}}}}{16}{20} \\
\ganttbar{\parbox{3cm}{\hyperlink{ME46060}{\uline{ME46060 3EC}}}}{16}{20} \\
\ganttbar{\parbox{3cm}{\hyperlink{ME41035}{\uline{ME41035 3EC}}}}{21}{25} \\
\node [anchor=west] at ([xshift=3.5cm]current bounding box.south west) {\textbf{总学分 = 70 EC}}; % Total EC
\end{ganttchart}
\end{minipage}
\vspace{\betsubsec} %section间留白

这个选课方案稍微有点问题的,Q3,Q4的学习压力太大了,应该尽可能减少在Q3,Q4的选修课,尽可能地把时间花在必修选课上。Q1,Q2的大部分课还是比较轻松的,可以多选点纯考试并且往年通过率较高的课。

\begin{flushright}
张先治; 04/07/2023
\end{flushright}

\subsection{Example B}
\begin{minipage}{\linewidth}
\begin{ganttchart}[
    hgrid,
    vgrid,
    x unit=0.3cm, % adjust this value to fit the chart within the page
    y unit chart=0.7cm % adjust this value to fit the chart within the page
]{1}{25}
\gantttitle{Q1}{5} \gantttitle{Q2}{5} \gantttitle{Q3}{5} \gantttitle{Q4}{5} \gantttitle{暑假}{5} \\
\ganttbar{\parbox{3cm}{\hyperlink{control}{\uline{SC42001 5EC}}}}{1}{5} \\
\ganttbar{\parbox{3cm}{\hyperlink{RO47006}{\uline{RO47006 5EC}}}}{6}{10} \\
\ganttbar{\parbox{3cm}{\hyperlink{ME46007}{\uline{ME46007 3EC}}}}{11}{15} \\
\ganttbar{\parbox{3cm}{\hyperlink{BM41040}{\uline{BM41040 5EC}}}}{11}{20} \\
\ganttbar{\parbox{3cm}{\hyperlink{ME41055}{\uline{ME41055 4EC}}}}{11}{20} \\
\ganttbar{\parbox{3cm}{\hyperlink{ME41006}{\uline{ME41006 4EC}}}}{16}{20} \\
\ganttbar{\parbox{3cm}{\hyperlink{ME46006}{\uline{ME46006 4EC}}}}{1}{4} \\
\ganttbar{\parbox{3cm}{\hyperlink{ME44210}{\uline{ME44210 3EC}}}}{1}{4} \\
\ganttbar{\parbox{3cm}{\hyperlink{AFD}{\uline{ME45042 5EC}}}}{1}{10} \\
\ganttbar{\parbox{3cm}{\hyperlink{ME41096}{\uline{ME41096 5EC}}}}{1}{10} \\
\ganttbar{\parbox{3cm}{\hyperlink{ME46115}{\uline{ME46115 4EC}}}}{1}{10} \\
\ganttbar{\parbox{3cm}{\hyperlink{ME41120}{\uline{ME41120 3EC}}}}{11}{15} \\
\ganttbar{\parbox{3cm}{\hyperlink{WI4771TU}{\uline{WI4771TU 3EC}}}}{6}{10} \\
\ganttbar{\parbox{3cm}{\hyperlink{BM41155}{\uline{BM41155 4EC}}}}{11}{15} \\
\ganttbar{\parbox{3cm}{\hyperlink{WM0349WB}{\uline{WM0349WB 3EC}}}}{16}{20} \\


\node [anchor=west] at ([xshift=3.5cm]current bounding box.south west) {\textbf{总学分 = 60 EC}}; % Total EC
\end{ganttchart}
\end{minipage}
\vspace{\betsubsec} %section间留白

这支选课,问题主要在于笔者正巧卡满的60学分,全然没有多少余裕。笔者选课时候想着破釜沉舟,希望以此激励自己,有限的科目里面好好学习不要挂科。最终虽然的确顺利在第一年结束时候修满了所有学分,但或许这样精确,对差错没有宽容的选课,一方面带来了额外的精神压力;另一方面,于一些人偏好来看,或许同样的学费,如此少的学分,性价比并不很高。

学习内容方面,由于笔者始终没有一个明确的学习方向。这第一年的时间里,对未来预期毕设等的领域更多是边学习边探索着,因而选课并未指向某个领域专门优化,更多全凭兴趣,所以看起来比较散乱与随机。

而Workload,笔者体感Q4略显繁忙(但也是可以寻常应对的程度),Q3非常轻松;Q1与Q2的时间分配比较均匀。到了Q3,Q4,半年学习过后,多少都会有些烦躁与Fatigue;同样不是很建议选太多课程;个人感觉这套选课,至少关乎于Q3和Q4工作量的地方,是比较合理的。
\begin{flushright}
王昊辰; 08/07/2023
\end{flushright}


