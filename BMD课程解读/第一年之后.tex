\vspace{\betsubsec} %section间留白
\subsection{第一年以后}
\subsubsection{ME51015 实习}
实习,价值15学分,于BMD而言似乎是大多数人第二年的路径(但根据我HTE和Energy的荷兰同学讲,于他们,15学分的课才是寻常);但也不强制。在2022-2023年的Form2上,它可以与价值15学分的int Interdisciplinary Project (JIP)互换,但其实于Form2以外,似乎任意15学分的课都可以替代掉实习这一项。它的时间长度,受到荷兰法律规定,一般而言会是3个月;但某些特殊的项目,例如Surgery for All里,前往欠发达地区的志愿者项目,会花费更长时间。

其实TUD的语境下,实习一词,比较泛泛。具体所前往的职位或地点,因人而异,不一定非要前往公司里面:也可以是大学研究所,慈善组织的志愿者等等。地点也完全不框定在荷兰以内;可以是世界各地,回国实习甚至也是一种可能。

实习往往需要一位校内人员作为相关过程的监督,来确保你实习时候的工作契合TUD所认为的15EC。也就是说如果不是通过校内职工私人管道找到的实习的话,你还需要私下里寻到愿意监督你的教职工。这并非不可能,笔者认识的人里面是有相关案例的。而若是希望透过校内教授私人关系找寻的话,笔者个人经验果然还是TUD根基比较深厚的老师们资源比较丰富吧;至少我个人接触过的两位prof.抬头的老师,都或多或少有着让他们自信可以塞人进去的熟识公司,而资历较浅的许多普通lecturer,就仅仅会给你提供一个他有所合作的公司表单,仍需要自己去公司申请。这些由老师们提供的职位,往往不会被对应公司列在官网上面,实习的待遇也并非正式职员,据Bob所说一般一个月也就几百欧。有些导师,例如Paul,会把合作公司直接列在自己项目组的\href{https://www.bitegroup.nl/internships/}{\uline{官网上面}};但于更多的教职工而言,什么岗位,什么公司,还是需要和他们正式线下谈话才可以知晓的。

当然实习的supervisor,与后来毕业项目的supervisor不一定同一人物,实习其实也并非一定得时间尺度上排布在毕业项目开始之前。只不过,至少根据我supervisor的说法,同一导师,毕设起始前进行实习会是比较惯常的操作。原因是,某些情况下实习的工作与毕业项目可能有所关联,甚至毕业项目可能是实习期间单位派遣任务的衍生。虽然这样的情况比较少见,但的确存在可能,当然具体情况具体分析,最终还得是与导师商量的结果。

何时开始找实习?这个问题的答案同样极度因人而异。一般来讲,若是出国的项目最好还是留足处理种种文件的时间(Paul的说法是欧盟内2个月,欧盟外半年)。但要是透过导师,寻常的前往境内的公司;很多导师会对自己的速度十分自信。Paul扬言提前2周其实就足矣;而Prof. Jenny Dankelman,根据笔者经验,她也会觉得2周时间就是足够的。然而,鉴于许多教职工回邮件极慢,一周一封的速度并不罕见;而预约见面讨论的slot,也时常会被排布到他们终于回复邮件的一两周以后。鉴于此;其实建议,若是计划透过导师寻找实习,自Q3各个项目组开始在午休时间开设宣讲会的同时,就开始进行(虽然或许在见面时有的导师会抱怨太早了)。
\begin{flushright}
王昊辰; 14/06/2023
\end{flushright}

\subsubsection{TUD4040 Joint Interdisciplinary Project}
JIP项目是一个企业合作的联合项目,项目时间在Q1,持续十二周左右,申请时间一般在前一个学期(Q4)的5到6月,可以作为代替实习/额外课程的15学分。项目的宣讲会和Q4组织的各个实验室的宣讲一起进行。项目内容覆盖不同的领域,具体可以前往\href{https://www.jointinterdisciplinaryproject.nl/}{\uline{项目官网}}/bs上的课程页面查看。报名流程在线上进行,需要提交cv和动机信,提交之后一般来说都会尽量分配第一志愿,基本没有报名但没办法被选中的情况。小组组成之后就不能退出了,实际操作上在七月初之前提出要退出都是可以的。项目要求fulltime,实际工作上和队友商量好的话,可以在这期间参加1-2门课程或是开始literature review,但是要注意考试时间和final review的时间可能会撞上。项目在第一周举行problem definition presentation,在第六周举行midterm presentation,这两次都是不计分,面向相同领域的小组和他们的导师,主要提供交流的空间。评分基于final review和报告,主要是基于前者。同时每两周需要提交小组blog和个人的成长报告,这部分准时提交不要太敷衍一般都会给接近满分(占20\%)。项目工作量和工作形式每个小组都不同,一般来说工作量适中。项目周期很短,成员和企业的导师沟通确定项目方向和预期结果是最初的工作重点。有兴趣的话可以在毕设继续和企业导师合作,接着JIP项目或是开启新项目都是可行的。

\begin{flushright}
刘彦菁; 08/06/2023
\end{flushright}
\subsubsection{IFEEMCS520100 Fundamentals of Artificial Intelligence}
如果你不能找到心仪的实习,恰好对人工智能感兴趣,那么这门课

\subsubsection{毕业项目}
%王昊辰

\subsubsection{导师锐评}
