\vspace{\betsubsec} %section间留白
\section{Towards the Second Year}
\subsection{ME51015 Internship / Research Assignment}
实习,价值15学分,于BMD而言似乎是大多数人第二年的路径(但根据我HTE和Energy的荷兰同学讲,于他们,15学分的课才是寻常);但也不强制。在2022-2023年的Form2上,它可以与价值15学分的Joint Interdisciplinary Project (JIP)互换,但其实于Form2以外,似乎任意15学分的课都可以替代掉实习这一项。它的时间长度,受到荷兰法律规定,一般而言会是3个月;但某些特殊的项目,例如Surgery for All里,前往欠发达地区的志愿者项目,会花费更长时间。

其实TUD的语境下,实习一词,比较泛泛。具体所前往的职位或地点,因人而异,不一定非要前往公司里面:也可以是大学研究所,慈善组织的志愿者等等。地点也完全不框定在荷兰以内;可以是世界各地,回国实习甚至也是一种可能。

实习往往需要一位校内人员作为相关过程的监督,来确保你实习时候的工作契合TUD所认为的15EC。也就是说如果不是通过校内职工私人管道找到的实习的话,你还需要私下里寻到愿意监督你的教职工。这并非不可能,笔者认识的人里面是有相关案例的。而若是希望透过校内教授私人关系找寻的话,笔者个人经验果然还是TUD根基比较深厚的老师们资源比较丰富吧;至少我个人接触过的两位prof.抬头的老师,都或多或少有着让他们自信可以塞人进去的熟识公司,而资历较浅的许多普通lecturer,就仅仅会给你提供一个他有所合作的公司表单,仍需要自己去公司申请。这些由老师们提供的职位,往往不会被对应公司列在官网上面,实习的待遇也并非正式职员,据Bob所说一般一个月也就几百欧。有些导师,例如Paul,会把合作公司直接列在自己项目组的\href{https://www.bitegroup.nl/internships/}{\uline{官网上面}};但于更多的教职工而言,什么岗位,什么公司,还是需要和他们正式线下谈话才可以知晓的。

当然实习的supervisor,与后来毕业项目的supervisor不一定同一人物,实习其实也并非一定得时间尺度上排布在毕业项目开始之前。只不过,至少根据我supervisor的说法,同一导师,毕设起始前进行实习会是比较惯常的操作。原因是,某些情况下实习的工作与毕业项目可能有所关联,甚至毕业项目可能是实习期间单位派遣任务的衍生。虽然这样的情况比较少见,但的确存在可能,当然具体情况具体分析,最终还得是与导师商量的结果。

何时开始找实习?这个问题的答案同样极度因人而异。一般来讲,若是出国的项目最好还是留足处理种种文件的时间(Paul的说法是欧盟内2个月,欧盟外半年)。但要是透过导师,寻常的前往境内的公司;很多导师会对自己的速度十分自信。Paul扬言提前2周其实就足矣;而Prof. Jenny Dankelman,根据笔者经验,她也会觉得2周时间就是足够的。然而,鉴于许多教职工回邮件极慢,一周一封的速度并不罕见;而预约见面讨论的slot,也时常会被排布到他们终于回复邮件的一两周以后。鉴于此;其实建议,若是计划透过导师寻找实习,自Q3各个项目组开始在午休时间开设宣讲会的同时,就开始进行(虽然或许在见面时有的导师会抱怨太早了)。
\begin{flushright}
王昊辰; 14/06/2023
\end{flushright}

\subsection{TUD4040 Joint Interdisciplinary Project}
JIP项目是一个企业合作的联合项目,项目时间在Q1,持续十二周左右,申请时间一般在前一个学期(Q4)的5到6月,可以作为代替实习/额外课程的15学分。项目的宣讲会和Q4组织的各个实验室的宣讲一起进行。项目内容覆盖不同的领域,具体可以前往\href{https://www.jointinterdisciplinaryproject.nl/}{\uline{项目官网}}/bs上的课程页面查看。报名流程在线上进行,需要提交cv和动机信,提交之后一般来说都会尽量分配第一志愿,基本没有报名但没办法被选中的情况。小组组成之后就不能退出了,实际操作上在七月初之前提出要退出都是可以的。项目要求fulltime,实际工作上和队友商量好的话,可以在这期间参加1-2门课程或是开始literature review,但是要注意考试时间和final review的时间可能会撞上。项目在第一周举行problem definition presentation,在第六周举行midterm presentation,这两次都是不计分,面向相同领域的小组和他们的导师,主要提供交流的空间。评分基于final review和报告,主要是基于前者。同时每两周需要提交小组blog和个人的成长报告,这部分准时提交不要太敷衍一般都会给接近满分(占20\%)。项目工作量和工作形式每个小组都不同,一般来说工作量适中。项目周期很短,成员和企业的导师沟通确定项目方向和预期结果是最初的工作重点。有兴趣的话可以在毕设继续和企业导师合作,接着JIP项目或是开启新项目都是可行的。

\begin{flushright}
刘彦菁; 08/06/2023
\end{flushright}
\subsection{IFEEMCS520100 Fundamentals of Artificial Intelligence}
如果你不能找到心仪的实习,恰好对人工智能感兴趣,那么这门课

\subsection{Graduation Project}

今天是2023年6月24号,是笔者来到TUD第一年的期末考试周;这个时间节点,可以想见,笔者还并没有开始毕业项目;但考虑到笔者的确已经确定了毕业项目的方向,谈妥了导师;更重要的是,为了保证这一手册可以在7月中旬发布;笔者还是以第一年的角度,针对挑选项目组,面会导师等的全过程吐点愚见出来吧。

其实来到欧洲以后,很多事情不用特别远视,慢慢跟着似乎大多数人的节奏,事情总会自动解决的。毕业项目也是一样,Q1于Q2,个人感觉没有必要做什么前期的准备;或许有些你潜在的毕业导师会在Q1或者Q2开课,但为此专门的选课笔者愚见以为也没有太大的必要吧:讲师教授的往往是自己的项目相关的学科,若是这门学科提不起兴趣没有选,以此为毕设应该也会多少痛苦吧。

自Q3开始,大约3月中旬,学校会开始各个项目组的宣讲;宣讲的时间是午休吃饭时间,并不提供饮食。当然这些宣讲也都会录像,可是录像,至少在2022-2023学年,似乎是等到所有的宣讲全部完毕的五月份才会在BrightSpance(BS)上放出 (2022-2023年的宣讲可以在BMD的BS页面看见);由此多少会耽误些时间吧。个人认为,听到感兴趣项目的课后,就可以发送邮件给相关老师预约开会了;有些课题会比较火爆,可能等到所有宣讲的完毕,就已经没有剩余位置了。

而这些项目组,他们研究的方向大致确定,但交给你的项目,往往私人定制。约见老师时候其实不用紧张,很多话直说就好;笔者个人讲话,可能由于本科英国的经历,被英国人影响,就多次因为英国式屈曲盘旋委婉讲话被导师误解。这些老师大多一届届学生带得多了,也很明白学生小九九的;比方同时平行的约了多个导师,直接跟见面时候跟导师明说就好,说明自己先了解一下并不一定最终肯定确认前来,完全没有必要藏着掖着,最后难堪。

真正见到了导师,关于实习,关于毕设项目;直抒胸臆地谈话(可能不止一次)以后。或许导师就会给你确认一个课题与实习单位了,这些过程里,大多其实顺着导师的安排就好;当然自己也要发挥主观能动。这些老师其实感觉很忙,有的时候会字面意思的漏掉邮件(我的导师比如,后来跟我诚挚道歉);个人感觉,如果一周都还没有回音,是可以适度发邮件礼貌的提醒一下,或是趁着上班前(9点前的10几分钟大概率可以看到导师前来上班)或午休的时间,去到办公室跟逮住他们当面询问的。

值得注意的是,笔者从与导师的第一次会面到现在,其实导师完全没有在官方层面主动的确认过她成为了我的导师。笔者憋不住,曾经写了一个邮件礼貌的询问一下我究竟是他的supervisee了没;在回复中笔者才得到了肯定的回答。因此个人感觉其实在真正第二年来到,签署毕设的Form前,整个过程其实是比较随意的,就算没有得到导师的官方确认也无需着急,实在不放心发邮件问一下就行了。

于2022-2023学年来看,参加了宣讲会,主要可供选择的大实验室如下:

\begin{center}
\begin{tabular*}{\textwidth}{@{\extracolsep{\fill}} p{0.5\textwidth} | p{0.5\textwidth}}
\textbf{Lab} & \textbf{Members} \\
\hline
Biomechatronic Design \& \\Biorobotics lab & Arno Stienen \& Helke Vallery \\
\hline
Bicycle Lab & Jason K. Moore \\
\hline
BITE & Paul Breedveld \\
\hline
Cardiovascular Biomechanics Lab & Ali Akyildiz \\
\hline
Musculoskeletal Modeling & Ajay Seth \& Eline Von Der Kruk \\
\hline
Joint interdisciplinary project & Birgit De Bruin \\
\hline
Medical imaging & Frans Vos \\
\hline
Devices for global Surgery and Forensics & Jenny Dankelman\& Roos Oosting \\
\hline
Clinical Biomechanics & Jaap Harlaar \& Mariska Wesseling \\
\hline
Soft Robotics in Healthcare & Aimee Sakes \\
\hline
NMC Lab & Winfred Mugge \& Mark Van De Ruit \\
\hline
Biomaterials 3D printing & Jie Zhou \& Mohammad Minzaal \\
\hline
Biomaterials \& Tissue Biomechanics & Amir Zadpoor \\
\end{tabular*}
\end{center}

这些宣讲会的录像和PPT,都可以去BMD的BrightSpace页面查看;其中,笔者将挑选几个有过谈话的项目组进行大概的阐述;但也警惕,Bob扬言这些课题组每年项目都会有所变化。笔者所诉说情形仅仅提供参考。
\begin{flushright}
王昊辰; 24/06/2023
\end{flushright}

\subsubsection{BITE}

\subsubsection{Devices for global Surgery and Forensics}

\subsubsection{Biomechatronic Design \& Biorobotics lab}




