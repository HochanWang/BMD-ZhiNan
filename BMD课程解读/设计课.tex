\vspace{\betsubsec} %section间留白
\section{ME-BMD Design Project Courses}
\subsection{ME41096 Bio Inspired Design}
\begin{minipage}{0.45\textwidth}
\centering
\begin{tikzpicture}
\begin{polaraxis}[
    width=0.65\textwidth,
    height=0.65\textwidth,
    yticklabels={},
    xtick={0,60,...,300},
    xticklabels={知识性, 授课质量, 作业友好,工作量适宜,易通过, 有趣},
    ymin=0,
    ymax=5,
    ytick={1,2,...,5},
    y grid style={gray},
    grid=both,
    minor grid style=gray,
    major grid style={gray},
]
\addplot[mark=*, ultra thick, black, data cs=polar] coordinates {
    (0,1)  % 知识性得分,这里修改为你的分数
    (60,3)  % 授课质量得分,这里修改为你的分数
    (120,5)  % 作业友好得分,这里修改为你的分数
    (180,5)  % 工作量得分,这里修改为你的分数
    (240,5)  % 易通过得分,这里修改为你的分数
    (300,3)  % 有趣得分,这里修改为你的分数
    (360,1)  % 为了闭合图形,最后的一个点应该和第一个点一样,这里修改为你的分数
};
\end{polaraxis}
\end{tikzpicture}
\end{minipage}%
\begin{minipage}{0.45\textwidth}
\raggedleft
\begin{tabular}{r|c}
\textbf{ } & \textbf{22-23} \\ \hline
通过率 & 100.0\% \\ 
平均分 & 7.65 \\ 
参与学生数 & 198 \\ 
学分 & 5 EC\\
\end{tabular}
\end{minipage}\\

先定个结论,属实良心课程。

这门课,顾名思义,就是仿生设计。话虽如此,实际操作中却更多是已经想到了某种解决方案,而后套到一个生物的机构上面。与其说是工程与设计,最后的产出,大多数组来看其实都是蛮科幻甚至抽象的;所以这方面完全没有太大必要担心,通过率100\%,已然说明问题;属实良心至极。

而设计的项目,会在课程开始的时候,从一系列项目里,让你填表选出自己想去的项目,想要一起合作的人。当然,若是像笔者一样的自闭儿,其实sign Up as an Individual也是完全没有问题的(甚至后来发现大多荷兰人都没有提前自己设组,全是个人身份填表)。而后就是持续2个Q的设计,一周隔一周的,会进行Presentation,汇报一下进度什么的;但Presentation,其实是完全不会纳入分数考量的;字面意思的,仅仅是给你一个集思广益获取同学反馈的渠道,所以也完全不用担心吧。

课程期间也会有Lecture,除了刚开始的几节的确有关设计作业有关课程。此后大多完全关于一些仿生设计知识,启迪式地,往往是Guest Lecture; 荷兰人极少出席,甚至有整个lecture hall个位数学生的尴尬场面。但笔者个人看看录像,其实觉得课程大多还蛮有价值的。

总之,这课就是一个设计作业,最后上交一篇小组报告。工作量很合理;Lecturer,也就是Paul为人也十分不错。我是十分推荐的。而且恶臭功利角度,顺带一提,Paul的研究组有很多有趣的毕设项目可选,且与一些公司和医院都有着很不错的关系。选这门课,和Paul混个脸熟,其实对于毕设与实习有微微帮助的。
\begin{flushright}
王昊辰; 02/06/2023
\end{flushright}

\subsection{ME46015 Precision Mechanism Design}
%待完成;张先治



\subsection{ME46115 Compliant Mechanisms}
\begin{minipage}{0.45\textwidth}
\centering
\begin{tikzpicture}
\begin{polaraxis}[
    width=0.65\textwidth,
    height=0.65\textwidth,
    yticklabels={},
    xtick={0,60,...,300},
    xticklabels={知识性, 授课质量, 作业友好,工作量适宜,易通过, 有趣},
    ymin=0,
    ymax=5,
    ytick={1,2,...,5},
    y grid style={gray},
    grid=both,
    minor grid style=gray,
    major grid style={gray},
]
\addplot[mark=*, ultra thick, black, data cs=polar] coordinates {
    (0,4)  % 知识性得分,这里修改为你的分数
    (60,5)  % 授课质量得分,这里修改为你的分数
    (120,3)  % 作业友好得分,这里修改为你的分数
    (180,2)  % 工作量得分,这里修改为你的分数
    (240,3)  % 易通过得分,这里修改为你的分数
    (300,4)  % 有趣得分,这里修改为你的分数
    (360,4)  % 为了闭合图形,最后的一个点应该和第一个点一样,这里修改为你的分数
};
\end{polaraxis}
\end{tikzpicture}
\end{minipage}%
\begin{minipage}{0.45\textwidth}
\raggedleft
\begin{tabular}{r|c}
\textbf{ } & \textbf{21-22} \\ \hline
通过率 & 99.1\%\\ 
平均分 & 7.74 \\ 
参与学生数 & 113 \\ 
学分 & 4 EC\\
\end{tabular}
\end{minipage}\\

与BID (Bio Inspired Design)一样的是,这门课最终的评分大致也取决于你的最终报告。所不同的是,这门课的期中,也就是Q1结束时(这课占2个Q),会有额外的一个考试;这个考试,就我所在的的小组而言,通过率并不十分理想。我们小组4人,其中也就是包括我在内2人通过,于此推演,甚至只有50\%.

关于授课内容,主要也就是顾名思义,柔性机构。会阐述给你一些Pseudo-Rigid-Body Model(PRBM)分析的理论。但是笔者经验算下来,要想应付评分大头的设计项目,其实这些教授的知识并不是十分足够,自己也是需要进行一些课程以外的拓展与阅读的。

这课,同BID一样的是,每两周也要进行presentation。但又不同于BID的科幻,这门课最终产出的结果是的确需要你进行打样与实验的。所以你真的就得把你设计的作品做出来,由此带来了更多的工作量。大概就是在圣诞前你们的设计和打样就得基本完成掉。

理论上,在这门课的进行中,是需要使用Ansys APDL进行FEM的。然而其实际操作,这些软件更多的是取决于小组个人的选择,例如我们小组,就事实上使用了Nastran。之前很多人问起我TUD究竟用什么CAD软件,什么FEM软件。其实这真的没有在实际操作中有太严厉的标准,更多的是只要小组一致同意,就是可以合理操作的。

结论,抛开性价比而言。这课我是觉得还蛮好的。但仅仅4学分,总让我觉得与工作量不甚相配。总的来说,选还是不选,还是希望按照个人兴趣决定罢。
\begin{flushright}
王昊辰; 02/06/2023
\end{flushright}

\subsection{ME41085 Biomechatromics}
\begin{minipage}{0.45\textwidth}
\centering
\begin{tikzpicture}
\begin{polaraxis}[
    width=0.65\textwidth,
    height=0.65\textwidth,
    yticklabels={},
    xtick={0,60,...,300},
    xticklabels={知识性, 授课质量, 作业友好,工作量适宜,易通过, 有趣},
    ymin=0,
    ymax=5,
    ytick={1,2,...,5},
    y grid style={gray},
    grid=both,
    minor grid style=gray,
    major grid style={gray},
]
\addplot[mark=*, ultra thick, black, data cs=polar] coordinates {
    (0,4)  % 知识性得分,这里修改为你的分数
    (60,5)  % 授课质量得分,这里修改为你的分数
    (120,3)  % 作业友好得分,这里修改为你的分数
    (180,2)  % 工作量得分,这里修改为你的分数
    (240,3)  % 易通过得分,这里修改为你的分数
    (300,4)  % 有趣得分,这里修改为你的分数
    (360,4)  % 为了闭合图形,最后的一个点应该和第一个点一样,这里修改为你的分数
};
\end{polaraxis}
\end{tikzpicture}
\end{minipage}%
\begin{minipage}{0.45\textwidth}
\raggedleft
\begin{tabular}{r|c}
\textbf{ } & \textbf{21-22} \\ \hline
通过率 & 100\% \\ 
平均分 & 7.59 \\ 
参与学生数 & 81 \\
学分 & 4 EC\\
\end{tabular}
\end{minipage}\\
