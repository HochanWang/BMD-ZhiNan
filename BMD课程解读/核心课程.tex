\subsection{ME Core Courses}
\subsubsection{ME41055 Multibody Dynamics B}
\begin{minipage}{0.45\textwidth}
\centering
\begin{tikzpicture}
\begin{polaraxis}[
    width=0.65\textwidth,
    height=0.65\textwidth,
    yticklabels={},
    xtick={0,60,...,300},
    xticklabels={知识性, 授课质量, 作业友好,工作量适宜,易通过, 有趣},
    ymin=0,
    ymax=5,
    ytick={1,2,...,5},
    y grid style={gray},
    grid=both,
    minor grid style=gray,
    major grid style={gray},
]
\addplot[mark=*, ultra thick, black, data cs=polar] coordinates {
    (0,5)  % 知识性得分,这里修改为你的分数
    (60,5)  % 授课质量得分,这里修改为你的分数
    (120,1)  % 作业友好得分,这里修改为你的分数
    (180,1)  % 工作量得分,这里修改为你的分数
    (240,3)  % 易通过得分,这里修改为你的分数
    (300,4)  % 有趣得分,这里修改为你的分数
    (360,5)  % 为了闭合图形,最后的一个点应该和第一个点一样,这里修改为你的分数
};
\end{polaraxis}
\end{tikzpicture}
\end{minipage}%
\begin{minipage}{0.45\textwidth}
\raggedleft
\begin{tabular}{r|c|c}
\textbf{指标} & \textbf{数据22-23} & \textbf{数据21-22}\\ \hline
通过率 & \% & 81.1\%\\ 
平均分 &   & 6.75\\ 
参与学生数 &  & 111\\
学分 & 4 EC & 4 EC\\
\end{tabular}
\end{minipage}\\

这门课的老师是美国人,2021-2022学年刚刚从UC Davis跳槽来TUD;感觉算是比较好说话的,教学质量也很不错。但是,这门课几乎可以说是整个第一年里,workload最大的一门课了;它每周都会有作业,Q3作业量是正常的,但到了Q4,很轻易地就可以每周在作业上花费十余小时。不过,今年已经通过AvL向学校进行了反馈,明年或许会有所改进吧。

其实公平讲,workload负担大也不全是这个老师的过错。这门课的打分机制,在2022-2023学年是这样的。考试占了最后总评分的60\%,平时作业40\%。但是,当你最终考试的成绩高于平时成绩的话,平时成绩就不会被考虑。这个老师辩解,如此设置是为了确保每个学生,就算你考试发挥不佳,最终成绩也不至于太难看。他扬言既然最终成绩既然总归需要及格,由此就算是不做作业也是没关系的。

的确承认,他讲的其实是有道理的,他说每周的作业并没有被设计地需要全部被完成,也无需全部完成。但作为学生,至少在Delft,我遇到的无论哪国人,每人似乎每周都会打磨作业到全部正确为止。或许是希望能由此尽可能地,在就算最终考试发会不佳的情况下分数也不至于太过于难看?

但这门课的内容,确实受益匪浅。它的内核理论并不深奥,其中大多数在本科阶段的Dynamics课其实多少会被提及;从牛顿欧拉,到凯恩方程,到拉格朗日和TMT法。就是一系列的解决运动问题的方法吧。但至少不同于我本科阶段的,完全用笔和纸解决比较异常简单的考试问题;这节课是完全基于编程计算,而一半以上的教学其实也是在教你如何操作。主要是用Python,探用Simpy,NumPy,SunDial等等包来计算复杂的动态;后期还会要求你把这些模拟以影像形式呈现。

这个老师似乎不是很喜欢用BrightSpace,教学内容的录像会被发布在Youtube,而所采用的教学材料,是他自己做的一个\href{https://moorepants.github.io/learn-multibody-dynamics/}{网站};以章节地形式叙述了我上述的全部内容。而这些内容,除开编程,基于一本1985年的已经绝版的经典教材,但相比那本经典教材,个人感觉他网站上的文字会更加通俗易懂,非常说人话。甚至笔者个人认为,再有一定基础的情况下,他的网站甚至算是运动学最好懂的教材之一了。就算不选这门课(虽然BMD而言是必修),也十分值得一读。
\begin{flushright}
王昊辰; 15/06/2023
\end{flushright}

\subsubsection{ME41106 Intelligent Vehicles 3ME}
\begin{minipage}{0.45\textwidth}
\centering
\begin{tikzpicture}
\begin{polaraxis}[
    width=0.65\textwidth,
    height=0.65\textwidth,
    yticklabels={},
    xtick={0,60,...,300},
    xticklabels={知识性, 授课质量, 作业友好,工作量适宜,易通过, 有趣},
    ymin=0,
    ymax=5,
    ytick={1,2,...,5},
    y grid style={gray},
    grid=both,
    minor grid style=gray,
    major grid style={gray},
]
\addplot[mark=*, ultra thick, black, data cs=polar] coordinates {
    (0,4)  % 知识性得分,这里修改为你的分数
    (60,5)  % 授课质量得分,这里修改为你的分数
    (120,3)  % 作业友好得分,这里修改为你的分数
    (180,1)  % 工作量得分,这里修改为你的分数
    (240,5)  % 易通过得分,这里修改为你的分数
    (300,4)  % 有趣得分,这里修改为你的分数
    (360,4)  % 为了闭合图形,最后的一个点应该和第一个点一样,这里修改为你的分数
};
\end{polaraxis}
\end{tikzpicture}
\end{minipage}%
\begin{minipage}{0.45\textwidth}
\raggedleft
\begin{tabular}{r|c}
\textbf{指标} & \textbf{数据} \\ \hline
通过率 & 90.7\% \\ 
平均分 & 7.57 \\ 
参与学生数 & 107 \\ 
学分 & 5 EC\\
\end{tabular}
\end{minipage}\\

该课主要讲述了自动驾驶的相关技术技术,包含了detection,localization 和 trajectory planning 3个方面,是一门对自动驾驶的很不错的入门课(同时也意味着课程的深度并不高)。该课的主讲人Dr.Gavrila 之前一直再奔驰做自动驾驶, 所以该课不管从有趣性和授课质量上都属于上乘。同时该课还是机器人学院的Machine Perception 的姊妹课,两门课的内容有2/3的重叠,所以不能同时选择。听一听 Machine perception 的guest Lecture也是蛮有意思的。

该课的作业也属于比较有意思的,可以再作业中实现一些detection以及localization的算法,并配有很不错的可视化,对于理解很有帮助。但是由于作业需要用线上的jupyter hub做,所以有的时候服务器,网络的问题也是真的恶心(这点在做大作业的时候体现的极其明显)。

由于这门课是机器人学院开的,所以通过率实属不错(机器人学院的课基本付出了时间认真做很难挂人),中国人普遍能得到8分左右(图示数据为Machine Perception的课程通过率和平均分, 但两者基本相近)。但是机器人学院的课普遍工作量很大很占时间(平时作业+大作业+考试),所以会导致考期前两周比较忙。

\begin{flushright}
马润禹; 03/06/2023
\end{flushright}

\subsubsection{ME44210 Drive \& Energy Systems}
\begin{minipage}{0.45\textwidth}
\centering
\begin{tikzpicture}
\begin{polaraxis}[
    width=0.65\textwidth,
    height=0.65\textwidth,
    yticklabels={},
    xtick={0,60,...,300},
    xticklabels={知识性, 授课质量, 作业友好,工作量适宜,易通过, 有趣},
    ymin=0,
    ymax=5,
    ytick={1,2,...,5},
    y grid style={gray},
    grid=both,
    minor grid style=gray,
    major grid style={gray},
]
\addplot[mark=*, ultra thick, black, data cs=polar] coordinates {
    (0,2)  % 知识性得分,这里修改为你的分数
    (60,4)  % 授课质量得分,这里修改为你的分数
    (120,5)  % 作业友好得分,这里修改为你的分数
    (180,5)  % 工作量得分,这里修改为你的分数
    (240,5)  % 易通过得分,这里修改为你的分数
    (300,3)  % 有趣得分,这里修改为你的分数
    (360,2)  % 为了闭合图形,最后的一个点应该和第一个点一样,这里修改为你的分数
};
\end{polaraxis}
\end{tikzpicture}
\end{minipage}%
\begin{minipage}{0.45\textwidth}
\raggedleft
\begin{tabular}{r|c}
\textbf{指标} & \textbf{数据} \\ \hline
通过率 & 80.0\% \\ 
平均分 & 7.07 \\ 
参与学生数 & 155 \\ 
学分 & 3 EC\\
\end{tabular}
\end{minipage}\\

这门课,是关于电机,变压器等的基础知识,会大概介绍各种常见电机,AC-DC转换,变压器等的原理和计算。若是在高中物理阶段学过选修3-1的话,会觉得学起来非常简单,但同时也就意味着所得到的知识不是很多吧。

它的考试非常容易通过,老师会在很早的时候就放出非常多得往年试卷,其超高的通过率和较高的平均分足以说明。

总结讲,一门非常好的用来水学分的课。

\begin{flushright}
王昊辰; 28/05/2023
\end{flushright}




\subsubsection{ME45042 Advanced Fluid Dynamics}
\begin{minipage}{0.45\textwidth}
\centering
\begin{tikzpicture}
\begin{polaraxis}[
    width=0.65\textwidth,
    height=0.65\textwidth,
    yticklabels={},
    xtick={0,60,...,300},
    xticklabels={知识性, 授课质量, 作业友好,工作量适宜,易通过, 有趣},
    ymin=0,
    ymax=5,
    ytick={1,2,...,5},
    y grid style={gray},
    grid=both,
    minor grid style=gray,
    major grid style={gray},
]
\addplot[mark=*, ultra thick, black, data cs=polar] coordinates {
    (0,4)  % 知识性得分,这里修改为你的分数
    (60,5)  % 授课质量得分,这里修改为你的分数
    (120,5)  % 作业友好得分,这里修改为你的分数
    (180,5)  % 工作量得分,这里修改为你的分数
    (240,2)  % 易通过得分,这里修改为你的分数
    (300,4)  % 有趣得分,这里修改为你的分数
    (360,4)  % 为了闭合图形,最后的一个点应该和第一个点一样,这里修改为你的分数
};
\end{polaraxis}
\end{tikzpicture}
\end{minipage}%
\begin{minipage}{0.45\textwidth}
\raggedleft
\begin{tabular}{r|c}
\textbf{指标} & \textbf{数据} \\ \hline
通过率 & 53.4\% \\ 
平均分 & 6.00 \\ 
参与学生数 & 88 \\ 
学分 & 5 EC\\

\end{tabular}
\end{minipage}\\

流体力学这门课,至少在2022年由Dr. Tam执教的时候,是很有趣也很有知识性的。但有传闻说2023年教纲会改掉,而执教的教师也会换掉。

就2022-2023学年来说的话;这门课的Q1部分主要就是在重新讲述一些基础知识,老生常谈的量纲分析,N-S方程推导之类;而Q2部分,就是关于invisid flow了,会学习像古早的工程师一样使用Joukowsky transform,complex field之类来计算机翼升力等等; 课程最后也会学习一下Boundary Layer, 提一嘴各种Wave与稳定性。

其授课质量,在我主观看来数一数二。这个老师来自MIT,用的也是MIT的教纲和资料;上课完全没有PPT,纯纯板书推导。美中不足,Dr. Tam的字迹比较难以辨认,可能需要一两周的适应。

关于作业,并不强制,无需上交也不参与评分。作业比较有难度,而且参考答案也不是很清晰。但我的体验是,就算如我一样完全不做作业对最终考试也不会有多少影响。

这门课可能是因为难度的原因,通过率和均分并不是很好看,由此虽然是提供给整个ME学院可选的核心课程,在荷兰人中却并不火爆,许多人会知难而退,这点从相对较少的参与学生数中也可以看出。但我个人觉得,要是对流体力学稍微有点兴趣,都还是很推荐的。

\begin{flushright}
王昊辰; 28/05/2023
\end{flushright}



\subsubsection{ME46006 Physics for Mechanical Engineers}
\begin{minipage}{0.45\textwidth}
\centering
\begin{tikzpicture}
\begin{polaraxis}[
    width=0.65\textwidth,
    height=0.65\textwidth,
    yticklabels={},
    xtick={0,60,...,300},
    xticklabels={知识性, 授课质量, 作业友好,工作量适宜,易通过, 有趣},
    ymin=0,
    ymax=5,
    ytick={1,2,...,5},
    y grid style={gray},
    grid=both,
    minor grid style=gray,
    major grid style={gray},
]
\addplot[mark=*, ultra thick, black, data cs=polar] coordinates {
    (0,2)  % 知识性得分,这里修改为你的分数
    (60,4)  % 授课质量得分,这里修改为你的分数
    (120,5)  % 作业友好得分,这里修改为你的分数
    (180,5)  % 工作量得分,这里修改为你的分数
    (240,4)  % 易通过得分,这里修改为你的分数
    (300,3)  % 有趣得分,这里修改为你的分数
    (360,2)  % 为了闭合图形,最后的一个点应该和第一个点一样,这里修改为你的分数
};
\end{polaraxis}
\end{tikzpicture}
\end{minipage}%
\begin{minipage}{0.45\textwidth}
\raggedleft
\begin{tabular}{r|c|c}
\textbf{指标} & \textbf{数据} & \textbf{数据21-22}\\ \hline
通过率 & 64.0\% & 69.7\%\\ 
平均分 & 6.54 & 6.59\\ 
参与学生数 & 197 & 198\\ 
学分 & 4 EC & 4 EC\\
\end{tabular}
\end{minipage}\\

物理,基本就是高中物理选修部分的延申。因为我自己是没有上过国内本科,不太清楚国内的大学物理深度几何。但我幻想,由于tud的荷兰人们本科阶段,是没有继续在高中物理上精进的,所以这门课的内容应与国内大学物理多有重合。若是国内本科,学完大学物理以后,这门课想必会觉得非常轻松吧。

其教学内容,主要就是一些电磁学,非常基础的量子力学,以及不超过高中物理3-4难度的光学。总的来说并不是一门很难的课。通过率数据看起来不怎么漂亮的原因,大概可能还是教学速度吧。两周四节课,自欧姆定律,猪突讲完麦克斯韦的程度。

这门课没有要交的作业,但每周会有推荐完成的练习。而这些练习里面,基于2022-2023学年的试卷来看,会有1,2道原题出现在考试。不过考试难度并不是很高,那些原题,个人实践就算没有做过,考试也是可以完成的。主观总结讲,这是一门放眼整个第一年,都可以认定为非常容易水学分的课程。
\begin{flushright}
王昊辰; 28/05/2023
\end{flushright}


\subsubsection{ME46007 Measurement Technology}
\begin{minipage}{0.45\textwidth}
\centering
\begin{tikzpicture}
\begin{polaraxis}[
    width=0.65\textwidth,
    height=0.65\textwidth,
    yticklabels={},
    xtick={0,60,...,300},
    xticklabels={知识性, 授课质量, 作业友好,工作量适宜,易通过, 有趣},
    ymin=0,
    ymax=5,
    ytick={1,2,...,5},
    y grid style={gray},
    grid=both,
    minor grid style=gray,
    major grid style={gray},
]
\addplot[mark=*, ultra thick, black, data cs=polar] coordinates {
    (0,4)  % 知识性得分,这里修改为你的分数
    (60,3)  % 授课质量得分,这里修改为你的分数
    (120,5)  % 作业友好得分,这里修改为你的分数
    (180,4)  % 工作量得分,这里修改为你的分数
    (240,2)  % 易通过得分,这里修改为你的分数
    (300,3)  % 有趣得分,这里修改为你的分数
    (360,4)  % 为了闭合图形,最后的一个点应该和第一个点一样,这里修改为你的分数
};
\end{polaraxis}
\end{tikzpicture}
\end{minipage}%
\begin{minipage}{0.45\textwidth}
\raggedleft
\begin{tabular}{r|c|c}
\textbf{指标} & \textbf{数据} & \textbf{数据21-22} \\ \hline
通过率 & 59.3\% & 66\%\\ 
平均分 & 6.19 & 6.33\\ 
参与学生数 & 241 &186\\ 
学分 & 3 EC &\\
\end{tabular}
\end{minipage}\\

\subsubsection{ME46055 Engineering Dynamics}
%待完成;张先治
\begin{minipage}{0.45\textwidth}
\centering
\begin{tikzpicture}
\begin{polaraxis}[
    width=0.65\textwidth,
    height=0.65\textwidth,
    yticklabels={},
    xtick={0,60,...,300},
    xticklabels={知识性, 授课质量, 作业友好,工作量适宜,易通过, 有趣},
    ymin=0,
    ymax=5,
    ytick={1,2,...,5},
    y grid style={gray},
    grid=both,
    minor grid style=gray,
    major grid style={gray},
]
\addplot[mark=*, ultra thick, black, data cs=polar] coordinates {
    (0,4)  % 知识性得分,这里修改为你的分数
    (60,4)  % 授课质量得分,这里修改为你的分数
    (120,4)  % 作业友好得分,这里修改为你的分数
    (180,4)  % 工作量得分,这里修改为你的分数
    (240,4)  % 易通过得分,这里修改为你的分数
    (300,3)  % 有趣得分,这里修改为你的分数
    (360,4)  % 为了闭合图形,最后的一个点应该和第一个点一样,这里修改为你的分数
};
\end{polaraxis}
\end{tikzpicture}
\end{minipage}%
\begin{minipage}{0.45\textwidth}
\raggedleft
\begin{tabular}{r|c}
\textbf{指标} & \textbf{数据} \\ \hline
通过率 & 78\% \\ 
平均分 & 7.04\\ 
参与学生数 & 200 \\ 
学分 & 4 EC\\
\end{tabular}
\end{minipage}\\

\subsection{ME46000 Nonlinear Mechanics}
\begin{minipage}{\textwidth}
\centering
\begin{tabularx}{\textwidth}{l|X|X|X|X}
\textbf{ } &\textbf{通过率} & \textbf{平均分} & \textbf{参与学生数} & \textbf{学分} \\ \hline
数据21-22 & 52.2\% & 5.95 & 136 & 4 EC \\
\end{tabularx}
\end{minipage}



\subsubsection{ME46085 Mechatronic System Design}
\begin{minipage}{0.45\textwidth}
\centering
\begin{tikzpicture}
\begin{polaraxis}[
    width=0.65\textwidth,
    height=0.65\textwidth,
    yticklabels={},
    xtick={0,60,...,300},
    xticklabels={知识性, 授课质量, 作业友好,工作量适宜,易通过, 有趣},
    ymin=0,
    ymax=5,
    ytick={1,2,...,5},
    y grid style={gray},
    grid=both,
    minor grid style=gray,
    major grid style={gray},
]
\addplot[mark=*, ultra thick, black, data cs=polar] coordinates {
    (0,2)  % 知识性得分,这里修改为你的分数
    (60,4)  % 授课质量得分,这里修改为你的分数
    (120,3)  % 作业友好得分,这里修改为你的分数
    (180,3)  % 工作量得分,这里修改为你的分数
    (240,4)  % 易通过得分,这里修改为你的分数
    (300,3)  % 有趣得分,这里修改为你的分数
    (360,2)  % 为了闭合图形,最后的一个点应该和第一个点一样,这里修改为你的分数
};
\end{polaraxis}
\end{tikzpicture}
\end{minipage}%
\begin{minipage}{0.45\textwidth}
\raggedleft
\begin{tabular}{r|c}
\textbf{指标} & \textbf{数据} \\ \hline
通过率 & 85.4\% \\ 
平均分 & 7.18\\ 
参与学生数 & 164 \\ 
学分 & 4 EC\\
\end{tabular}
\end{minipage}\\

比较推荐的一节课,主要内容为运动学方程,共振频率,PID控制,电磁驱动器,洛伦兹驱动器以及压电驱动器。
成绩的40\%由是三次作业。第一是。
成绩的60\%是期末考试,试卷上至少得到35/60并且再与作业加权平均之后大于6才能通过课程。开卷考试,只允许携带指定参考书,参考书在leegwater有卖,58欧元。需要注意的是,允许在书上做笔记
\subsubsection{SC42001 Control System Design}
\begin{minipage}{0.45\textwidth}
\centering
\begin{tikzpicture}
\begin{polaraxis}[
    width=0.65\textwidth,
    height=0.65\textwidth,
    yticklabels={},
    xtick={0,60,...,300},
    xticklabels={知识性, 授课质量, 作业友好,工作量适宜,易通过, 有趣},
    ymin=0,
    ymax=5,
    ytick={1,2,...,5},
    y grid style={gray},
    grid=both,
    minor grid style=gray,
    major grid style={gray},
]
\addplot[mark=*, ultra thick, black, data cs=polar] coordinates {
    (0,4)  % 知识性得分,这里修改为你的分数
    (60,3)  % 授课质量得分,这里修改为你的分数
    (120,5)  % 作业友好得分,这里修改为你的分数
    (180,3)  % 工作量得分,这里修改为你的分数
    (240,3)  % 易通过得分,这里修改为你的分数
    (300,2)  % 有趣得分,这里修改为你的分数
    (360,4)  % 为了闭合图形,最后的一个点应该和第一个点一样,这里修改为你的分数
};
\end{polaraxis}
\end{tikzpicture}
\end{minipage}%
\begin{minipage}{0.45\textwidth}
\raggedleft
\begin{tabular}{r|c}
\textbf{指标} & \textbf{数据} \\ \hline
通过率 & 66.8\% \\ 
平均分 & 6.67 \\ 
参与学生数 & 151 \\ 
学分 & 5 EC\\
\end{tabular}
\end{minipage}\\

几乎是所有 ME 和 EE 的中国学生都会选择的课。这门课基本对于控制系统进行了概述 从最开始的 state space 稳定性,可控性, 可观测性分析到卡尔曼滤波再到模型预测控制都有涉猎,是一门很不错的控制学科的入门课。就工作量而言,这门课程没有必须交的作业,每两周会有习题并会设有专门的习题课。从个人角度而言,由于是一门纯理论课,所以很难说这门课很有趣。

就考试而言,最优策略是刷近几年的考题。考试难度并不高,但是由于全是选择题且需要答对20题中的14题才能通过,所以极易翻车,需要对计算过程谨慎一些。
\begin{flushright}
马润禹; 03/06/2023
\end{flushright}

