\vspace{\betsubsec} %section间留白
\section{ME-BMD Electives}
\subsection{BM41155 3D Printing}\hypertarget{BM41155}{} 
\begin{minipage}{0.45\textwidth}
\centering
\begin{tikzpicture}
\begin{polaraxis}[
    width=0.65\textwidth,
    height=0.65\textwidth,
    yticklabels={},
    xtick={0,60,...,300},
    xticklabels={知识性, 授课质量, 作业友好,工作量适宜,易通过, 有趣},
    ymin=0,
    ymax=5,
    ytick={1,2,...,5},
    y grid style={gray},
    grid=both,
    minor grid style=gray,
    major grid style={gray},
]
\addplot[mark=*, ultra thick, black, data cs=polar] coordinates {
    (0,3)  % 知识性得分,这里修改为你的分数
    (60,3)  % 授课质量得分,这里修改为你的分数
    (120,5)  % 作业友好得分,这里修改为你的分数
    (180,5)  % 工作量得分,这里修改为你的分数
    (240,5)  % 易通过得分,这里修改为你的分数
    (300,4)  % 有趣得分,这里修改为你的分数
    (360,3)  % 为了闭合图形,最后的一个点应该和第一个点一样,这里修改为你的分数
};
\end{polaraxis}
\end{tikzpicture}
\end{minipage}%
\begin{minipage}{0.45\textwidth}
\raggedleft
\begin{tabular}{r|c}
\textbf{ } & \textbf{22-23} \\ \hline
通过率 & 88.6\% \\ 
平均分 & 6.97 \\ 
参与学生数 & 166 \\
学分 & 4 EC\\
\end{tabular}
\end{minipage}\\

强烈推荐,最容易拿的4学分。主要内容为3D打印的流程以及不同3D打印方法。

这节课的成绩完全由期末考试决定,考试内容主要为课程的PPT,完全就是考记忆力不需要任何理解和计算,把往年考试题背住就能拿不错的分。笔者上课一直摸鱼,考前突击背了3天就得了8.5。复习资料主要是Studocu上的总结以及往年考试题。

有一个小的小组作业,8个人一组设计一个梁的结构并3D打印出来,进行强度测试,这个作业不占分。每节课都需要签到,但并不清楚出勤率是否与成绩有某种隐形的联系。
\begin{flushright}
张先治; 09/06/2023
\end{flushright}

\subsection{ME41035 Special Topics in Sports Engineering}\hypertarget{ME41035}{} 
% 待完成;张先治 q5的课7.17前写完


\subsection{ME41065 System Identification and Parameter Estimation}\hypertarget{ME41065}{} 
\begin{minipage}{0.45\textwidth}
\centering
\begin{tikzpicture}
\begin{polaraxis}[
    width=0.65\textwidth,
    height=0.65\textwidth,
    yticklabels={},
    xtick={0,60,...,300},
    xticklabels={知识性, 授课质量, 作业友好,工作量适宜,易通过, 有趣},
    ymin=0,
    ymax=5,
    ytick={1,2,...,5},
    y grid style={gray},
    grid=both,
    minor grid style=gray,
    major grid style={gray},
]
\addplot[mark=*, ultra thick, black, data cs=polar] coordinates {
    (0,4)  % 知识性得分,这里修改为你的分数
    (60,2)  % 授课质量得分,这里修改为你的分数
    (120,3)  % 作业友好得分,这里修改为你的分数
    (180,3)  % 工作量得分,这里修改为你的分数
    (240,5)  % 易通过得分,这里修改为你的分数
    (300,4)  % 有趣得分,这里修改为你的分数
    (360,4)  % 为了闭合图形,最后的一个点应该和第一个点一样,这里修改为你的分数
};
\end{polaraxis}
\end{tikzpicture}
\end{minipage}%
\begin{minipage}{0.45\textwidth}
\raggedleft
\begin{tabular}{r|c|c}
\textbf{ } & \textbf{22-23} & \textbf{21-22}\\ \hline
通过率 & 87.5\% & 91.7\% \\ 
平均分 & 6.94 & 6.65\\ 
参与学生数 & 48 &44\\
学分 & 7 EC&\\
\end{tabular}
\end{minipage}\\

系统识别与参数估计是一个两学期的课程,主要是信号系统以及控制的一些知识,内容比较基础。

一共12节课,每上三节课需要完成一次作业,使用MATLAB完成,作业不算分,但是必须要全部通过才能录入成绩,没有严格的Deadline,在Deadline之前提交可以得到反馈。

成绩的40\%由4次Digitial Test 决定,每上三节课都进行,形式是闭卷选择题,内容为概念以及作业每考完两次Digitial Test可以报名查看正确答案,但是不能拍照或做笔记。在期末考试之前可以进行Digitial Test的补考,补考的最高分数为6。

成绩的60\%是期末考试,笔答闭卷考试,考试内容更偏向于概念理解,可以参考的往年考试题只有14,16的两张卷,\href{https://www.studeersnel.nl/nl/home}{\uline{Studocu}}上有一些比较好的总结可以用来复习。

总的来说是一门不错的课程比较推荐,能对相关领域的知识有大概的理解,相应的知识在Mechatronic System Design, Neuromehchanics \& Motor Control等课程中都有体现。作业难度适中,考试难度不大,上课有点无聊,老师基本是在念PPT,最重要的是它有7学分还是比较值的。
\begin{flushright}
张先治; 07/06/2023
\end{flushright}

\subsection{ME41120 Freehand Sketching of Products and Mechanisms}\hypertarget{ME41120}{} 
\begin{minipage}{0.45\textwidth}
\centering
\begin{tikzpicture}
\begin{polaraxis}[
    width=0.65\textwidth,
    height=0.65\textwidth,
    yticklabels={},
    xtick={0,60,...,300},
    xticklabels={知识性, 授课质量, 作业友好,工作量适宜,易通过, 有趣},
    ymin=0,
    ymax=5,
    ytick={1,2,...,5},
    y grid style={gray},
    grid=both,
    minor grid style=gray,
    major grid style={gray},
]
\addplot[mark=*, ultra thick, black, data cs=polar] coordinates {
    (0,5)  % 知识性得分,这里修改为你的分数
    (60,5)  % 授课质量得分,这里修改为你的分数
    (120,5)  % 作业友好得分,这里修改为你的分数
    (180,4)  % 工作量得分,这里修改为你的分数
    (240,5)  % 易通过得分,这里修改为你的分数
    (300,5)  % 有趣得分,这里修改为你的分数
    (360,5)  % 为了闭合图形,最后的一个点应该和第一个点一样,这里修改为你的分数
};
\end{polaraxis}
\end{tikzpicture}
\end{minipage}%
\begin{minipage}{0.45\textwidth}
\raggedleft
\begin{tabular}{r|c}
\textbf{ } & \textbf{22-23} \\ \hline
通过率 & 100\% \\ 
平均分 & 7.71 \\ 
参与学生数 & 40 \\
学分 & 3 EC\\
\end{tabular}
\end{minipage}\\

真的是非常非常好的一门课,说是第一年最有价值的一门课也不为过!

小班化教学,采用邮件预约制;因为十分火爆,想要上这门课需要在Q1刚开始时候就发邮件,诉说自己的动机进行预约。Q3真正上的时候,40人会分为2个班级;每个班级20人,配备1位Lecturer和一位助教,属实小班化教学了。

这门课没有考试,评分就是每周作业和你上课时候的笔记。每周作业一般就是画一个物体,笔者经验,不完美主义精雕细琢,一周至多2,3小时。这些作业文档,都会在课程结束时候要求你扫描上传;笔者就是因为平时上课没有做笔记,最后成绩并不好看。但就算如此,也是所有参与的人都通过了! 通过率100\%毕竟。

至于教学内容;主要就是教授你画画,一些透视技巧,怎样打上阴影。这课要求你必须是此前毫无画画基础的。笔者上这门课之前,画图纯纯火柴人水平;这课以后,能够画出三维的,甚至有点黑白照片感觉的物体了(我的最终作业画了一块手表)。课程结束时候还会像模地像样的举办画展,给参加课程并通过的学生全部发上颇有仪式氛围的结课证书;感觉十分不错的。

如果有意向想要选这门课,高兴的话,其实可以在国内就买好一些必要的器材:Copic 的N3,N5,N8都是必需品(N3和N5可以买2支或者买补充墨水,一支可能不够用);以及A3的马克笔绘画纸和针管笔(推荐买支0.1的买支0.4的)。在荷兰买这些非常之贵,比如马克笔纸,平均一张就要0.35欧。
\begin{flushright}
王昊辰; 02/06/2023
\end{flushright}

\subsection{RO47005 Planing and Decision Making}\hypertarget{RO47005}{} 
\begin{minipage}{0.45\textwidth}
\centering
\begin{tikzpicture}
\begin{polaraxis}[
    width=0.65\textwidth,
    height=0.65\textwidth,
    yticklabels={},
    xtick={0,60,...,300},
    xticklabels={知识性, 授课质量, 作业友好,工作量适宜,易通过, 有趣},
    ymin=0,
    ymax=5,
    ytick={1,2,...,5},
    y grid style={gray},
    grid=both,
    minor grid style=gray,
    major grid style={gray},
]
\addplot[mark=*, ultra thick, black, data cs=polar] coordinates {
    (0,4)  % 知识性得分,这里修改为你的分数
    (60,4)  % 授课质量得分,这里修改为你的分数
    (120,4)  % 作业友好得分,这里修改为你的分数
    (180,1)  % 工作量得分,这里修改为你的分数
    (240,5)  % 易通过得分,这里修改为你的分数
    (300,4)  % 有趣得分,这里修改为你的分数
    (360,4)  % 为了闭合图形,最后的一个点应该和第一个点一样,这里修改为你的分数
};
\end{polaraxis}
\end{tikzpicture}
\end{minipage}%
\begin{minipage}{0.45\textwidth}
\raggedleft
\begin{tabular}{r|c}
\textbf{ } & \textbf{22-23} \\ \hline
通过率 & 89\% \\ 
平均分 & 7.66 \\ 
参与学生数 & 89 \\ 
学分 & 5 EC\\
\end{tabular}
\end{minipage}\\

该课主要概述了机器人的运动规划,就本人而言感觉这是一门对该领域很好的入门课,对于想研究(BIO)Robotics的同学是一门不错的课。这门课的授课质量,知识型,有趣性,作业布置都属上乘,美中不足的是课程再前期讲的很快,对机器人领域不了解的小伙伴听起来会很吃力,而且老师的西班牙口音稍微有点重,所以我在最开始几周基本没听懂。

和机器人学院的其他五分必修课一样,这门课也是作业+大作业+考试的配置,所以课业压力会很大。大作业需要自己和组员完成一款规划算法的设计与分析,认真完成的话还是很有意义。

\begin{flushright}
马润禹; 03/06/2023
\end{flushright}

\subsection{RO47003 Robot Software Practicals}\hypertarget{RO47003}{} 
\begin{minipage}{0.45\textwidth}
\centering
\begin{tikzpicture}
\begin{polaraxis}[
    width=0.65\textwidth,
    height=0.65\textwidth,
    yticklabels={},
    xtick={0,60,...,300},
    xticklabels={知识性, 授课质量, 作业友好,工作量适宜,易通过, 有趣},
    ymin=0,
    ymax=5,
    ytick={1,2,...,5},
    y grid style={gray},
    grid=both,
    minor grid style=gray,
    major grid style={gray},
]
\addplot[mark=*, ultra thick, black, data cs=polar] coordinates {
    (0,5)  % 知识性得分,这里修改为你的分数
    (60,5)  % 授课质量得分,这里修改为你的分数
    (120,5)  % 作业友好得分,这里修改为你的分数
    (180,4)  % 工作量得分,这里修改为你的分数
    (240,5)  % 易通过得分,这里修改为你的分数
    (300,5)  % 有趣得分,这里修改为你的分数
    (360,5)  % 为了闭合图形,最后的一个点应该和第一个点一样,这里修改为你的分数
};
\end{polaraxis}
\end{tikzpicture}
\end{minipage}%
\begin{minipage}{0.45\textwidth}
\raggedleft
\begin{tabular}{r|c}
\textbf{ } & \textbf{22-23} \\ \hline
通过率 & 无数据 \\ 
平均分 & 无数据 \\ 
参与学生数 & 无数据 \\
学分 & 5 EC\\
\end{tabular}
\end{minipage}\\

这门课个人感觉很有用,内容覆盖了Linux,git,C++,和ROS的基础知识,是本科没有相关基础的同学的不二选择!

linux主题会涉及一些基本的操作系统概念与文件管理的方法,git主题会对版本管理,分支合并等等概念进行介绍。c++会涉及一些基础的语法以及最重要的编译概念。随后的ROS会涉及基本概念并在后期大致介绍常用的ROS C++库如pcl。

这门课的作业给我的体验很好。需要完成三次不计分的平时小组作业,和一次独立且计分的小项目。小组作业会在上完Linux和git后开始,形式为两人利用git进行合作,往gitlab上提交代码。每次作业都会有一个很长的详细的文档介绍,基本上认真读完作业就没有压力。平时作业虽不评分但需要在Gitlab上对其他组的代码进行点评。可以说这样的作业形式让我充分实践了课堂所学。最后的独立小项目我当时是要用pcl库编写一个简单的ROS程序,老师会提供很多的模块帮助你看到很有趣的最终效果。

最后这门课会有一次考试,内容为简单的概念问答。
\begin{flushright}
孙天辰; 04/06/2023
\end{flushright}


\subsection{WI4771TU Object Oriented Scientific Programming C++}\hypertarget{WI4771TU}{} 
\begin{minipage}{0.45\textwidth}
\centering
\begin{tikzpicture}
\begin{polaraxis}[
    width=0.65\textwidth,
    height=0.65\textwidth,
    yticklabels={},
    xtick={0,60,...,300},
    xticklabels={知识性, 授课质量, 作业友好,工作量适宜,易通过, 有趣},
    ymin=0,
    ymax=5,
    ytick={1,2,...,5},
    y grid style={gray},
    grid=both,
    minor grid style=gray,
    major grid style={gray},
]
\addplot[mark=*, ultra thick, black, data cs=polar] coordinates {
    (0,4)  % 知识性得分,这里修改为你的分数
    (60,4)  % 授课质量得分,这里修改为你的分数
    (120,3)  % 作业友好得分,这里修改为你的分数
    (180,4)  % 工作量得分,这里修改为你的分数
    (240,5)  % 易通过得分,这里修改为你的分数
    (300,3)  % 有趣得分,这里修改为你的分数
    (360,4)  % 为了闭合图形,最后的一个点应该和第一个点一样,这里修改为你的分数
};
\end{polaraxis}
\end{tikzpicture}
\end{minipage}%
\begin{minipage}{0.45\textwidth}
\raggedleft
\begin{tabular}{r|c}
\textbf{ } & \textbf{22-23} \\ \hline
通过率 & 100\% \\ 
平均分 & 8.15 \\ 
参与学生数 & 198 \\
学分 & 3 EC\\
\end{tabular}
\end{minipage}\\

主要就是教会你C++这门语言吧。其实这课没什么好说的;教学质量一般般,老师一般般,每周作业难度也是一般般,完全一股一般般的腔调。当然这里的一般般都是TUD标准下的一般般了,每周作业,笔者经验,大概3小时左右可以完成吧。

这课程没有考试,最终评分主要依照一个大作业。在我们这届,也就是2022-2023学年,是一个Passive Walker的计算。需要你自己写出2种积分求解器。这是很难的。但由于不是考试,所以可以,至少私下里,到处点头哈腰交流答案;或者之前国内的同学询问之类。总之通过不是什么问题,这也在其100\%的通过率和较高的平均分上可以体现。

总结讲,我觉得这课还是蛮值得的。无论学分的性价比还是知识性而言。
\begin{flushright}
王昊辰; 02/06/2023
\end{flushright}


\subsection[ME46060 Engineering Optimisation: Concepts and Application]{ME46060 Engineering Optimisation: Concepts and Application}\hypertarget{ME46060}{} 

\begin{minipage}{0.45\textwidth}
\centering
\begin{tikzpicture}
\begin{polaraxis}[
    width=0.65\textwidth,
    height=0.65\textwidth,
    yticklabels={},
    xtick={0,60,...,300},
    xticklabels={知识性, 授课质量, 作业友好,工作量适宜,易通过, 有趣},
    ymin=0,
    ymax=5,
    ytick={1,2,...,5},
    y grid style={gray},
    grid=both,
    minor grid style=gray,
    major grid style={gray},
]
\addplot[mark=*, ultra thick, black, data cs=polar] coordinates {
    (0,3)  % 知识性得分,这里修改为你的分数
    (60,3)  % 授课质量得分,这里修改为你的分数
    (120,4)  % 作业友好得分,这里修改为你的分数
    (180,4)  % 工作量得分,这里修改为你的分数
    (240,4)  % 易通过得分,这里修改为你的分数
    (300,3)  % 有趣得分,这里修改为你的分数
    (360,3)  % 为了闭合图形,最后的一个点应该和第一个点一样,这里修改为你的分数
};
\end{polaraxis}
\end{tikzpicture}
\end{minipage}%
\begin{minipage}{0.45\textwidth}
\raggedleft
\begin{tabular}{r|c|c}
\textbf{指标} & \textbf{22-23}& \textbf{21-22} \\ \hline
通过率 &\% & 96.2\% \\ 
平均分 & & 7.82 \\ 
参与学生数&  & 131 \\
学分 & & 3 EC\\
\end{tabular}
\end{minipage}\\

这门课的主要内容为约束/非约束的条件下进行优化,单变量/多变量的优化算法,有限元分析的敏感度分析以及拓扑优化。虽然上课内容中的公式比较难懂,但是MATLAB把这些算法都集成了,实际使用的时候直接用就行。

一共有12次作业,问题的数学模型已经提供,基本上是让你熟悉使用MATLAB中的优化工具箱。三次线上考试比较难,单纯是用来拉开分数的。

这门课的成绩主要由报告决定,作业以及三次线上考试不直接记入成绩,在报告及格的前提下,作为报告分的加成,所以理论上只交报告也能通过这门课。报告可以单人完成也可以两人组队,需要你选择一样东西进行优化设计,选择什么看你擅长什么,可以是压力容器,齿轮组,水瓶,钳子,螺纹。主要是要整理出数学模型,剩下的基本上就是走一遍作业中的流程。

Q4的时间比较紧张,建议尽早开始写报告,如果你的课题难度适中,那么不需要等到拓扑优化的课上完就可以开始了。
\begin{flushright}
张先治;07/07/2023
\end{flushright}
\subsection[BM41050 Applied Experimental Methods: Medical Instruments]{BM41050 Applied Experimental Methods: Medical Instruments}\hypertarget{BM41050}{} 
%祁晨晨
\begin{minipage}{0.45\textwidth}
\centering
\begin{tikzpicture}
\begin{polaraxis}[
    width=0.65\textwidth,
    height=0.65\textwidth,
    yticklabels={},
    xtick={0,60,...,300},
    xticklabels={知识性, 授课质量, 作业友好,工作量适宜,易通过, 有趣},
    ymin=0,
    ymax=5,
    ytick={1,2,...,5},
    y grid style={gray},
    grid=both,
    minor grid style=gray,
    major grid style={gray},
]
\addplot[mark=*, ultra thick, black, data cs=polar] coordinates {
    (0,4)  % 知识性得分,这里修改为你的分数
    (60,4)  % 授课质量得分,这里修改为你的分数
    (120,5)  % 作业友好得分,这里修改为你的分数
    (180,5)  % 工作量得分,这里修改为你的分数
    (240,5)  % 易通过得分,这里修改为你的分数
    (300,4)  % 有趣得分,这里修改为你的分数
    (360,4)  % 为了闭合图形,最后的一个点应该和第一个点一样,这里修改为你的分数
};
\end{polaraxis}
\end{tikzpicture}
\end{minipage}%
\begin{minipage}{0.45\textwidth}
\raggedleft
\begin{tabular}{r|c}
\textbf{ } & \textbf{22-23} \\ \hline
通过率 & 无数据 \\ 
平均分 & 无数据 \\ 
参与学生数 & 无数据 \\
学分 & 4 EC\\
\end{tabular}
\end{minipage}\\

这门课是个大水课,正课只有三次,之后便是自行做实验(在家实验,约耗时3h)与四页的实验报告。这就是这门课的全部内容。任课教师原话:“之前好多同学都反映这门课4个EC太多了,它只值2个EC。但是我觉得既然我们的课程已经如此紧张了,为何不用这门课来平衡一下呢?”

通过这门课,可以了解整个科学实验从实验设计到论文写作的全过程,对今后的学习研究有很大帮助。
\begin{flushright}
祁晨晨; 07/06/2023
\end{flushright}

\subsection{BM41130 Tissue Biomechanics}\hypertarget{BM41130}{} 
%祁晨晨
\begin{minipage}{0.45\textwidth}
\centering
\begin{tikzpicture}
\begin{polaraxis}[
    width=0.65\textwidth,
    height=0.65\textwidth,
    yticklabels={},
    xtick={0,60,...,300},
    xticklabels={知识性, 授课质量, 作业友好,工作量适宜,易通过, 有趣},
    ymin=0,
    ymax=5,
    ytick={1,2,...,5},
    y grid style={gray},
    grid=both,
    minor grid style=gray,
    major grid style={gray},
]
\addplot[mark=*, ultra thick, black, data cs=polar] coordinates {
    (0,5)  % 知识性得分,这里修改为你的分数
    (60,4)  % 授课质量得分,这里修改为你的分数
    (120,3)  % 作业友好得分,这里修改为你的分数
    (180,2)  % 工作量得分,这里修改为你的分数
    (240,1)  % 易通过得分,这里修改为你的分数
    (300,4)  % 有趣得分,这里修改为你的分数
    (360,5)  % 为了闭合图形,最后的一个点应该和第一个点一样,这里修改为你的分数
};
\end{polaraxis}
\end{tikzpicture}
\end{minipage}%
\begin{minipage}{0.45\textwidth}
\raggedleft
\begin{tabular}{r|c}
\textbf{ } & \textbf{22-23} \\ \hline
通过率 & 70.1\% \\ 
平均分 & 6.45 \\ 
参与学生数 & 97 \\
学分 & 3 EC\\
\end{tabular}
\end{minipage}\\

十分硬核的一门课,其既设计生理知识,又涉及计算。简而言之,不论你本科背景偏生物还是偏工程,这门课总有一部分会让你很难上手,甚至含泪挂科,在第一个学期给你来自TUD的第一记暴击。第一次考试之后仅有60\%的通过率,经过不懈的补考,又有10\%的朋友挣扎上岸,即使上岸,成绩也惨不忍睹,堪称TUD高压适应性课程。

抛开成绩不谈,这门课的知识性毋庸置疑,为BME/BMD的同学提供了有关组织与组织模型的详尽的生理与工程知识,还是很值得一学的(反正大家都过不了,相当于大家都过了)。

课程内容包括信息量极大的讲座若干,两次不用提交但不写期末傻眼的练习,以及两次讲解练习的习题课。由于本课开课仅两年且因为成绩爆炸导致评教哀鸿遍野,任课教授正不断调整课程设置,具体课程内容可能有较大变化。
\begin{flushright}
祁晨晨; 07/06/2023
\end{flushright}