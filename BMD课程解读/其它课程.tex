\vspace{\betsubsec} %section间留白
\section{Alternative Electives}

至少于Bob 治下的BMD而言,除开Form2上明确列出的课程,在简单阐述理由的情况下,在完成必须的学分后,选择表格以外甚至其它学院的选修课,都是可行的选项。

\subsection{EPA1223 Macro-economics for Policy Analysis}
\begin{minipage}{0.45\textwidth}
\centering
\begin{tikzpicture}
\begin{polaraxis}[
    width=0.65\textwidth,
    height=0.65\textwidth,
    yticklabels={},
    xtick={0,60,...,300},
    xticklabels={知识性, 授课质量, 作业友好,工作量适宜,易通过, 有趣},
    ymin=0,
    ymax=5,
    ytick={1,2,...,5},
    y grid style={gray},
    grid=both,
    minor grid style=gray,
    major grid style={gray},
]
\addplot[mark=*, ultra thick, black, data cs=polar] coordinates {
    (0,5)  % 知识性得分,这里修改为你的分数
    (60,3)  % 授课质量得分,这里修改为你的分数
    (120,5)  % 作业友好得分,这里修改为你的分数
    (180,5)  % 工作量得分,这里修改为你的分数
    (240,5)  % 易通过得分,这里修改为你的分数
    (300,5)  % 有趣得分,这里修改为你的分数
    (360,5)  % 为了闭合图形,最后的一个点应该和第一个点一样,这里修改为你的分数
};
\end{polaraxis}
\end{tikzpicture}
\end{minipage}%
\begin{minipage}{0.45\textwidth}
\raggedleft
\begin{tabular}{r|c}
\textbf{ } & \textbf{21-22} \\ \hline
通过率 &75\% \\ 
平均分 & 6.56 \\ 
参与学生数 &136 \\
学分 & 5 EC\\
\end{tabular}
\end{minipage}\\

这门课不在可选的列表里,想参加的话给coordinator发邮件简单讲述动机并提交Form3即可。是一门对于工科学生来说很好的入门经济课程。遗憾的是由于和必修课时间撞上,我没有实际参与过lecture,所以也无从评价授课质量。老师会提供很详细的lecture notes,完全足以理解本节课的内容。课程讲述了几个基本的宏观经济学原理,需要掌握并理解基本图表并且根据材料进行简单的计算。每节课有在bs的课后小测和作业,不计入总分但是对于了解本节课的考试重点很有帮助。对于没有参加lecture的我来说,这门课每周的学习时间在2-3小时。考试是闭卷但是可以携带一页A4纸的手写笔记,考试内容是纯选择题,包含概念理解和简单计算,通过和得到相对比较高的分数都不算困难。甚至由于没有平时分,考前突击也是可选项。由于是和本专业完全不相关的领域,所以学习过程中会出现一些生词,但是有一种看课外书的摸鱼错觉。总之还是很推荐的。另,本节课开设在Q4。

\begin{flushright}
刘彦菁; 08/06/2023
\end{flushright}