\subsection{ME Social Courses}
\subsubsection{WM0349WB Philosophy of Engineering Science and Design}
\begin{minipage}{0.45\textwidth}
\centering
\begin{tikzpicture}
\begin{polaraxis}[
    width=0.65\textwidth,
    height=0.65\textwidth,
    yticklabels={},
    xtick={0,60,...,300},
    xticklabels={知识性, 授课质量, 作业友好,工作量适宜,易通过, 有趣},
    ymin=0,
    ymax=5,
    ytick={1,2,...,5},
    y grid style={gray},
    grid=both,
    minor grid style=gray,
    major grid style={gray},
]
\addplot[mark=*, ultra thick, black, data cs=polar] coordinates {
    (0,3)  % 知识性得分,这里修改为你的分数
    (60,5)  % 授课质量得分,这里修改为你的分数
    (120,5)  % 作业友好得分,这里修改为你的分数
    (180,5)  % 工作量得分,这里修改为你的分数
    (240,3)  % 易通过得分,这里修改为你的分数
    (300,5)  % 有趣得分,这里修改为你的分数
    (360,3)  % 为了闭合图形,最后的一个点应该和第一个点一样,这里修改为你的分数
};
\end{polaraxis}
\end{tikzpicture}
\end{minipage}%
\begin{minipage}{0.45\textwidth}
\raggedleft
\begin{tabular}{r|c}
\textbf{指标} & \textbf{数据} \\ \hline
通过率 & \% \\ 
平均分 &  \\ 
参与学生数 &  \\
学分 & 3 EC\\
\end{tabular}
\end{minipage}\\
%待完成;王昊辰

\subsubsection{WM1401T/WM1402TU Ethics of Healthcare Technologies}

\begin{minipage}{0.45\textwidth}
\centering
\begin{tikzpicture}
\begin{polaraxis}[
    width=0.65\textwidth,
    height=0.65\textwidth,
    yticklabels={},
    xtick={0,60,...,300},
    xticklabels={知识性, 授课质量, 作业友好,工作量适宜,易通过, 有趣},
    ymin=0,
    ymax=5,
    ytick={1,2,...,5},
    y grid style={gray},
    grid=both,
    minor grid style=gray,
    major grid style={gray},
]
\addplot[mark=*, ultra thick, black, data cs=polar] coordinates {
    (0,2)  % 知识性得分,这里修改为你的分数
    (60,3)  % 授课质量得分,这里修改为你的分数
    (120,5)  % 作业友好得分,这里修改为你的分数
    (180,5)  % 工作量得分,这里修改为你的分数
    (240,5)  % 易通过得分,这里修改为你的分数
    (300,3)  % 有趣得分,这里修改为你的分数
    (360,2)  % 为了闭合图形,最后的一个点应该和第一个点一样,这里修改为你的分数
};
\end{polaraxis}
\end{tikzpicture}
\end{minipage}%
\begin{minipage}{0.45\textwidth}
\raggedleft
\begin{tabular}{r|c}
\textbf{指标} & \textbf{数据} \\ \hline
通过率 &100/100 \% \\ 
平均分 & 8.19/8.11 \\ 
参与学生数 &21/117 \\
学分 & 3/5 EC\\
\end{tabular}
\end{minipage}\\

医疗技术伦理有两个课程代码,WM1401TU为3学分,WM1420TU为5学分,WM1420TU需要在WM1401TU的基础上额外完成一个小论文,小论文可以以两到三人的小组完成。每周需要完成思考题的作业。必须参加两次case study的tutorial。
2023年的考试为线上考试,考试内容出自课后思考题,没有监考,可以在考试期间浏览任意材料。
总的来说就是一门送分课,并没有太大的学习意义,作业的负担不大,考试难度不大,基本都出在思考题,而且可以用ChatGPT来帮助你回答。建议在相对轻松的Q1上完伦理课从而减轻后续学期的学习压力。

\begin{flushright}
张先治; 07/06/2023
\end{flushright}

\subsubsection{EPA1223 Macro-economics for Policy Analysis}

\begin{minipage}{0.45\textwidth}
\centering
\begin{tikzpicture}
\begin{polaraxis}[
    width=0.65\textwidth,
    height=0.65\textwidth,
    yticklabels={},
    xtick={0,60,...,300},
    xticklabels={知识性, 授课质量, 作业友好,工作量适宜,易通过, 有趣},
    ymin=0,
    ymax=5,
    ytick={1,2,...,5},
    y grid style={gray},
    grid=both,
    minor grid style=gray,
    major grid style={gray},
]
\addplot[mark=*, ultra thick, black, data cs=polar] coordinates {
    (0,5)  % 知识性得分,这里修改为你的分数
    (60,3)  % 授课质量得分,这里修改为你的分数
    (120,5)  % 作业友好得分,这里修改为你的分数
    (180,5)  % 工作量得分,这里修改为你的分数
    (240,5)  % 易通过得分,这里修改为你的分数
    (300,5)  % 有趣得分,这里修改为你的分数
    (360,5)  % 为了闭合图形,最后的一个点应该和第一个点一样,这里修改为你的分数
};
\end{polaraxis}
\end{tikzpicture}
\end{minipage}%
\begin{minipage}{0.45\textwidth}
\raggedleft
\begin{tabular}{r|c}
\textbf{指标} & \textbf{数据21-22} \\ \hline
通过率 &75\% \\ 
平均分 & 6.56 \\ 
参与学生数 &136 \\
学分 & 5 EC\\
\end{tabular}
\end{minipage}\\

这门课不在可选的列表里,想参加的话给coordinator发邮件简单讲述动机并提交Form3即可。是一门对于工科学生来说很好的入门经济课程。遗憾的是由于和必修课时间撞上,我没有实际参与过lecture,所以也无从评价授课质量。老师会提供很详细的lecture notes,完全足以理解本节课的内容。课程讲述了几个基本的宏观经济学原理,需要掌握并理解基本图表并且根据材料进行简单的计算。每节课有基于bs的小测,不计入总分但是对于了解本节课的考试重点很有帮助。每周的学习时间在2-3小时。考试是闭卷但是可以携带一页A4纸的手写笔记,考试内容是纯选择题,包含概念理解和简单计算,通过和得到相对比较高的分数都不算困难。由于是和本专业完全不相关的领域,所以学习过程中生词可能比较多,但是有一种看课外书的摸鱼错觉。总之还是很推荐的。另,本节课开设在Q4。

\begin{flushright}
刘彦菁; 08/06/2023
\end{flushright}