\vspace{\betsubsec} %section间留白
\section{ME Social Courses}

\subsection[RO47008/RO47009 Robot and Society]{RO47008/RO47009 Robot and Society}\hypertarget{RO47008}{} 
\begin{minipage}{0.45\textwidth}
\centering
\begin{tikzpicture}
\begin{polaraxis}[
    width=0.65\textwidth,
    height=0.65\textwidth,
    yticklabels={},
    xtick={0,60,...,300},
    xticklabels={知识性, 授课质量, 作业友好,工作量适宜,易通过, 有趣},
    ymin=0,
    ymax=5,
    ytick={1,2,...,5},
    y grid style={gray},
    grid=both,
    minor grid style=gray,
    major grid style={gray},
]
\addplot[mark=*, ultra thick, black, data cs=polar] coordinates {
    (0,5)  % 知识性得分,这里修改为你的分数
    (60,5)  % 授课质量得分,这里修改为你的分数
    (120,5)  % 作业友好得分,这里修改为你的分数
    (180,5)  % 工作量得分,这里修改为你的分数
    (240,5)  % 易通过得分,这里修改为你的分数
    (300,5)  % 有趣得分,这里修改为你的分数
    (360,5)  % 为了闭合图形,最后的一个点应该和第一个点一样,这里修改为你的分数
};
\end{polaraxis}
\end{tikzpicture}
\end{minipage}%
\begin{minipage}{0.45\textwidth}
\raggedleft
\begin{tabular}{r|c}
\textbf{ } & \textbf{22-23} \\ \hline
通过率 & 无数据 \\ 
平均分 &  无数据\\ 
参与学生数 &  无数据\\
学分 & 3 EC\\
\end{tabular}
\end{minipage}\\

这门课有两个代码,一个3EC一个4EC,虽然培养方案里给的是四学分的课,但可以三学分的课也是可以认证的。这两个的区别就是会多一个报告,我个人觉得四学分很不划算。所以在这里推荐大家上三学分的,授课考试平时的讨论没有任何区别。

作业就是有几次讨论课,需要写一些简短的总结和反馈,不复杂并且不赋分,十分友好,参与就行。考试同理,开卷,可带资料,U盘以及使用谷歌等。考试就是根据材料回答问题。由于上这个课之前ChatGPT还没有出来,所以我不是很确定这种方式会不会变更,如若未变更,在这里强烈推荐这门课。

\begin{flushright}
熊俊彦; 17/07/2023
\end{flushright}

\subsection[WM0349WB Philosophy of Engineering Science and Design]{WM0349WB Philosophy of Engineering Science and Design}\hypertarget{WM0349WB}{} 
\begin{minipage}{0.45\textwidth}
\centering
\begin{tikzpicture}
\begin{polaraxis}[
    width=0.65\textwidth,
    height=0.65\textwidth,
    yticklabels={},
    xtick={0,60,...,300},
    xticklabels={知识性, 授课质量, 作业友好,工作量适宜,易通过, 有趣},
    ymin=0,
    ymax=5,
    ytick={1,2,...,5},
    y grid style={gray},
    grid=both,
    minor grid style=gray,
    major grid style={gray},
]
\addplot[mark=*, ultra thick, black, data cs=polar] coordinates {
    (0,3)  % 知识性得分,这里修改为你的分数
    (60,5)  % 授课质量得分,这里修改为你的分数
    (120,5)  % 作业友好得分,这里修改为你的分数
    (180,5)  % 工作量得分,这里修改为你的分数
    (240,5)  % 易通过得分,这里修改为你的分数
    (300,5)  % 有趣得分,这里修改为你的分数
    (360,3)  % 为了闭合图形,最后的一个点应该和第一个点一样,这里修改为你的分数
};
\end{polaraxis}
\end{tikzpicture}
\end{minipage}%
\begin{minipage}{0.45\textwidth}
\raggedleft
\begin{tabular}{r|c}
\textbf{ } & \textbf{22-23} \\ \hline
通过率 & 88.5\% \\ 
平均分 &  7.10\\ 
参与学生数 &  61\\
学分 & 3 EC\\
\end{tabular}
\end{minipage}\\

这门课,强烈推荐;笔者个人角度,觉得堪称整个第一年我所上过课里最有趣的。它分为两部分,Lecture和Seminar。其中Lecture并非强制,但个人觉得因授课质量极高,且非常有趣而十分值得。Seminar强制出勤,席间表现占到总分的40\%,每节都会由一组人进行主持(主持表现会占据一定分数),针对特定主题开展讨论,氛围很好。

如前所言,Lecture授课质量是极高的;老师是一位阿根廷人,英语完全没有什么口音,上课激情澎湃;并不是代尔夫特大多数老师那样,站在讲台上面,泛泛地念着PPT。他会经常走下讲台,在一排排学生走廊里走来又走去。上课的互动性很强,甚至主观毛估估一半内容都由学生提问临时引申涉及。

主要教授内容就是哲学,什么是科学?什么是工程?休谟疑问,Kuhn的科学观点等等等,会讲授百家之言,马克思哲学也占据相当大的内容。但是,这位讲师笔者政治光谱比较正统红色,对CCP时有锐评(其实他对各国都锐评),与大陆角度看他的看法略显政治不正确;如果这让你感觉不舒服,劝还是不要选了吧。

Seminar的主持,会在Seminar进行的前两周提供大概的文档和阅读材料。基于阅读材料如何展开,完全因组而异,自由度很高。依照22-23学年来看,大多数组都会通过摇色子或者抽卡的形式,在Seminar上针对某一问题对在场的人进行分组;而后开展辩论。Seminar完成后需要上交报告,内容涵盖如何主持的计划提纲,以及Seminar进行中的记录;推荐在主持Seminar的同时进行记录或者录音,防止遗忘。

这门课,除了在所主持Seminar的那周需要小组完成会议报告外,没有什么额外笔头的课后作业;但阅读量较高,平均每周需要阅读30页左右的哲学书籍;由于笔者个人觉得书籍内容十分有趣且引人入胜,完全课外书感觉,所以主观也并不会觉得压力很大。

考试的话,难度略高,完全简答题。老师会在考前几周上传模拟试卷等以供复习。虽说考试略难,但笔者主观怀疑评卷其实并不完全基于你考场所写。笔者在29/06/2023考试,考试所写至少也有数千词,30/06/2023下午成绩却已经公布;实在难以想象老师是怎样完成的评卷。结合他上课时常提及他根本不想组织考试,完全学院要求才迫不得已,实在略显可疑。
\begin{flushright}
王昊辰; 30/06/2023
\end{flushright}

\subsection{WM1401TU/WM1402TU Ethics of Healthcare Technologies}\hypertarget{WM1401TU}{} 
\begin{minipage}{0.45\textwidth}
\centering
\begin{tikzpicture}
\begin{polaraxis}[
    width=0.65\textwidth,
    height=0.65\textwidth,
    yticklabels={},
    xtick={0,60,...,300},
    xticklabels={知识性, 授课质量, 作业友好,工作量适宜,易通过, 有趣},
    ymin=0,
    ymax=5,
    ytick={1,2,...,5},
    y grid style={gray},
    grid=both,
    minor grid style=gray,
    major grid style={gray},
]
\addplot[mark=*, ultra thick, black, data cs=polar] coordinates {
    (0,2)  % 知识性得分,这里修改为你的分数
    (60,3)  % 授课质量得分,这里修改为你的分数
    (120,5)  % 作业友好得分,这里修改为你的分数
    (180,5)  % 工作量得分,这里修改为你的分数
    (240,5)  % 易通过得分,这里修改为你的分数
    (300,3)  % 有趣得分,这里修改为你的分数
    (360,2)  % 为了闭合图形,最后的一个点应该和第一个点一样,这里修改为你的分数
};
\end{polaraxis}
\end{tikzpicture}
\end{minipage}%
\begin{minipage}{0.45\textwidth}
\raggedleft
\begin{tabular}{r|c}
\textbf{ } & \textbf{22-23} \\ \hline
通过率 &100/100 \% \\ 
平均分 & 8.19/8.11 \\ 
参与学生数 &21/117 \\
学分 & 3/5 EC\\
\end{tabular}
\end{minipage}\\

医疗技术伦理有两个课程代码,WM1401TU为3学分,WM1420TU为5学分,WM1420TU需要在WM1401TU的基础上额外完成一个小论文,小论文可以以两到三人的小组完成。每周需要完成思考题的作业。必须参加两次Case Study的Tutorial。

2023年的考试为线上考试,考试内容出自课后思考题,没有监考,可以在考试期间浏览任意材料。
总的来说就是一门送分课,并没有太大的学习意义,作业的负担不大,考试难度不大,基本都出在思考题,而且可以用ChatGPT来帮助你回答。建议在相对轻松的Q1上完伦理课从而减轻后续学期的学习压力。
\begin{flushright}
张先治; 07/06/2023
\end{flushright}

