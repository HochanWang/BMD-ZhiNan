\vspace{\betsubsec} %section间留白
\subsection{ME Social Courses}
\subsubsection{WM0349WB Philosophy of Engineering Science and Design}
\begin{minipage}{0.45\textwidth}
\centering
\begin{tikzpicture}
\begin{polaraxis}[
    width=0.65\textwidth,
    height=0.65\textwidth,
    yticklabels={},
    xtick={0,60,...,300},
    xticklabels={知识性, 授课质量, 作业友好,工作量适宜,易通过, 有趣},
    ymin=0,
    ymax=5,
    ytick={1,2,...,5},
    y grid style={gray},
    grid=both,
    minor grid style=gray,
    major grid style={gray},
]
\addplot[mark=*, ultra thick, black, data cs=polar] coordinates {
    (0,3)  % 知识性得分,这里修改为你的分数
    (60,5)  % 授课质量得分,这里修改为你的分数
    (120,5)  % 作业友好得分,这里修改为你的分数
    (180,5)  % 工作量得分,这里修改为你的分数
    (240,3)  % 易通过得分,这里修改为你的分数
    (300,5)  % 有趣得分,这里修改为你的分数
    (360,3)  % 为了闭合图形,最后的一个点应该和第一个点一样,这里修改为你的分数
};
\end{polaraxis}
\end{tikzpicture}
\end{minipage}%
\begin{minipage}{0.45\textwidth}
\raggedleft
\begin{tabular}{r|c}
\textbf{指标} & \textbf{数据} \\ \hline
通过率 & \% \\ 
平均分 &  \\ 
参与学生数 &  \\
学分 & 3 EC\\
\end{tabular}
\end{minipage}\\

这门课,强烈推荐;笔者个人角度,觉得堪称整个第一年我所上过课里最有趣的。它分为两部分,Lecture和Seminar。其中lecture并非强制,但个人觉得因为授课质量很高,非常有趣十分值得;Seminar强制出勤,每节都会由一组人进行主持(主持表现会占据一定分数),针对特定主题开展讨论,氛围很好。

如前所言,Lecture授课质量是极高的;老师是一位阿根廷人,英语完全没有什么口音,上课激情澎湃;并不是代尔夫特大多数老师那样,站在讲台上面,泛泛地讲着ppt。他会经常走下讲台,在一排排学生走廊里走来又走去。上课的互动性很强,甚至主观毛估估一半内容都由学生临时引申涉及。主要教授内容就是哲学,什么是科学?什么是工程?会讲授百家之言,马克思哲学也占据相当大的内容。但是,这位讲师笔者个人鉴定,政治光谱比较正统红色,对CCP看法比较政治不正确;如果这让你感觉不舒服,劝还是不要选了吧。

而Seminar的主持,会在Seminar进行的前两周提供大概的文档和阅读材料。基于阅读材料如何展开,完全因组而异,自由度很高。依照22-23学年来看,大多数组都会通过摇色子或者抽卡的形式,在Seminar上针对某一问题对在场的人进行分组;而后开展辩论。Seminar完成后需要上交报告,内容涵盖如何主持的计划提纲,以及Seminar进行中的记录;推荐在主持Seminar的同时进行记录或者录音,防止遗忘。

\begin{flushright}
王昊辰; 17/06/2023
\end{flushright}

\subsubsection{WM1401T/WM1402TU Ethics of Healthcare Technologies}
\begin{minipage}{0.45\textwidth}
\centering
\begin{tikzpicture}
\begin{polaraxis}[
    width=0.65\textwidth,
    height=0.65\textwidth,
    yticklabels={},
    xtick={0,60,...,300},
    xticklabels={知识性, 授课质量, 作业友好,工作量适宜,易通过, 有趣},
    ymin=0,
    ymax=5,
    ytick={1,2,...,5},
    y grid style={gray},
    grid=both,
    minor grid style=gray,
    major grid style={gray},
]
\addplot[mark=*, ultra thick, black, data cs=polar] coordinates {
    (0,2)  % 知识性得分,这里修改为你的分数
    (60,3)  % 授课质量得分,这里修改为你的分数
    (120,5)  % 作业友好得分,这里修改为你的分数
    (180,5)  % 工作量得分,这里修改为你的分数
    (240,5)  % 易通过得分,这里修改为你的分数
    (300,3)  % 有趣得分,这里修改为你的分数
    (360,2)  % 为了闭合图形,最后的一个点应该和第一个点一样,这里修改为你的分数
};
\end{polaraxis}
\end{tikzpicture}
\end{minipage}%
\begin{minipage}{0.45\textwidth}
\raggedleft
\begin{tabular}{r|c}
\textbf{指标} & \textbf{数据} \\ \hline
通过率 &100/100 \% \\ 
平均分 & 8.19/8.11 \\ 
参与学生数 &21/117 \\
学分 & 3/5 EC\\
\end{tabular}
\end{minipage}\\

医疗技术伦理有两个课程代码,WM1401TU为3学分,WM1420TU为5学分,WM1420TU需要在WM1401TU的基础上额外完成一个小论文,小论文可以以两到三人的小组完成。每周需要完成思考题的作业。必须参加两次case study的tutorial。
2023年的考试为线上考试,考试内容出自课后思考题,没有监考,可以在考试期间浏览任意材料。
总的来说就是一门送分课,并没有太大的学习意义,作业的负担不大,考试难度不大,基本都出在思考题,而且可以用ChatGPT来帮助你回答。建议在相对轻松的Q1上完伦理课从而减轻后续学期的学习压力。

\begin{flushright}
张先治; 07/06/2023
\end{flushright}

