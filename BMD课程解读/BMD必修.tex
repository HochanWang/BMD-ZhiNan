\vspace{\betsubsec} %section间留白
\subsection{ME-BMD Obligatory Courses}
\subsubsection{BM41040 Neuromechanics \& Motor Control}
\begin{minipage}{0.45\textwidth}
\centering
\begin{tikzpicture}
\begin{polaraxis}[
    width=0.65\textwidth,
    height=0.65\textwidth,
    yticklabels={},
    xtick={0,60,...,300},
    xticklabels={知识性, 授课质量, 作业友好,工作量适宜,易通过, 有趣},
    ymin=0,
    ymax=5,
    ytick={1,2,...,5},
    y grid style={gray},
    grid=both,
    minor grid style=gray,
    major grid style={gray},
]
\addplot[mark=*, ultra thick, black, data cs=polar] coordinates {
    (0,4)  % 知识性得分,这里修改为你的分数
    (60,3)  % 授课质量得分,这里修改为你的分数
    (120,3)  % 作业友好得分,这里修改为你的分数
    (180,4)  % 工作量得分,这里修改为你的分数
    (240,3)  % 易通过得分,这里修改为你的分数
    (300,3)  % 有趣得分,这里修改为你的分数
    (360,4)  % 为了闭合图形,最后的一个点应该和第一个点一样,这里修改为你的分数
};
\end{polaraxis}
\end{tikzpicture}
\end{minipage}%
\begin{minipage}{0.45\textwidth}
\raggedleft
\begin{tabular}{r|c}
\textbf{指标} & \textbf{数据21-22} \\ \hline
通过率 & 68.7\% \\ 
平均分 &  6.3\\ 
参与学生数 &  57\\
学分 & 5 EC\\
\end{tabular}
\end{minipage}\\

\subsubsection{ME41006 Musculoskeletal Modeling and Simulation}
\begin{minipage}{0.45\textwidth}
\centering
\begin{tikzpicture}
\begin{polaraxis}[
    width=0.65\textwidth,
    height=0.65\textwidth,
    yticklabels={},
    xtick={0,60,...,300},
    xticklabels={知识性, 授课质量, 作业友好,工作量适宜,易通过, 有趣},
    ymin=0,
    ymax=5,
    ytick={1,2,...,5},
    y grid style={gray},
    grid=both,
    minor grid style=gray,
    major grid style={gray},
]
\addplot[mark=*, ultra thick, black, data cs=polar] coordinates {
    (0,3)  % 知识性得分,这里修改为你的分数
    (60,2)  % 授课质量得分,这里修改为你的分数
    (120,1)  % 作业友好得分,这里修改为你的分数
    (180,1)  % 工作量得分,这里修改为你的分数
    (240,3)  % 易通过得分,这里修改为你的分数
    (300,4)  % 有趣得分,这里修改为你的分数
    (360,3)  % 为了闭合图形,最后的一个点应该和第一个点一样,这里修改为你的分数
};
\end{polaraxis}
\end{tikzpicture}
\end{minipage}%
\begin{minipage}{0.45\textwidth}
\raggedleft
\begin{tabular}{r|c|c}
\textbf{指标} & \textbf{数据22-23} & \textbf{数据21-22}\\ \hline
通过率 & \% & 100\%\\ 
平均分 &   & 7.08\\ 
参与学生数 &  & 91\\
学分 & 4 EC & 3 EC\\
\end{tabular}
\end{minipage}\\

这门课,和ME41055的老师一样,也是最近几年(2023年角度看)才从美国来到TUD(这个老师来自斯坦福);同样类似的,是这两门课的Workload 也都格外的高。其长度仅仅一个Q,但于此期间,需在前3周,以每周一个的速度完成掉三支作业;前三周的负荷,基本刚刚就是赶上一个Deadline,就又被堆下来了另一个,非常的夸张。

2022-2023学年而言,前两支作业都是非常开放的研究型问题;第一个为Passive Walker, 需要你在1周内,优化使其能在崎岖路面行走;走得越远,分数越高;除此还要回答理论性的题目和录制一个2分钟的presentation,非常耗费时间,加上其他课,每天甚至能睡4小时;一周下来,绝对感慨自己炎黄超人。第二个作业稍好些,是优化一个肌肉使其在拔河比赛中尽可能战胜对手。而第三个作业,比较轻松,循规蹈矩按照说明操作就好。

在完成三个作业以后,还会有一个最终的研究任务。很多研究课题可以选择,需要在三周时间内做完,时间也比较紧张:第一周上交proposal,第二周上传一个5分钟的presentation,第三周写完论文;猪突猛进直呼逆天。而且论文和presentation上交的两周,是与考试周与复习周重合的。笔者私下感觉,最好在最终作业刚颁布,甚至还没正式确定时候,就抓紧研究,速战速决。

但是,比较宽慰的是,这门课是没有考试的。也就是说,平时若是愿意花时间死磕作业,成绩应该不会难看。笔者的三次作业,都已经上传,\href{https://drive.google.com/drive/folders/1mJun-EmYGX1DovFISP9lXZNkdnutQpuh?usp=sharing}{\uline{于此链接可以查看}},希望有点用处吧。
\begin{flushright}
王昊辰; 16/06/2023
\end{flushright}


\subsubsection{RO47006 Human Robot Interaction}
\begin{minipage}{0.45\textwidth}
\centering
\begin{tikzpicture}
\begin{polaraxis}[
    width=0.65\textwidth,
    height=0.65\textwidth,
    yticklabels={},
    xtick={0,60,...,300},
    xticklabels={知识性, 授课质量, 作业友好,工作量适宜,易通过, 有趣},
    ymin=0,
    ymax=5,
    ytick={1,2,...,5},
    y grid style={gray},
    grid=both,
    minor grid style=gray,
    major grid style={gray},
]
\addplot[mark=*, ultra thick, black, data cs=polar] coordinates {
    (0,4)  % 知识性得分,这里修改为你的分数
    (60,4)  % 授课质量得分,这里修改为你的分数
    (120,5)  % 作业友好得分,这里修改为你的分数
    (180,5)  % 工作量得分,这里修改为你的分数
    (240,5)  % 易通过得分,这里修改为你的分数
    (300,4)  % 有趣得分,这里修改为你的分数
    (360,4)  % 为了闭合图形,最后的一个点应该和第一个点一样,这里修改为你的分数
};
\end{polaraxis}
\end{tikzpicture}
\end{minipage}%
\begin{minipage}{0.45\textwidth}
\raggedleft
\begin{tabular}{r|c}
\textbf{指标} & \textbf{数据} \\ \hline
通过率 & 100\% \\ 
平均分 & 7.36 \\ 
参与学生数 & 193 \\ 
学分 & 5 EC\\
\end{tabular}
\end{minipage}\\

这门课,其实相当于一大堆知识的大杂烩;完全是把诸多杂七杂八的研究领域糅杂一起的产物。

它所教授的内容,总得概括,大概三部分吧:其一是把人体看作一个控制机构,由此也会衍生2个作业。第二部分,就是人机交互了,会讲述这方面的一些历史,从人去适应机器到机器适应人。最后一部分,是一系列前沿的研究,会请到TUD各个研究室的人来客座。

我个人来讲,也可能我一直以来都蛮文科的,是蛮享受这门课的;但似乎我看荷兰人风评,他们又大多觉得挺扯淡的;无论如何,通过率都摆在了这里,实在良心必修课。

这课的评分主要是2方面;作业及考试。作业的P1和P1B部分为小组完成,而P2是个人。P2的期限与寒假重合,作业上讲,工作量十分怡人,是一门并不会感到压力的课程。

考试的话,是开卷的。允许你携带任意数量的资料。似乎几乎每个人都是3,400页资料起步;基本人均缩放打印了全部PPT。考题分为2部分,选择与后面的开放问题,两者都会就课上讲到的某些犄角旮旯知识大做文章。个人建议在上课时候就在PPT上做好标注,提前构建一个类似于知识点字典一样的目录,方便考试时候快速翻阅查找。

P1的A与B两个作业因为是小组完成,我这里就不提供啦。 但P2,虽然似乎每年都略有不同,但我还是提供一下通往我2022年所做\href{https://drive.google.com/file/d/1GwwX7ZxEE8R7JeDm6NcHQu_IA34ZqyRe/view?usp=sharing}{\uline{的链接吧}}。这里我的第一题计算并不准确,除此以外分数都还是在8分以上的。

而且需要注意的是,P2作业有些非常犄角旮旯的要求,比如引用就会有些细则;还请详细阅读要求啊。这些规定不符合的话,听说会有直接0分的情况。
\begin{flushright}
王昊辰; 02/06/2023
\end{flushright}


